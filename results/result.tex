
Following chapters describe the results we gathered during the evaluation of our models. All results are generated using Python version 3.6.0\footnote{\url{https://www.python.org/} Accessed 14th May 2018} and scikit-learn version 0.19.1\footnote{\url{http://scikit-learn.org/stable/} Accessed 14th May 2018}. 

As explained in the Chapter \ref{chap-method-evaluation}, we first evaluate both models individually and lastly combine the results to create a final prediction which is evaluated by a human expert. Our similarity detection is trained with SOCO data set and authorship identification with OHPE and OHJA without using the exams. A summary of these exam tasks is given below.

\begin{table}[ht]
\centering
\caption{Submission count and average line count for exam tasks. A refers to OHPE and B for OHJA.}
\label{tbl-exam-data}
\begin{tabular}{|c||c|c|c|c|c|c|c|}
\hline
\bf Task        & 1.A & 2.A & 3.A & 4.A & 1.B & 2.B & 3.B \\ \hline
\bf Submissions & 244 & 242 & 227 & 240 & 200 & 198 & 197 \\ \hline
\bf Avg. LOC    & 37  & 39  & 47  & 110 & 160 & 86  & 150 \\ \hline
\end{tabular}
\end{table}

\noindent
It's clear from the Table \ref{tbl-exam-data} that OHJA's tasks are more longer than OHPE's. Some of the tasks of OHPE's exam have a very low average line count that creates a challenge for the detection.

\subsection{Document similarity} \label{chap-sd-result}
% macro-averaged results

We start evaluating our similarity detection by tuning the hyperparameters $n$ for $n$-gram length and $\varepsilon$ for the epsilon-range \ie minimum distance to other documents. Results are gained by turning all documents into binary vector based on the SOCO labels \ie vector $\bolditt{y}$ where $y_i = 1$ and $y_j = 1$ if $i$th and $j$th documents are reported as plagiarized pairs. Our predictions are then compared to this golden standard.

Table \ref{tbl-sd-socot-fone} shows averaged $F_1$-score, weighted by label counts, for the SOCO-T data. One can see from it that the $F_1$-score is highest when $n \in [4, 7]$ and $\varepsilon \in [0.4, 0.6]$. However, allowing 40-50\% dissimilarity between documents means that there is a high chance for false-positives, especially when submissions are relatively short and the task is well-defined like in OHPE and OHJA, meaning that the solution space for a given task can be limited. Therefore to avoid over-fitting similarity detection to SOCO's training data we use also two test sets of SOCO.   

\newpage

\begin{table}[ht]
\centering
\caption{Average $F_1$-score for $n$-gram length and $\varepsilon$-range for SOCO-T containing 115 cases of plagiarism. The smaller the $\varepsilon$-range is, the more similar documents have to be. $F_1$-scores close or over 0.8 are bolded.}
\label{tbl-sd-socot-fone}
\scalebox{0.75}{
    \def\arraystretch{1.5}
    \begin{tabular}{|c||c|c|c|c|c|c|c|c|c|c|} \hline
    \backslashbox{\bf Epsilon}{\bf $N$-gram} & 1 & 2 & 3 & 4 & 5 & 6 & 7 & 8 & 9 & 10 \\ \hline
    0.1 & 0.31  & 0.69  & 0.63  & 0.60  & 0.59  & 0.56 & 0.55  & 0.55   & 0.52  & 0.52   \\ \hline
    
    0.2 & 0.28  & 0.59  & 0.73  & 0.66  & 0.63  &  0.62  & 0.60  & 0.59  & 0.56  &  0.55 \\\hline
    
    0.3 &  0.27  & 0.43  & \bf 0.78 & 0.73  & 0.70  & 0.67  & 0.64 & 0.63 & 0.59  & 0.58   \\ \hline
    
    0.4 & 0.27  & 0.31  & 0.72  & \bf 0.81  & \bf 0.78  & 0.72   &  0.71  & 0.69  & 0.65  & 0.64  \\ \hline
    
    0.5 & 0.27  & 0.29  & 0.57  & \bf 0.80  & \bf 0.81  & \bf 0.80  & \bf 0.81  & \bf 0.78 &  0.77   & 0.74   \\ \hline
    
    0.6 & 0.27  & 0.27  & 0.39  & 0.71  & \bf 0.83  & \bf 0.89  & \bf 0.90  &  \bf 0.86  & \bf 0.85 & \bf 0.85   \\ \hline
    
    \end{tabular}
}
\end{table}




\begin{table}[ht]
\centering
\caption{Precision with respect to plagiarized class, ranging various $n$-gram lengths and $\varepsilon$-ranges for SOCO-T. Values close or over 0.9 are bolded.}
\label{tbl-sd-soco-prec}
\scalebox{0.75}{
    \def\arraystretch{1.5}
    \begin{tabular}{|c||c|c|c|c|c|c|c|c|c|c|} \hline
    \backslashbox{\bf Epsilon}{\bf $n$-gram} 
        & 1 & 2 & 3 & 4 & 5 & 6 & 7 & 8 & 9 & 10 \\ \hline
    0.1 & 0.45  & 0.77  & \bf 1.00  & \bf 1.00  & \bf 1.00  & \bf 1.00 & \bf 1.00  & \bf 1.00   & \bf 1.00  & \bf 1.00   \\ \hline
    
    0.2 & 0.45  & 0.53  & \bf 0.98  & \bf 1.00  & \bf 1.00  &  \bf 1.00  & \bf 1.00  & \bf 1.00  & \bf 1.00  &  \bf 1.00 \\\hline
    
    0.3 &  0.44  & 0.48  &  0.83 & \bf 1.00  & \bf 1.00  & \bf 1.00  & \bf 1.00 & \bf 1.00 & \bf 1.00  & \bf 1.00  \\ \hline
    
    0.4 & 0.44  & 0.45  & 0.63  & \bf 0.87  & \bf 0.97  & \bf 0.98  &  \bf 0.98  & \bf 1.00  & \bf 1.00 & \bf 1.00  \\ \hline
    
    0.5 & 0.44  & 0.45  & 0.54  & 0.75  & \bf 0.90  &  \bf 0.92  &  \bf 0.97  & \bf 0.98 &  \bf 1.00   & \bf 1.00   \\ \hline
    
    0.6 & 0.44  & 0.44  & 0.47  & 0.62  & 0.77  & \bf 0.87  & \bf 0.94  & \bf 0.93  & \bf 0.95 & \bf 0.96   \\ \hline
    
    \end{tabular}
}
\end{table}

\noindent
We see from the Table \ref{tbl-sd-soco-prec}, that as we grow the number of $n$-grams, the precision starts converging to 1.00. Having a high precision means that the set of retrieved documents contains high number of true positives, as we have effectively minimized the amount of false positives, and no document is falsely accused of plagiarism. This happens because longer $n$-grams grow the size of vocabulary $\mathbb{V}$, thus making already dissimilar documents even more dissimilar and allowing the threshold to grow. The most smallest $n$-gram having a near perfect precision over plagiarized class is when $n=3$ and $\varepsilon \in [0.1, 0.2]$. This kind of high similarity value ranging between 80-99\% is also used in other studies \cite{AASCPD2012, OTIOLSS2015, Heblikar2015NormalizationBS, BUAA2009}. 

One sees from the following table that the $F_1$-score starts to deteriorate in all cases, when no plagiarism occurs in a set of documents. One must either have a high similarity threshold or increase the $n$-gram length to get a high $F_1$-score, because having a low threshold quickly introduces false positives. The model thus becomes too sensitive and retrieves documents where similarity has occurred naturally, adding work for the human expert who must go through the detected pairs and label them again. 


\begin{table}[ht]
\centering
\caption{$F_1$-score for SOCO-C1, which contains no cases of plagiarism. False-positives are  introduced as the threshold gets lower.}
\label{tbl-sd-sococ1-fone}
\scalebox{0.75}{
    \def\arraystretch{1.5}
    \begin{tabular}{|c||c|c|c|c|c|c|c|c|c|c|}
    \hline
    \backslashbox{\bf Epsilon}{\bf $n$-gram}    & 1    & 2    & 3    & 4    & 5    & 6    & 7    & 8    & 9    & 10   \\ \hline
    0.1 & 0.24 & \bf 0.94 & \bf 0.99 & \bf 0.99 & \bf 0.99 & \bf 0.99 & \bf 0.99 & \bf 0.99 & \bf 0.99 & \bf 0.99 \\ \hline
    0.2 & 0.11 & 0.56 & \bf 0.98 & \bf 0.99 & \bf 0.99 & \bf 0.99 & \bf 0.99 & \bf 0.99 & \bf 0.99 & \bf 0.99 \\ \hline
    0.3 & 0.06 & 0.38 & \bf 0.95 & \bf 0.99 & \bf 0.98 & \bf 0.99 & \bf 0.99 & \bf 0.99 & \bf 0.99 & \bf 0.99 \\ \hline
    0.4 & 0.03 & 0.20  & \bf 0.87 & \bf 0.98 & \bf 0.98 & \bf 0.98 & \bf 0.98 & \bf 0.98 & \bf 0.98 & \bf 0.98 \\ \hline
    0.5 & 0.03 & 0.16 & 0.59 & \bf 0.95 & \bf 0.98 & \bf 0.98 & \bf 0.98 & \bf 0.98 & \bf 0.98 & \bf 0.98 \\ \hline
    0.6 & 0.02 & 0.08 & 0.29 & \bf 0.88 & \bf 0.96 & \bf 0.98 & \bf 0.98 & \bf 0.98 & \bf 0.98 & \bf 0.98 \\ \hline
    \end{tabular}
}
\end{table}


\begin{table}[ht]
\centering
\caption{$F_1$-score for SOCO-C2, which contains 28 cases of plagiarism.}
\label{tbl-sd-sococ2-fone}
\scalebox{0.75}{
   \def\arraystretch{1.5}
   \begin{tabular}{|c||c|c|c|c|c|c|c|c|c|c|}
    \hline
     \backslashbox{\bf Epsilon}{\bf $n$-gram}     & 1    & 2    & 3    & 4    & 5    & 6    & 7    & 8    & 9    & 10   \\ \hline
    0.1 & 0.34 & \bf 0.92 & \bf 1.00 & \bf 1.00 & \bf 1.00 & \bf 1.00 & \bf 1.00 & \bf 1.00 & \bf 1.00 & \bf 1.00 \\ \hline
    0.2 & 0.27 & 0.57 & \bf 1.00 & \bf 1.00 & \bf 1.00 & \bf 1.00 & \bf 1.00 & \bf 1.00 & \bf 1.00 & \bf 1.00 \\ \hline
    0.3 & 0.20 & 0.38 & \bf 0.92 & \bf 1.00 & \bf 1.00 & \bf 1.00 & \bf 1.00 & \bf 1.00 & \bf 1.00 & \bf 1.00 \\ \hline
    0.4 & 0.15 & 0.31 & 0.75 & \bf 0.97 & \bf 0.97 & \bf 0.99 & \bf 0.99 & \bf 0.99 & \bf 0.99 & \bf 1.00 \\ \hline
    0.5 & 0.15 & 0.27 & 0.47 & \bf 0.91 & \bf 0.97 & \bf 0.97 & \bf 0.97 & \bf 0.97 & \bf 0.99 & \bf 0.99 \\ \hline
    0.6 & 0.15 & 0.22 & 0.33 & 0.78 & \bf 0.92 & \bf 0.97 & \bf 0.97 & \bf 0.97 & \bf 0.97 & \bf 0.97 \\ \hline
    \end{tabular}
}
\end{table}

\noindent
As in Table \ref{tbl-sd-sococ1-fone}, Table \ref{tbl-sd-sococ2-fone} shows that having $n=3$ with similarity threshold being around 80\%, yields one of the highest $F_1$-score with the lowest $n$ used. 

Taking the best scoring models over all scores for each $n$-gram and excluding $n \geq 8$ as they aren't improving the performance compared to $n=7$, we end up with five model candidates A ($n=3, \varepsilon=0.2$), B ($n=4, \varepsilon=0.4$), C ($n=5, \varepsilon=0.5$), D ($n=6, \varepsilon=0.6$), E ($n=7, \varepsilon=0.6$). As we compare these five models by their relative size of the largest cluster across all formed clusters of OHPE's exam tasks, we see in Figure \ref{fig-sd-clust-size} that in majority of the cases model A's largest cluster has the lowest relative size. In Figure \ref{fig-sd-clust-size:a} models B and D have relative size close to 1.0, meaning that the largest cluster contains almost every single retrieved document. The most probable cause for this is that all submissions share natural similarity, caused by the restricted task description or that the solution space for a given task might be very limited. Thus this kind of super cluster can contain a lot of false positives in form of similar documents which are not necessarily plagiarized, but rather correct similar solutions for the task.

\newpage

\begin{figure}[ht] 
  \begin{subfigure}[b]{0.5\linewidth}
    \centering
    \setlength\figureheight{3.5cm}
    \setlength\figurewidth{\textwidth}
    \input{plots/result/SD/ohpe_sd_cluster_size1.tikz}
    \caption{OHPE 1st exam task.} 
    \label{fig-sd-clust-size:a} 
    \vspace{4ex}
  \end{subfigure}%% 
  \begin{subfigure}[b]{0.5\linewidth}
    \centering
    \setlength\figureheight{3.5cm}
    \setlength\figurewidth{\textwidth}
    \input{plots/result/SD/ohpe_sd_cluster_size2.tikz}
    \caption{OHPE 2nd exam task.} 
    \label{fig-sd-clust-size:b} 
    \vspace{4ex}
  \end{subfigure}
   \begin{subfigure}[b]{0.5\linewidth}
    \centering
    \setlength\figureheight{3.5cm}
    \setlength\figurewidth{\textwidth}
    \input{plots/result/SD/ohpe_sd_cluster_size3.tikz}
    \caption{OHPE 3rd exam task.} 
    \label{fig-sd-clust-size:c} 
    \vspace{4ex}
  \end{subfigure}%% 
  \begin{subfigure}[b]{0.5\linewidth}
    \centering
    \setlength\figureheight{3.5cm}
    \setlength\figurewidth{\textwidth}
    \input{plots/result/SD/ohpe_sd_cluster_size4.tikz}
    \caption{OHPE 4th exam task.} 
    \label{fig-sd-clust-size:d} 
    \vspace{4ex}
  \end{subfigure} 
\caption{Relative size of the largest cluster in OHPE.}
\label{fig-sd-clust-size}
\end{figure}



\begin{figure}[!h] 
  \centering
  \begin{subfigure}[b]{0.5\linewidth}
    \setlength\figureheight{3.5cm}
    \setlength\figurewidth{\textwidth}
    \input{plots/result/SD/ohja_sd_cluster_size1.tikz}
    \caption{OHJA 1st exam task.} 
    \label{fig-sd-clust-size-ohja:a} 
    \vspace{4ex}
  \end{subfigure}%
  \begin{subfigure}[b]{0.5\linewidth}
    \setlength\figureheight{3.5cm}
    \setlength\figurewidth{\textwidth}
    \input{plots/result/SD/ohja_sd_cluster_size2.tikz}
    \caption{OHJA 2nd exam task.} 
    \label{fig-sd-clust-size-ohja:b} 
    \vspace{4ex}
  \end{subfigure}
   \begin{subfigure}[b]{0.5\linewidth}
    \setlength\figureheight{3.5cm}
    \setlength\figurewidth{\textwidth}
    \input{plots/result/SD/ohja_sd_cluster_size3.tikz}
    \caption{OHJA 3rd exam task.} 
    \label{fig-sd-clust-size-ohja:c} 
    %\vspace{4ex}
  \end{subfigure}
\caption{Relative size of the largest cluster in OHJA.}
\label{fig-sd-clust-size-ohja}
\end{figure}

Figure \ref{fig-sd-clust-size-ohja} shows same results for OHJA, which is an advanced course where a lot more programming skills are required from the students. This allows the tasks to be more difficult and longer, and as we see, the relative size has gone down in all cases compared to OHJA. Cluster sizes are now a lot smaller as exam tasks are more open-ended and more demanding. In other words, exam tasks have a range of multiple solutions and ways to do them, which minimizes the natural similarity between documents. This can be seen as a trend where none of the models now suffer from forming clusters containing majority of the retrieved documents, as the size is around 0.5 (50\%) at maximum. 


Retrieving majority of the documents is not optimal for plagiarism detection as these documents often needs to be manually inspected at the end. When inspecting the retrieval rate \ie 

\begin{equation}
    \text{Retrieval rate} = \dfrac{\#\text{Documents retrieved}}{\#\text{Documents in total}}
\end{equation}

\noindent
We can see that this rate should not get values near 1.0, as this would mean that all documents are detected as plagiarism, which is very unlikely in cases with hundreds of documents. In other words, models with a very high retrieval rates are almost guaranteed to have a high number of false positives unless there are only a small set of documents which are all plagiarized from one another. 

The retrieval rates of all models can be seen from the Figure \ref{fig-sd-retrieval-rate}.

\begin{figure}[ht] 
  \centering
  \begin{subfigure}[b]{0.8\linewidth}
    \centering
    \setlength\figureheight{5cm}
    \setlength\figurewidth{\textwidth}
    \input{plots/result/SD/ohpe_retrieval_rate.tikz}
    \caption{OHPE's exam tasks and the retrieval rates. Most models consider around 60-70\% of all document to be over the similarity threshold.} 
    \label{fig-sd-retrieval-rate:a} 
    \vspace{1ex}
  \end{subfigure}
  
  \begin{subfigure}[b]{0.8\linewidth}
    \centering
    \setlength\figureheight{5cm}
    \setlength\figurewidth{\textwidth}
    \input{plots/result/SD/ohja_retrieval_rate.tikz}
    \caption{OHJA's exam tasks and the retrieval rates. Models have similar retrieval rates and the rate of retrieval is low.} 
    \label{fig-sd-retrieval-rate:b} 
  \end{subfigure}
\caption{Retrieval rates of all models across every exam task. The model A keeps the lowest retrieval rate overall.}
\label{fig-sd-retrieval-rate}
\end{figure}

\noindent
The Figure \ref{fig-sd-retrieval-rate:a} shows how the the model A keeps the retrieval rate lowest around 50\%, meaning that half of the documents contain too much similarities between each others, and as said before it's very unlike so many documents are plagiarized. When comparing model A to other models, they claim the partition to be even higher, which is certainly not true. Interestingly in Figure \ref{fig-sd-retrieval-rate:b}, the models agree quite well, only having some level of disagreement with first exam exercise of OHJA. The second exercise shows good agreement, as all models have near 5\% retrieval rate.

Results on similarity detection show that tuning the two parameters $n$ and $\varepsilon$ is very data dependent as choosing the best performing combination might lead to very different results for other data sets. In our case, we choose the model A ($n=3, \varepsilon=0.2$) for the final evaluation, because that model had a decent $F_1$-score in SOCO-T, the precision for SOCO-T was nearly perfect, $F_1$ for both SOCO-C1 and SOCO-C2 were near 1.00. The model A also kept the largest cluster relatively small compared to other models and the retrieval rate for both OHPE and OHJA was the lowest, implying it could maintain a low rate of false positives. Keeping the rate of false positives minimal is more valuable us than retrieving every single plagiarism case, so we allow the model's detection rate to suffer with the benefit of having a high precision.  

To get perspective of how well our chosen model compares to the state of the art Java plagiarism detection tools, we first run JPlag detection for OHPE's and OHJA's exam tasks, then run our model for the same set of exercises and finally report the Jaccard similarity between the set of detected documents. For the JPlag, we use its default parameters and collect all document pairs where reported similarity score is above the same threshold as our model's $\varepsilon$-range, which in practice this means all documents where the reported similarity is over 80\% are collected. Following tables show results for both OHPE and OHJA with five metrics: documents detected by JPlag, documents detected by our chosen model, size of the intersection between the set of detected documents, number of unique documents retrieved in total and the Jaccard similarity score.

\begin{table}[ht]
\centering
\caption{Retrieval metrics for model A compared to JPlag with OHPE's exam tasks.}
\begin{tabular}{|c||c|c|c|c|}
\hline
\bf Exercise & 1. & 2. & 3. & 4. \\ \hline
\bf JPlag - Documents retrieved & 127 & 134 & 106 & 156 \\ \hline
\bf Model A - Documents retrieved & 109 & 130 & 111 & 114\\ \hline
\bf Common documents & 98  & 109 & 95 & 102\\ \hline
\bf Unique documents & 138 & 155 & 122 & 168\\ \hline
\bf Jaccard similarity    & 0.71  & 0.70  & 0.78  & 0.61  \\ \hline
\end{tabular}
\label{tbl-jacc-sd-ohpe}
\end{table}

\noindent
Table \ref{tbl-jacc-sd-ohpe} shows how our model agrees quite well with JPlag, as around 100 documents per exam task are shared. But even with the state of the art tool like JPlag one retrieves a lot of documents with a high threshold like 80\% for OHPE, as the retrieval rate with JPlag for all OHPE's tasks is around 50\%. This implies that even JPlag introduces false positives for restricted tasks, and minimizing false positives is a problem for every plagiarism detection tool.   

\begin{table}[ht]
\centering
\caption{Retrieval metrics for model A compared to JPlag with OHJA's exam tasks. JPlag retrieves just a few documents when using 80\% threshold.}
\begin{tabular}{|c||c|c|c|c|}
\hline
\bf Exercise & 1. & 2. & 3. \\ \hline
\bf JPlag - Documents retrieved & 2 & 2 & 0  \\ \hline
\bf Model A - Documents retrieved & 15 & 9 & 9 \\ \hline
\bf Common documents & 2  & 2 & 0\\ \hline
\bf Unique documents & 15 & 9 & 9\\ \hline
\bf Jaccard similarity    & 0.13  & 0.22  & 0.00  \\ \hline
\end{tabular}
\label{tbl-jacc-sd-ohja}
\end{table}

\noindent
The retrieval rate for OHJA's tasks for all our model candidates was very low, and this same result is reflected in Table \ref{tbl-jacc-sd-ohja} where JPlag retrieves only two documents or no documents at all. It seems that our model retrieves more documents than JPlag, but without a human interference it's impossible to say which one of the models is more correct. However, the retrieval from tasks 1. and 2. share the same two documents that JPlag detected, meaning that our model performs similar to JPlag but the scoring it produces is more consistent which can be seen when we inspect the third task where the level of agreement was the lowest. 

As we inspect every pair our model retrieved from OHJA's third task and compare the similarity scores to JPlag, we get five unique document pairs which are denoted here as $p_i, i \in [0, 5]$, formed by a total of nine documents. The results are visible in Figure \ref{fig-jplag-sd-ohja3}.

\begin{figure}[ht]
    \centering
    \setlength\figureheight{5cm}
    \setlength\figurewidth{0.8\textwidth}
    \input{plots/result/SD/model_a_vs_jplag_ohja3.tikz}
    \caption{The difference between JPlag's reported similarity value and our model for OHJA's final exam task.}
    \label{fig-jplag-sd-ohja3}
\end{figure}

\noindent
Figure \ref{fig-jplag-sd-ohja3} visualizes how our model keeps the similarity score near 80\% for every pair, whereas JPlag's score varies. The most similar scores are with pairs $p_1$ and $p_4$, where the difference is around 0.1 compared to our model. In other cases, it seems that JPlag can produce more specific results, because the comparing process differs from ours. We use the whole used vocabulary to produce the similarity score, whereas JPlag forms the score by string matching the token streams. 

We have now trained and evaluated our similarity detection model. The model we chose uses $n$-gram length of three, and retrieves any document where the calculated similarity value is above the 80\% threshold, which is reflected as $\varepsilon$-range of 0.2 in our clustering method. Our model was compared to JPlag and the retrieved documents were mostly the same, but there were some variance in number of documents retrieved. In following chapter, we train and evaluate the second model, the authorship identification.







\subsection{Authorship identification}
Our authorship identification model requires only one parameter to be tuned, the length
of character-level $n$-grams to be extracted. We tune this parameter based on the average $F_1$-score and accuracy
over seven split points for both OHPE and OHJA. For every split the final exercise is left out as an test data, 80\% of the remaining used for training and 20\% for validation. 

Course feedback is filtered out:

\begin{table}[ht]
\centering
\caption{(OHPE  1-7, OHJA 8-14) based on last exercise. Profile size is document count per student rounded to nearest integer.}
\label{lbl-result-ai-ohpe-ohja-stat}
\scalebox{0.65}{
    \begin{tabular}{|c||c|c|c|c|c|c|c||c|c|c|c|c|c|c|}
    \hline
    \bf Week         & \bf 1. & \bf 2. & \bf 3. & \bf 4. & \bf 5. & \bf 6. & \bf 7. & \bf 8. & \bf 9. & \bf 10. & \bf 11. & \bf 12. & \bf 13. & \bf 14. \\ \hline
    \bf Students     & 230  & 239  &  189 & 174  & 127  & 138  & 53  & 144  & 114  & 137   & 90   & 111   & 121   &  113  \\ \hline
    \bf AVG Profile size & 24  & 40  & 64  & 76  & 85  & 94  & 102  & 11  & 21  &  30  &  36  &  43  & 50   & 53    \\ \hline
    \end{tabular}
}
\end{table}


Results are shown below.


\begin{table}[ht]
\centering
\caption{Macro-averaged $F_1$-score (OHPE) for validation data}
\label{lbl-result-ai-f1-ohpe}
\begin{tabular}{|c|c|c|c|c|c|c|c|c|} \hline
\backslashbox{\bf $n$-gram}{\bf Week}  & 1 & 2 & 3 & 4 & 5 & 6 & 7 \\ \hline
4     &     0.01 & 0.03  & 0.03  & 0.04  & 0.04  & 0.04  & 0.04    \\ \hline
6     &  0.02    & 0.04  & 0.05  & 0.05  & 0.05  & 0.05  & 0.05    \\ \hline
8     & 0.02     & 0.04  & 0.05  & 0.06  & 0.06  & 0.06  & 0.06    \\ \hline
10    &  0.02    & 0.05  & 0.06  & 0.06  & 0.07  & 0.07  & 0.07     \\ \hline
12    & 0.02     & 0.05  & 0.06  & 0.06  & 0.07  & 0.07  & 0.07     \\ \hline
14    & 0.02     & 0.05  & 0.06  & 0.07  & 0.07  & 0.07  & 0.07    \\ \hline
\end{tabular}
\end{table}


Because the result were so poor, we limited the amount of students. Using the last week where students have the largest profile, we plot the accurace for various author sizes.

\newpage

\begin{figure}[ht]
\centering
\setlength\figureheight{7cm}
\setlength\figurewidth{\textwidth}
\input{plots/ohpeohja_ai_ng14.tikz}
\caption{Accuracy on validation set when different size of author pools are being used. Accuracy deteriorates when more than two authors are present. (ng=14, char, 7 weeks))} \label{fig-ohpeohja-ai-ng14}
\end{figure}

From Figure \ref{fig-ohpeohja-ai-ng14} we see that the model completely fails to predict the author when the size of authors is increased. 



\begin{table}[ht]
\centering
\caption{Test}
\label{asdasd}
\begin{tabular}{|c|c|c|c|c|c|c|c|}
          & NG & 4 & 6 & 8 & 10 & 12 & 14 \\
In set of &    &   &   &   &    &    &    \\
5         &    &   &   &   &    &    &    \\
10        &    &   &   &   &    &    &    \\
20        &    &   &   &   &    &    &    \\
25        &    &   &   &   &    &    &   
\end{tabular}
\end{table}

\begin{table}[ht]
\centering
\caption{Macro-averaged $F_1$-score (OHJA) for validation data}
\label{lbl-result-ai-f1-ohja}
\begin{tabular}{|c|c|c|c|c|c|c|c|c|} \hline
\backslashbox{\bf $n$-gram}{\bf Week}  & 1 & 2 & 3 & 4 & 5 & 6 & 7 \\ \hline
4     &      &   &   &   &   &   &     \\ \hline
6     &      &   &   &   &   &   &     \\ \hline
8     &      &   &   &   &   &   &     \\ \hline
10    &      &   &   &   &   &   &      \\ \hline
12    &      &   &   &   &   &   &      \\ \hline
14    &      &   &   &   &   &   &     \\ \hline
\end{tabular}
\end{table}




\newpage


\begin{table}[ht]
\centering
\caption{My caption}
\label{lbl-result-ai-best-model}
\begin{tabular}{|c|c|c|c|c|c|c|c|} \hline
Week & 1 & 2 & 3 & 4 & 5 & 6 & 7 \\
ACC  &   &   &   &   &   &   &  \\ \hline
\end{tabular}
\end{table}

\subsection{PLGDetect}
Because the Multinomial Naïve Bayes and the SCAP evaluated poorly with our data sets, we decide not to use authorship identification for the final results as even reducing the amount of authors would diminish the possibility of finding any plagiarists as random sampling would leave some students out of the detection. This is a drawback for our approach and we discuss the implications at the discussion. However, our similarity detection model evaluated well and can be still used for exploring and detecting the possible plagiarists. What we can't do is to restrict efficiently the amount of false detections by using the authorship identification model.

Before we can discuss the final results, we must consider an issue with the retrieval rate of our similarity detection. Looking from the Table \ref{tbl-jacc-sd-ohpe} and Table \ref{tbl-jacc-sd-ohja} there are around 500 total documents retrieved, which is too many documents for the human expert to go through in reasonable time. To overcome this issue the we select only a subset of the exam tasks reducing the amount of documents to 144. These are OHPE's third exam task (3.A) and all of the exam tasks of OHJA's (1.B, 2.B, 3.B). A brief description of each selected task is given below.

\paragraph{3.A (OHPE)} Students were required to fill a method to find the most common number from the Java's ArrayList structure. The methods name, return value and parameters were given as a template. 

\paragraph{1.B (OHJA)} Students were required to make a text interface for adding books with name and year information. The outline of the text interface was given for the students. After the initial adding phase, added books were printed in wanted order.

\paragraph{2.B (OHJA)} This task measured how well students are able to manipulate text data. The task required to have a small text interface to read a text file, censor every occurrence of a given word and write the results to a new text file. This exercise had a hint, which recommended to use a specific Java class to read and write text files. 

\paragraph{3.B (OHJA)} Task required to create a text interface to emulate a simple storage management software. The actions that had to be implemented were adding, listing, searching, removing items and exiting the interface. A small piece of code was given as a hint for this exercise.

\mbox{}\\
\noindent
In all exam tasks, also the scoring and example output was given for the students, so that they could mimic the wanted functionality of these programs. The reason behind this was to guide the student into right direction and also to be able to automatically score the submissions.

To see the difference between these tasks, descriptive statistics about them is given in Table \ref{tbl-plagdet-desc-stat}. It shows how OHPE differs from OHJA, as its task is quite constrained having only around 50 lines to get a correct answer. OHPE also creates a lot more clusters, as the similarities between OHJA's submissions are more varied.


\begin{table}[ht]
\centering
\caption{Results before the evaluation by the human expert. These results are produced by our similarity detection model which uses parameters $n=3$ for the $n$-gram length and $\varepsilon=0.2$ for the maximum allowed distance between the documents, which reflects that the documents have to score over 80\% similarity in order to cluster them together.}
\begin{tabular}{|c||c|c|c|c|}
\hline
\bf Task                & 3.A & 1.B & 2.B & 3.C  \\ \hline
\bf Number of submissions & 227 & 200 & 198 & 197 \\ \hline
\bf Average line count         & 47   & 160    & 85   & 150     \\ \hline
\bf Documents retrieved & 111 & 15 & 9 & 9 \\ \hline
\bf Clusters emerged & 15 & 5 & 3 & 4 \\ \hline
\end{tabular}
\label{tbl-plagdet-desc-stat}
\end{table}

\noindent
As the final result, we first show the pair-level detection results and then the more general result, which shows the precision with respect to documents considered containing plagiarism. For each of these tasks we inspect every cluster and the true and false positives in them, where the results are given by our human expert who has manually gone through detected documents. Results for each task is given in following figures, where we show the frequencies of retrieved pairs compared to true positives. Note that this format is more fine grained than what we have used before as earlier we have reported only the number of documents detected, and that we had to prune the first cluster of OHPE's third task, as it contained nearly 410 pairs. Pruning was done by keeping only the pairs where the cosine similarity was 1.0.

\newpage


\begin{figure}[ht] 
    \setlength\figureheight{7cm}
    \setlength\figurewidth{\textwidth}
    \input{plots/result/AI/plgdet_cluster_tpfp_ohpe_3.tikz}
    \caption{Detected and true pairs of 3.A OHPE. False positives in the first cluster were mostly correct submissions which were similar to model solution. Fourth cluster contained almost empty submissions and sixth cluster similarly wrong solutions with two highly suspicious authors.}
    \label{fig-plgdet-res3a}
\end{figure}

\begin{figure}[!h] 
    \setlength\figureheight{6cm}
    \setlength\figurewidth{\textwidth}
    \input{plots/result/plgdet/plgdet_cluster_tpfp_ohja_1.tikz}
    \caption{Detected and true pairs of 1.B OHJA. Most of the pairs were reported to be close to model solution without any signs of plagiarism. However, there were two pairs which were flagged for further attention.}
     \label{fig-plgdet-res1b}
\end{figure}

\newpage

\begin{figure}[ht] 
    \setlength\figureheight{6cm}
    \setlength\figurewidth{\textwidth}
    \input{plots/result/plgdet/plgdet_cluster_tpfp_ohja_2.tikz}
    \caption{Detected and true pairs of 2.B OHJA. All of the detected pairs in this task were false positives. However, two non-paired authors were flagged for further attention.}
    \label{fig-plgdet-res2b}
\end{figure}

\begin{figure}[!h] 
    \setlength\figureheight{6cm}
    \setlength\figurewidth{\textwidth}
    \input{plots/result/plgdet/plgdet_cluster_tpfp_ohja_3.tikz}
    \caption{Detected and true pairs of 3.B OHJA. Three pairs were flagged for further attention, but as difficult cases.}
    \label{fig-plgdet-res3b}
\end{figure}

\noindent
In Figures \ref{fig-plgdet-res3a}, \ref{fig-plgdet-res1b} and \ref{fig-plgdet-res3b}, we see that our approach is able to retrieve suspicious documents. As reported by the human expert, most of true positives contain direct copies and renaming of the variables. However, there exist false positives as seen in Figure \ref{fig-plgdet-res2b} where most of these false positives are caused by natural similarity between the submissions. The human expert reported also that in most of the cases one can't say for sure that the document pair is plagiarism. Therefore, the reported pairs are flagged if they are considered as suspicious and would require further information \eg other submissions done by the pair of authors. In the table below, one sees the document level results of false and true positives with the level of precision for each task.

\newpage


\begin{table}[ht]
\centering
\caption{Document-level results of our plagiarism detection. There are false positives introduced to our detection results.}
\begin{tabular}{|c|c|c|c|c|}
\hline
\bf Task      & 3.A   & 1.B   & 2.B & 3.B   \\ \hline
\bf True Positives        & 30   & 4    & 0  & 6    \\ \hline
\bf False Positives        & 26   & 11   & 9  & 3    \\ \hline
\bf Precision & 0.54 & 0.27 & 0.00  & 0.67 \\ \hline
\end{tabular}
\label{tbl-plgdet-final-res}
\end{table}

\noindent
The low precision in Table \ref{tbl-plgdet-final-res} shows how our model fails to limit the amount of false positives, which can be mostly due to the fact that we had to use only the similarity detection part of our approach. As seen before, all of the submissions for OHPE and OHJA contain a high level of natural similarity, which introduces many false positives even with as high threshold as 80\%. To help the work of our human expert, we had to prune the first cluster of OHPE's third task. In reality there would be near 400 detected document pairs, which are clearly all false positives due to the restricted solution space of the task.


After the human expert evaluated the detected documents, the five plagiarists caught in 2016 were revealed to us. Our model was able to retrieve documents belonging for all of these authors in OHPE's third task. 

