Our authorship identification model requires only one parameter to be tuned, the length
of character-level $n$-grams to be extracted. We tune this parameter based on the average $F_1$-score and accuracy
over seven split points for both OHPE and OHJA. For every split the final exercise is left out as an test data, 80\% of the remaining used for training and 20\% for validation. 

Course feedback is filtered out:

\begin{table}[ht]
\centering
\caption{(OHPE  1-7, OHJA 8-14) based on last exercise. Profile size is document count per student rounded to nearest integer.}
\label{lbl-result-ai-ohpe-ohja-stat}
\scalebox{0.65}{
    \begin{tabular}{|c||c|c|c|c|c|c|c||c|c|c|c|c|c|c|}
    \hline
    \bf Week         & \bf 1. & \bf 2. & \bf 3. & \bf 4. & \bf 5. & \bf 6. & \bf 7. & \bf 8. & \bf 9. & \bf 10. & \bf 11. & \bf 12. & \bf 13. & \bf 14. \\ \hline
    \bf Students     & 230  & 239  &  189 & 174  & 127  & 138  & 53  & 144  & 114  & 137   & 90   & 111   & 121   &  113  \\ \hline
    \bf AVG Profile size & 24  & 40  & 64  & 76  & 85  & 94  & 102  & 11  & 21  &  30  &  36  &  43  & 50   & 53    \\ \hline
    \end{tabular}
}
\end{table}


Results are shown below.


\begin{table}[ht]
\centering
\caption{Macro-averaged $F_1$-score (OHPE) for validation data}
\label{lbl-result-ai-f1-ohpe}
\begin{tabular}{|c|c|c|c|c|c|c|c|c|} \hline
\backslashbox{\bf $n$-gram}{\bf Week}  & 1 & 2 & 3 & 4 & 5 & 6 & 7 \\ \hline
4     &     0.01 & 0.03  & 0.03  & 0.04  & 0.04  & 0.04  & 0.04    \\ \hline
6     &  0.02    & 0.04  & 0.05  & 0.05  & 0.05  & 0.05  & 0.05    \\ \hline
8     & 0.02     & 0.04  & 0.05  & 0.06  & 0.06  & 0.06  & 0.06    \\ \hline
10    &  0.02    & 0.05  & 0.06  & 0.06  & 0.07  & 0.07  & 0.07     \\ \hline
12    & 0.02     & 0.05  & 0.06  & 0.06  & 0.07  & 0.07  & 0.07     \\ \hline
14    & 0.02     & 0.05  & 0.06  & 0.07  & 0.07  & 0.07  & 0.07    \\ \hline
\end{tabular}
\end{table}


Because the result were so poor, we limited the amount of students. Using the last week where students have the largest profile, we plot the accurace for various author sizes.

\newpage

\begin{figure}[ht]
\centering
\setlength\figureheight{7cm}
\setlength\figurewidth{\textwidth}
\input{plots/ohpeohja_ai_ng14.tikz}
\caption{Accuracy on validation set when different size of author pools are being used. Accuracy deteriorates when more than two authors are present. (ng=14, char, 7 weeks))} \label{fig-ohpeohja-ai-ng14}
\end{figure}

From Figure \ref{fig-ohpeohja-ai-ng14} we see that the model completely fails to predict the author when the size of authors is increased. 



\begin{table}[ht]
\centering
\caption{Test}
\label{asdasd}
\begin{tabular}{|c|c|c|c|c|c|c|c|}
          & NG & 4 & 6 & 8 & 10 & 12 & 14 \\
In set of &    &   &   &   &    &    &    \\
5         &    &   &   &   &    &    &    \\
10        &    &   &   &   &    &    &    \\
20        &    &   &   &   &    &    &    \\
25        &    &   &   &   &    &    &   
\end{tabular}
\end{table}

\begin{table}[ht]
\centering
\caption{Macro-averaged $F_1$-score (OHJA) for validation data}
\label{lbl-result-ai-f1-ohja}
\begin{tabular}{|c|c|c|c|c|c|c|c|c|} \hline
\backslashbox{\bf $n$-gram}{\bf Week}  & 1 & 2 & 3 & 4 & 5 & 6 & 7 \\ \hline
4     &      &   &   &   &   &   &     \\ \hline
6     &      &   &   &   &   &   &     \\ \hline
8     &      &   &   &   &   &   &     \\ \hline
10    &      &   &   &   &   &   &      \\ \hline
12    &      &   &   &   &   &   &      \\ \hline
14    &      &   &   &   &   &   &     \\ \hline
\end{tabular}
\end{table}




\newpage


\begin{table}[ht]
\centering
\caption{My caption}
\label{lbl-result-ai-best-model}
\begin{tabular}{|c|c|c|c|c|c|c|c|} \hline
Week & 1 & 2 & 3 & 4 & 5 & 6 & 7 \\
ACC  &   &   &   &   &   &   &  \\ \hline
\end{tabular}
\end{table}