\section{Sample programs} \label{appendix:programs}

These three functionally similar source codes belong to three imaginary authors A, B and C. They are used throughout the study as examples. The task for all is to create a program that calculates mean between three numbers: 5, 10, 2.

\begin{lstlisting}[language=Java, caption=Java example belonging to author A]
public class A{

     public static void main(String[] args){
        int a = 5;
        int b = 10;
        int c = 2;
        double d = (a + b + c)/(double)3;
        System.out.println(d);
     }
}
\end{lstlisting}

\begin{lstlisting}[language=Java, caption=Java example belonging to author B]
public class B{

     public static void main(String[] b){
        int sum = 5 + 10 + 2;
        double res = sum / 3.0;
        System.out.println(res);
     }
}
\end{lstlisting}

\begin{lstlisting}[language=Java, caption=Java example belonging to author C]
public class C{
     public static void main(String[] b){
        System.out.println((5 + 10 + 2)/3.0);
     }
}
\end{lstlisting}

\newpage

\section{Token list} \label{appendix:token-list}

\begin{table}[ht]
\centering
\caption{Token list for Java.}
\label{tbl-token-list}
\def\arraystretch{1.5}
\scalebox{0.75}{
    \begin{tabular}{c l p{7cm} l} 
     &  \bf Token  & \bf Equivalency          & \bf Example    \\ \hline
    1 & IMPORT & Import declaration & \texttt{import java.awt.*;} \\ \hline
    2 & PACKAGE & Package declaration & \texttt{package foo;} \\ \hline
    3 & VARDEF & Variable declaration & \texttt{int a;} \\ \hline
    4 & CLASS\{      &  Enter class declaration   &  \texttt{public class A\{} \\ \hline
    5 & CATCH\{ & Enter catch clause & \texttt{try \{\underline{catch (...)\{}\} \}} \\ \hline
    6 & INCLASS\{ & Statement inside a class & - \\ \hline
    7 & ENUM\{ & Enter enum declaration & \texttt{public enum Day \{} \\ \hline
    8 & APPLY & Method call, Explicit constructor invocation, Generic invocation  & \texttt{System.out.print(...);} \\ \hline
    9 & NEWCLASS & Create object & \texttt{new A(...);} \\ \hline
    10 & NEWARRAY & Create array object & \texttt{new int[5];} \\ \hline
    11 & TRY\{ & Enter try declaration & \texttt{try \{ } \\ \hline
    12 & INTERF\{ & Enter interface declaration & \texttt{interface Foo \{} \\ \hline
    13 & METHOD\{ & Enter method declaration & \texttt{void foo(int a) \{} \\ \hline
    14 & VOID & Void method & \texttt{void main(String[] args)} \\ \hline
    15 & CASE & Case in switch statement & \texttt{case MONDAY:} \\ \hline
    16 & CONSTR\{ & Enter constructor declaration & \texttt{public A(int a, int b) \{} \\ \hline
    17 & ARRINIT\{ & Enter array initialization & \texttt{new int[] \{1, 2\};} \\ \hline
    18 & ASSIGN & Variable assignment & \texttt{a += 5;} \\ \hline
    19 & COND & Conditional expression & \texttt{(a > b) ? a : b;} \\ \hline
    20 & LOOP\{ & Enter for, while, do statement & \texttt{for(...) \{}\\ \hline
    21 & IF\{ & Enter if clause & \texttt{if(...) \{} \\ \hline
    22 & THROW & Throw statement & \texttt{throw new Exception();} \\ \hline
    23 & BREAK & Break statement in loop & \texttt{break;} \\ \hline
    24 & CONTINUE & Continue statement in loop & \texttt{continue;} \\ \hline
    25 & RETURN & Return statement & \texttt{return a + b;} \\ \hline
    26 & SWITCH\{ & Enter switch statement & \texttt{switch(...) \{} \\ \hline
    \end{tabular}
}
\end{table}

\noindent
Table \ref{tbl-token-list} shows the token list used to transform a parse tree into a continuous string of tokens. Every token with ending bracket also has a reserved token when exiting the statement.