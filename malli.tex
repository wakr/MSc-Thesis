% --- Template for thesis / report with tktltiki2 class ---
% 
% last updated 2013/02/15 for tkltiki2 v1.02

\documentclass[english, grading]{tktltiki2}

% tktltiki2 automatically loads babel, so you can simply
% give the language parameter (e.g. finnish, swedish, english, british) as
% a parameter for the class: \documentclass[finnish]{tktltiki2}.
% The information on title and abstract is generated automatically depending on
% the language, see below if you need to change any of these manually.
% 
% Class options:
% - grading                 -- Print labels for grading information on the front page.
% - disablelastpagecounter  -- Disables the automatic generation of page number information
%                              in the abstract. See also \numberofpagesinformation{} command below.
%
% The class also respects the following options of article class:
%   10pt, 11pt, 12pt, final, draft, oneside, twoside,
%   openright, openany, onecolumn, twocolumn, leqno, fleqn
%
% The default font size is 11pt. The paper size used is A4, other sizes are not supported.
%
% rubber: module pdftex

% --- General packages ---

\PassOptionsToPackage{hyphens}{url}
\usepackage[hyphens]{url}
\usepackage[utf8]{inputenc}
\usepackage[T1]{fontenc}
\usepackage{lmodern}
\usepackage{microtype}
\usepackage{amsfonts,amsmath,amssymb,amsthm,booktabs,color,enumitem,graphicx}
\usepackage[pdftex,hidelinks]{hyperref}
\usepackage{longtable}
\usepackage{listings}
\usepackage{diagbox}

\usepackage{algorithm}
\usepackage{algorithmic}
%\usepackage[bottom]{footmisc}

% Automatically set the PDF metadata fields
\makeatletter
\AtBeginDocument{\hypersetup{pdftitle = {\@title}, pdfauthor = {\@author}}}
\makeatother

% --- Language-related settings ---
%
% these should be modified according to your language

% babelbib for non-english bibliography using bibtex
\usepackage[fixlanguage]{babelbib}
\selectbiblanguage{english}

% add bibliography to the table of contents
\usepackage[nottoc]{tocbibind}
% tocbibind renames the bibliography, use the following to change it back
\settocbibname{References}

% -- Handy shortcuts
\newcommand{\etal}{\textit{et al}. }
\newcommand{\ie}{\textit{i}.\textit{e}., }
\newcommand{\eg}{\textit{e}.\textit{g}. }
\newcommand{\cpp}{C\texttt{++}}
\newcommand{\bolditt}[1]{\mathbf{#1}}
\newcommand{\norm}[1]{\left\lVert#1\right\rVert}

% --- Theorem environment definitions ---
\newtheorem{lau}{Lause}
\newtheorem{lem}[lau]{Lemma}
\newtheorem{kor}[lau]{Korollaari}

\theoremstyle{definition}
\newtheorem{maar}[lau]{Definition}
\newtheorem{ong}{Ongelma}
\newtheorem{alg}[lau]{Algoritmi}
\newtheorem{esim}[lau]{Esimerkki}

\theoremstyle{remark}
\newtheorem*{huom}{Huomautus}


% --- tktltiki2 options ---
%
% The following commands define the information used to generate title and
% abstract pages. The following entries should be always specified:

\title{Automatic Software Plagiarism Detection}
\author{Kristian Wahlroos}
\date{\today}
\level{M.Sc. Thesis}
\abstract{
Plagiarism is an act of copying where one doesn't rightfully credit the original source. The motivations behind plagiarism can vary from gaining economical advantage to even completing academic courses in a desperate way. Plagiarism itself exists in various domains where people want to take credit from something they have worked on. These areas can include e.g. literature, art or software.   

In this study, automatic authorship identification and plagiarism detection from software is analyzed and a model is built on top of findings. The term \textit{automatic} here refers to a system which requires as little as possible human intervention as possible. The goal for the model is to point out possible plagiarism from a collection of documents, which here is specified as a collection of source code files written by various authors. This situation closely related to authorship identification, and thus as a statistical tool, supervised machine learning model is utilized. The latter problem can be stated as \emph{given a document, which author has written this} and \emph{does this document follow the previous style of the author}. 
}

% The following can be used to specify keywords and classification of the paper:

\keywords{plagiarism; authorship identification; }

% classification according to ACM Computing Classification System (http://www.acm.org/about/class/)
% This is probably mostly relevant for computer scientists
% uncomment the following; contents of \classification will be printed under the abstract with a title
% "ACM Computing Classification System (CCS):"
% \classification{}

% If the automatic page number counting is not working as desired in your case,
% uncomment the following to manually set the number of pages displayed in the abstract page:
%
% \numberofpagesinformation{16 sivua + 10 sivua liitteissä}
%
% If you are not a computer scientist, you will want to uncomment the following by hand and specify
% your department, faculty and subject by hand:
%
% \faculty{Matemaattis-luonnontieteellinen}
% \department{Tietojenkäsittelytieteen laitos}
% \subject{Tietojenkäsittelytiede}
%
% If you are not from the University of Helsinki, then you will most likely want to set these also:
%
% \university{Helsingin Yliopisto}
% \universitylong{HELSINGIN YLIOPISTO --- HELSINGFORS UNIVERSITET --- UNIVERSITY OF HELSINKI} % displayed on the top of the abstract page
% \city{Helsinki}
%


% 10-15 pages abstract

\begin{document}

% --- Front matter ---

\frontmatter      % roman page numbering for front matter

\maketitle        % title page
\makeabstract     % abstract page

\tableofcontents  % table of contents

% --- Main matter ---

\mainmatter       % clear page, start arabic page numbering

\section{Introduction}

What means term plagiarism?
What is plagiarism?
Why studied here --> 

\section{Background}


\subsection{Source code plagiarism}
\subsubsection{Code structure}
\subsubsection{Plagiarism types}
\subsubsection{Plagiarism motives}
\subsubsection{In-class plagiarism}
\subsubsection{Plagiarism in industry}
\subsubsection{Similarity measure}
\subsection{Authorship identification}
\subsection{Machine learning}
\subsubsection{Clustering}
\subsubsection{Classification}

\newpage

\section{Related Work}
Systematic literature review was conducted to gain information about the current state of source code plagiarism study in academia. The database that was utilized to query research papers is called \emph{Scopus\footnote{\url{https://www.scopus.com/}}}, which is a web service containing peer-reviewed scientific literature. The service itself holds links to papers which are published under for example \emph{ACM (Association for Computing Machinery)} and \emph{IEEE (Institute of Electrical and Electronics Engineers)}, both of these being major computer science institutions. 

Following subchapters describe how the review was conducted and what kind of results were found. 

\subsection{Review process}

Querying Scopus can be done in a similar way as querying databases in SQL-like languages. The used query inside Scopus was following
\begin{verbatim}
TITLE-ABS-KEY (("plagiarism" OR "authorship identification")  
                AND "source code") 
AND  (LIMIT-TO (SUBJAREA,"COMP"))
\end{verbatim}

\noindent
This query translates to finding papers which title, abstract or keywords contains the word \emph{plagiarism} or \emph{authorship identification} and at least the term \emph{source code}. The reason to choose these keywords was to find papers which study the problem of plagiarism finding from source code either in general terms, or by utilizing authorship identification techniques. Next, the query limits the area of study to computer science to focus on plagiarism studies which utilize techniques found from computer science.

After the initial search, the goal of the second step was to limit the amount of papers. This was done by excluding all papers that were any of the following types: a review of certain aspect of source code plagiarism e.g. student motives behind plagiarism, an improvement to some pre-existing algorithm\footnote{In this context meaning algorithmic speedup}, plugin to online learning management systems, application to competition where the used method wasn't explained, study that used either byte-level information or information gathered during running the program, hashing techniques\footnote{Using the size of compression as a metric}, system review which didn't address the method and theses. 

Beside these attributes, included papers needed to also test their proposed method in some way and the amount of documents in experiment phase needed to be larger than two. The reason for including this as a limiting factor, was to gather studies that used test sets to evaluate the performance of their model in terms of accuracy, as this allows to compare used techniques more critically. 


\subsection{Review results}

The total number of papers gathered by querying Scopus in the first part of the literature review was 187, and the date when the query was done was 7th of February 2018. The distribution of paper per year can be seen in the following plot.

\begin{figure}[h]
\centering
\includegraphics[width=\textwidth]{plots/Rplot.png}
\caption{Results of the first query to Scopus}
\end{figure}

\noindent
This set was filtered by terms described in the previous chapter, and the total number of papers inspected more carefully in this systematic literature review is 32. 

From the set of 32 studies, we look answers for following questions: \emph{how plagiarism can be detected from source code}, \emph{what are possible features that can be derived from source code} and \emph{how can one identify the author of a given source code}. We start first by grouping the papers by their themes to see what kind of different approaches there are to deal with the problem of plagiarism detection. After the initial classification, following aspects are identified from the studies: data, methods and test results. 


\subsubsection{Approaches}

The most high level division between papers could be done in a similar fashion as was used by the
query; dividing papers either to be about the detection of plagiarism or identifying the author of a given source code. However, during the literature review it was found that it's more clearer to make a division between similarity detection and identifying the authorship. This high-level division can be seen from the following table.

\begin{table}[ht]
    \caption{Papers divided into two high-level categories}
    \label{table-highcateq}
    \centering
    \begin{tabular}{ | c | c | }
        
        \hline
        {\bf Similarity detection} & {\bf Authorship identification} \\ \hline
    
        \cite{AFAPLI2015, LICD2010, AASCPD2012} & \cite{SCAANN2017, ABEC2014, CAPSCAP2014}   \\
        \cite{Heblikar2015NormalizationBS, USCR2014, AIR2015} &  \cite{SCANG2007, EJPFSAI2004, ACSBPD2012}\\
        \cite{OTIOLSS2015, BUAA2009, ramirez2015high} &  \cite{APASCAI2007, UCMHGAAI2007, ESHPFSCAC2008}\\
        \cite{Ohmann2015, TBCFPD2012, Fu2017WASTKAW} &  \cite{AIRTSCAA2009, TSUDIJSCAI2011, DNNSCAI2013} \\
        \cite{ASTMLPD2013, AAPSCDPTK2013, CPDPPD2013}    & \cite{SCAIUFL2013, SDNAIJSP2015, AISC2017} \\
        \cite{PACASCD2005, RCISCP2017} &  \\ \hline
        {\bf Number of papers} & {\bf Number of papers} \\ \hline
        17 & 15 \\ \hline
    \end{tabular}
\end{table}

\noindent
Even though papers divide quite evenly in table \ref{table-highcateq}, these high-level groups are still too large, and thus for the sake of clarity, we divide both into subgroups.

Similarity detection in itself can be further divided into at least two general categories based on the current tools \cite{RSCAD2016}: attribute and structure. Then naturally, as authorship identification uses features derived directly from the source code, we can use the same classification to authorship identification studies. However, based on the literature review, there are more finer categorizations that define the studies better based on the features they use, and thus we propose the following categories and their abbreviations: \emph{attribute counting (AC)}, \emph{segment matching (SM)}, \emph{n-gram (NG-STR)}, \emph{tree-based (AST-STR)} and lastly \emph{hybrid approaches (HYB-STR)}. If category has no studies under it, we leave the category out from the upcoming tables. These categories can be summarized briefly as following and are similar to categories identified from other similarity detection studies by Ali \etal \cite{OCPOCP2011}. 

\paragraph{Attribute counting}
Studies utilizing countable statistics, often referred as \emph{metrics}, that are gathered from source codes. This includes features like amount of words per line, number of lines per source code and number of keywords.

\paragraph{Segment matching}
Considers two source codes as two strings and finding maximum match between them i.e. longest common subsequence. These problems are also known as string matching problems, where one of the most famous algorithms is \emph{Greedy String Tiling} introduced early in \cite{SSGST1993}. We also categorize string similarity measures to this category like string edit distances.

\paragraph{$N$-gram}
Treating the source code as a string and splitting it via sliding window where the window size is the value of $n$ and the window traverses on either word or character level. This forms the vocabulary of the source code which is then transformed into occurrences of particular terms that are present, thus ultimately creating a vector representation of the source code. For example the statement \texttt{int a = 2} could be transformed into following word level 2-tuples using two as the value of $n$ (bigram). The first value of the following tuples is the $n$-gram extracted and the second value is the frequency: (\texttt{int a}, 1), (\texttt{a =}, 1), (\texttt{= 2}, 1). 

\paragraph{Tree-based methods}
Constructing a tree presentation from the source code, that captures the structure. The generation of a tree presentation usually requires some kind of parser because it's language specific feature. The inspection of a generated tree can be done via tree traversal methods for example using recursive functions. 

\paragraph{Hybrid methods}
Combine the usage of AST-structure with $n$-gram representation. For example it can be a method which traverses abstract syntax tree, prints it and generates $n$-gram representation from the output.
\\\\
The grouping of similarity detection papers can be seen from following table, where it's clear that most of the papers deal with similarity detection by utilizing structural features, indicated by the STR-ending, and many studies prefers to use $n$-gram representation of the source code.

\begin{table}[ht]
    \caption{Subgroups and sizes of similarity detection studies}
    \label{table-sdstudies}
    \centering
    \begin{tabular}{ | c | c | c | c | c |}
        
        \hline
        {\bf AC} & {\bf SM} & {\bf NG-STR} & {\bf AST-STR} & {\bf HYB-STR} \\ \hline
        \cite{PACASCD2005} & 
        \cite{LICD2010, ASTMLPD2013} & 
        \cite{AASCPD2012, USCR2014, AFAPLI2015} & 
        \cite{TBCFPD2012, AAPSCDPTK2013, AIR2015} & 
        \cite{BUAA2009, CPDPPD2013, RCISCP2017} \\
        & 
        & 
        \cite{Heblikar2015NormalizationBS, Ohmann2015, OTIOLSS2015} & 
        \cite{Fu2017WASTKAW} &
        \\
        & & \cite{ramirez2015high} &  & \\ \hline
        {\bf \#AC} & {\bf \#SM} & \multicolumn{3}{c |}{\bf \#STR} \\ \hline
        1 & 2 & \multicolumn{3}{c |}{14}
        \\ \hline
    \end{tabular}
\end{table}

When inspecting the division of authorship studies, we can see the division in table \ref{table-aistudies} is more evenly distributed contrast to similarity detection studies. More studies seems to utilize countable attributes from source codes and many also prefers to utilize $n$-grams, which is quite obvious when one considers that these methods are able to capture the writing style of an author from high-level features. For example authors can name the identifiers how they like, introduce comments and use various stylistic techniques when they write source code. 

\begin{table}[ht]
    \caption{Subgroups and sizes of authorship identification studies}
    \label{table-aistudies}
    \centering
    \begin{tabular}{ | c | c | c | c |}
        
        \hline
        {\bf AC} & {\bf NG-STR} & {\bf AST-STR} & {\bf HYB-STR} \\ \hline
        \cite{EJPFSAI2004, UCMHGAAI2007, APASCAI2007} & \cite{SCANG2007, ESHPFSCAC2008, AIRTSCAA2009} & \cite{SCAANN2017} & \cite{SDNAIJSP2015, AISC2017}\\ 
        \cite{ACSBPD2012, SCAIUFL2013, DNNSCAI2013} & \cite{TSUDIJSCAI2011, CAPSCAP2014, ABEC2014} & &\\ \hline
        {\bf \#AC} & \multicolumn{3}{c |}{\bf \#STR} \\ \hline
        6 & \multicolumn{3}{c |}{9}
        \\ \hline
    \end{tabular}
\end{table}

\newpage

Both of these results seems to show that utilizing structure is popular in both high-level classes, but quite dominant in similarity detection. However both groups of studies seems to show high popularity on $n$-gram methods, which is able to capture both individual style of the author and structural preferences. % source?


\subsubsection{Data}
The data used in testing phase of 32 gathered articles is presented next, where the focus is for the similarity detection on following attributes: number of total documents, is there any synthetic data used and the average number of lines of code (Avg. LOC). For the authorship identification we focus on features like documents per author and number of possible authors. The term \emph{document} here refers to the number of source code file samples per author. We summarize the findings from data utilizing the categorization that was made earlier.

\paragraph{Similarity Detection}\mbox{}\\
Attribute counting study by Moussiades and Vakali in \cite{PACASCD2005} uses two real data sets written in C\texttt{++}. They contain programming assignments and a forged set of programs. The first data set contains 24 programs having an average of 247 lines of code per submission, the second set is 51 programs having an average of 178 lines per source code. The forged data set is two modified versions from one program, trying to deliberately confuse state-of-the-art detectors.

Segment matching study by Brixtel et al. used three corpora on their evaluation and are written in Haskell, Python and C \cite{LICD2010}. Haskell corpus had 13 documents averaging 400 lines per each, Python 15 documents averaging 150 lines per each and C 19 documents averaging 250 lines per source code. Study by Zhang and Liu used 12 programs written in C that all reflected different plagiarism strategies \cite{ASTMLPD2013}. 

Studies utilizing $n$-grams are summarized into following table.

\begin{table}[ht]
\centering
\caption{Data used in similarity detection studies utilizing $n$-grams}
\label{table-ng-str-data}
\begin{tabular}{|c|c|c|c|c|c|c|c|}
          \hline
          \backslashbox{\bf Feature}{\bf Paper} & \cite{AASCPD2012} & \cite{USCR2014} & \cite{AFAPLI2015} & \cite{Heblikar2015NormalizationBS} & \cite{Ohmann2015} & \cite{OTIOLSS2015} & \cite{ramirez2015high} \\ \hline
\bf Documents  &  179  & 5302   & 191  & 1356  & 2935  & 5408  & 1277   \\ \hline
\bf Synthetic &  No  & No  &  No  & No  & No  &  No & No  \\ \hline
\bf Avg. LOC & NA  & NA  & NA  & NA & NA  & 63.7  & NA  \\ \hline
\end{tabular}
\end{table}

\noindent
It's visible from the table \ref{table-ng-str-data} that there are now a lot more documents used in experimentation and surprisingly synthetic data is not used at all. This is due to the usage of student submissions and competition data sets like \emph{Google Code Jam} submissions, which was for example utilized by Flores \etal in \cite{USCR2014}. 


\begin{table}[ht]
\centering
\caption{Data used in similarity detection studies utilizing abstract syntax tree}
\label{table-ast-str-data}
\begin{tabular}{|c|c|c|c|c|}
          \hline
          \backslashbox{\bf Feature}{\bf Paper} & \cite{TBCFPD2012} & \cite{AAPSCDPTK2013} & \cite{AIR2015} & \cite{Fu2017WASTKAW}\\ \hline
\bf Documents & 121 & 555 & NA & 22\,214  \\ \hline
\bf Synthetic & NA & No  & NA & Yes\\ \hline
\bf Avg. LOC & NA & 305.7 & NA & 20\\ \hline
\end{tabular}
\end{table}

\noindent
One can see from the table \ref{table-ast-str-data} that a study done by Fu \etal in \cite{Fu2017WASTKAW} has a large number of documents, and this due to two facts: they reported the size as pairs of documents and they used a generator to form a lot of forged documents from a small set of 10 original submissions. Ganguly and Jones in \cite{AIR2015} don't explicitly report the statistics of their data set, but refers to a competition test set called \emph{SOurce COde Re-use} (SOCO). This competition offers a set of C and Java files which contains known cases of cross-lingual plagiarism \cite{saez2014pan}. The train set size of SOCO is 338 files. 

Finally, hybrid study by Xiong \etal utilizes 40 assignments gathered from students \cite{BUAA2009}, Muddu \etal uses 5054 original files that they mutate to introduce copied code \cite{CPDPPD2013} and Ganguly \etal uses both train and test set of the SOCO competition, totaling around 12\,000 files \cite{RCISCP2017}. 


\paragraph{Authorship identification}\mbox{}\\
Usage of data in studies dealing with the problem of identifying the author and utilizing attribute counting are summarized to the following table, where we now turn the focus on the amount of candidate authors and documents per author reported in studies.

In table \ref{table-ai-ac-str-data}, one can see that there are two same data sets used in \cite{SCAIUFL2013, DNNSCAI2013}. This set was collected from \emph{SourceForge}\footnote{\url{https://sourceforge.net/}} projects and there are around 61 to 377 files per author. Rest of the attribute counting studies prefers to use \eg submissions gathered from students, as it's an easy way to gather tagged source code files.  

\begin{table}[ht]
\centering
\caption{Data used in authorship studies utilizing attribute counting}
\label{table-ai-ac-str-data}
\scalebox{0.9}{
    \begin{tabular}{|c|c|c|c|c|c|c|c|}
              \hline
              \backslashbox{\bf Feature}{\bf Paper} & \cite{EJPFSAI2004} & \cite{UCMHGAAI2007} & \cite{APASCAI2007}  & \cite{ACSBPD2012} & \cite{SCAIUFL2013} & \cite{DNNSCAI2013}\\ \hline
    \bf Authors  & 46 & 20 & 8  & 120 & 10 & 10 \\ \hline
    \bf Documents per author  & NA & 3 & 3  & NA & 61-377 & 61-377\\ \hline
    \bf Synthetic  & No & No & No & No & No & No\\ \hline
    \end{tabular}
}
\end{table}


Next, data sets from the second popular method $n$-grams used in authorship identification, are summarized into following table.

\begin{table}[ht]
\centering
\caption{Data used in authorship studies utilizing $n$-grams}
\label{table-ai-ng-str-data}
    \begin{tabular}{|c|c|c|c|c|c|c|}
              \hline
              \backslashbox{\bf Feature}{\bf Paper} & \cite{SCANG2007} & \cite{ESHPFSCAC2008} & \cite{AIRTSCAA2009} & \cite{TSUDIJSCAI2011} & \cite{CAPSCAP2014} & \cite{ABEC2014}\\ \hline
    \bf Authors  & 100 & 8 & 100 & 8 & 30 & 30\\ \hline
    \bf Documents per author  & 14 & 2 & 14-26 & 2 & NA & NA\\ \hline
    \bf Synthetic  & No & No & No & No & No & No\\ \hline
    \end{tabular}
\end{table}

\noindent
There exists three different data sets used by three different authors in table \ref{table-ai-ng-str-data}: Burrows \etal in \cite{SCANG2007, AIRTSCAA2009} used data set gathered from students C programming assignments, Frantzeskou \etal in \cite{ESHPFSCAC2008, TSUDIJSCAI2011} used open-source programs written in Java and Tennyson \etal in \cite{CAPSCAP2014, ABEC2014} used programs written in \cpp and Java which mixture of were open-source, sample and textbook programs.

The only study that mainly used abstract syntax tree in their authorship study is by Alsulami \etal in \cite{SCAANN2017}. They used \emph{Google Code Jam} to gather 700 Python source code files belonging to 70 programmers averaging around 10 submissions per author. 

Finally, the data used in two hybrid studies are summarized. Wisse and Veenman used repositories from version control website called \emph{GitHub} \cite{SDNAIJSP2015}. The largest author pool they had while testing was 30. Zhang \etal had the data set also gathered from websites like \emph{GitHub} in their study \cite{AISC2017}. Their largest data set with respect to the author size, was imbalanced set of 503 programs belonging to 53 authors. 

\paragraph{Summary}\mbox{}\\
When looking the data usage of plagiarism study as a whole, one can see that almost all studies use data that is non-synthetic \ie use real-life data, that can be gathered for example from students course submissions or from competitions like SOCO. In similarity detection studies the median of the amount of source codes used is 447 and very few studies reported the average lines of code, which is a bit problematic as it can be easier to find plagiarism from a small set of program lines than from larger programs. In authorship attribution the median of possible authors in studies is 30 and the documents per author ranges from two to as high as 377.

% tarkista mediaani

\subsubsection{Methods}
In this chapter we turn the focus to the actual methods used in various studies. We use the same 
classification as a baseline for studies that was made earlier. The math used in studies is generalized to match the style of this paper, which means that a document is represented as $d$, matrices are bold and upper-cased $\bolditt{A}$ and vectors are bold but lower-cased $\bolditt{a}$. 

\paragraph{Similarity detection}\mbox{}\\
As a recap, the problem of similarity detection can be described formally as following.

\newtheorem*{smd}{Similarity detection}

\begin{smd}
Given a set of source code documents $D = \{d_1,...,d_n\}$, define similarity function $sim: d_i, d_j \rightarrow [0, 1]$ such that $sim(d_i, d_j) = sim(d_j, d_i)$ and $sim(d_i, d_i) = 1$, with a optional threshold $\theta \in [0, 1]$ that defines the limit where two source codes are considered as too similar. With this definition, any pair of source code file $(d_i, d_j) \in D \times D$ can also be presented as a triplet $(d_i, d_j, s)$, where $i \neq j$ and $s$ is the similarity value between documents. 
\end{smd}

The attribute counting study by Moussiades and Vakali uses a graph clustering on top of pair-wise similarities calculated using the Jaccard coefficient \cite{PACASCD2005}. Authors use following form of Jaccard coeficcient in their study where $T$ is the indexed set of substitute keywords per source code 

\begin{equation}\label{jacc_eqn}
    sim(d_1, d_2) = \dfrac{|T(d_1) \cap T(d_2)|}{|T(d_1) \cup T(d_2)|}
\end{equation}
\noindent
% refer to plag. attack
The indexed set can be built considering language dependent keywords \eg \texttt{while, for, false and true} in \cpp, and marking their position with respect to the occurrences of same keywords previously. However, authors claim that to generalize the set more, substitution keywords should be used. This means that for example all occurrences of \texttt{for}- and \texttt{while} -loops should be counted together, which helps to protect against plagiarism attack. The graph clustering algorithm Moussiades and Vakali uses is called \emph{WMajorClust} which works by presenting all pairs of source codes as non-directed graph $G = (V, E)$ where the set of vertices $V$ represents the source codes while the set of edges $E$ are weighted by equation \ref{jacc_eqn}. We can also express the definition of $E$ by Moussiades and Vakali with following constraints

\begin{equation}\label{jacc_edges_eqn}
         E = \Big\{ \{ d_i, d_j, sim(d_i, d_j)\} \, | \, (d_i, d_j) \in D \times D \land sim(d_i, d_j) \geq \theta \Big\}
\end{equation}

\noindent
%chapter ref
In equation \ref{jacc_edges_eqn}, $\theta$ is a user-defined parameter and works as a minimum threshold value that separates non-plagiarized source codes from plagiarized ones \ie two source codes will not share an edge if their similarity is below $\theta$.

Segment matching study by Brixtel \etal presents their algorithm, which builds from three major steps \cite{LICD2010}: pre-filtering, segmentation and document distance calculation. Their pre-filtering is to normalize the source code in a way, that every keyword and parameter definitions is transformed into a single symbol. As a segmentation, authors split the source code by lines forming set of segments $S_k$ presenting the partitioned set of a single source code. Similarity calculation happens by first forming distance matrix $\bolditt{M}$ between two source codes $d_1, d_2$ and then comparing all pairs of segments $(s_i^1, s_j^2) \in S_1 \times S_2$ where $S_k = (s_1^k, ..., s_n^k)$, with \emph{Levenshtein edit distance}. Distance matrix $\bolditt{M}$ is then transformed into noise reduction matrix $\bolditt{H}$ by finding the maximal matching between segmentations. Finally, $\bolditt{H}$ is filtered into a matrix $\bolditt{P}$ by convolution and utilizing a threshold\footnote{Authors used $\theta = 0.7$}. With the matrix $\bolditt{P}$, distance between two pairs of documents can be calculated by Brixtel \etal as 

\begin{equation}
    sim(d_1, d_2) = 1 - \dfrac{1}{\min(|S_1|, |S_2|)}\sum_{i, j} 1 - \bolditt{P}_{(i, j)}
\end{equation}

\noindent
Zhang and Liu utilize AST-tree and their core method is mainly constructed from two methods \cite{ASTMLPD2013}: forming the AST-representation and similarity calculation. Their AST-representation is done by traversing the parsed AST-tree and turning it into textual format by printing the nodes, and similarity calculation is computed using \emph{Smith Waterman algorithm} that finds the optimal matching between two strings $S_1, S_2$. Zhang et Liu gives the formula for similarity calculation between two source codes as

\begin{equation}
    sim(d_1, d_2) = \dfrac{2 \cdot \text{ SLength}(d_1, d_2)}{|S_1| + |S_2|}
\end{equation}
\noindent
Where SLength is the length of maximal matching string obtained via  \emph{Smith Waterman algorithm}, and $|S_k|$ represents the character length of one segment. 


$N$-gram studies take a different approach. Cosma and Joy uses \emph{Latent Semantic Analysis} to find suspicious documents \cite{AASCPD2012}. They first preprocess the documents by removing \eg short terms and comments. Then all documents are first transformed into a term-by-file matrix $\bolditt{A}$, where each document is represented as a occurrences of each unique term, which is same as forming the unigrams of a document. Values of $\bolditt{A}$ are weighted, and then $\bolditt{A}$ is decomposed via \emph{singular value decomposition} into $\bolditt{A} = \bolditt{U}\mathbf{\Sigma}\bolditt{V}^\intercal$ where $\bolditt{U}$ represents terms by dimension, $\mathbf{\Sigma}$ singular values and $\bolditt{V}$ files by dimensions. The dimensionality reduction is performed for all these matrices by considering only the first 30 columns. Finally, the similarity between a query vector $\bolditt{q}$ representing term frequency of document $d_i$, and document $d_j$ represented as a column $\bolditt{a}_j$ of matrix $\bolditt{A}$ is calculated by \emph{cosine similarity} \cite{AASCPD2012}

\begin{equation}\label{cosine_sim_eqn}
    sim(\bolditt{q}, d_j) = \cos \Theta_j = \dfrac{\bolditt{a}_j^\intercal \bolditt{q}}{\norm{\bolditt{a}_j}_2 \norm{\bolditt{q}}_2} = \dfrac{\bolditt{a}_j \boldsymbol{\cdot} \bolditt{q}}{\sqrt{\sum \limits_{i} \bolditt{a}_{(j, i)}^2} \sqrt{\sum \limits_{i} \bolditt{q}_i^2}}
\end{equation}

\noindent
Acampora and Cosma \cite{AFAPLI2015} continues on same style as Cosma and Joy \cite{AASCPD2012}, first preprocessing the documents by lowercasing and removing comments, syntactical tokens and short terms. Then using singular value decomposition with weighting to form three matrices from the corpus of source codes. For the reduced matrix $\bolditt{V}$ however, they perform a \emph{Fuzzy C-Means} clustering algorithm, which is tuned with \emph{ANFIS} learning algorithm to optimize the hyperparameters of Fuzzy C-means \cite{AFAPLI2015}. The process returns a membership degree $\mu_{i, k}$ per document, indicating how close $i$th document is to $k$th cluster. 
\noindent
Flores \etal \cite{USCR2014} uses similar preprocessing approach to Cosma and Joy. They first process the documents by lower-casing them and removing repeated character, tabs with spaces. Then transform the documents into $3$-grams and weighting them by using a \emph{term frequency}. Finally, similarity is calculated using cosine similarity where $t$ is one of the 3-grams and $tf$ is the term frequency function \cite{USCR2014}. Formally this can be calculated in a same way as in equation \ref{cosine_sim_eqn} between two documents as

\begin{equation}
    sim(d_i, d_j) = \dfrac{\sum\limits_{t \in d_i \cap d_j} tf(t, d_i) tf(t, d_j) }
                          {\sqrt{\sum\limits_{t \in d_i} tf(t, d_i)^2 \sum\limits_{t \in d_j} tf(t, d_j)^2}}
\end{equation}

\noindent
Heblikar \etal \cite{Heblikar2015NormalizationBS} preprocesses also their documents by lower-casing, pruning repeated whitespace and removing single symbols. They then normalize the documents by considering most frequent terms, renaming similar terms under same symbols and ultimately filtering them completely out from the source codes. For detection phase, they use same approach as Flores \etal did in \cite{USCR2014} but use both 1-grams and 2-grams with \emph{term frequency - inverse document frequency} (tf-idf) weighting. Interestingly, also Ramírez-de-la-Cruz \etal in \cite{OTIOLSS2015} and Ramírez-de-la-Cruz \etal in \cite{ramirez2015high} decides to use cosine similarity and Jaccard coefficient. The only major difference being, that Ramírez-de-la-Cruz \etal uses additional structural and stylistic features, forming total combination of eight various similarity measurements \cite{OTIOLSS2015}. Where as Ramírez-de-la-Cruz \etal in \cite{ramirez2015high} uses cosine similarity with character 3-grams to calculate five different similarities: lexical, stylistic, comments, text\footnote{Referring here as any string passed in as an argument of a function} and structural. Lastly, Ohmann and Rahal proposes density-based clustering to form clusters of similar documents \cite{Ohmann2015}. Their similarity approach follows closely to other studies presented above: filtering and normalization as preprocessing, data format as word $n$-grams and similarity values gained by using cosine similarity. 

Tree-based studies mostly relies on calculating similarity between two tree structures $T_i, T_j$ obtained from the original documents $d_i, d_j$ by parsing them. For example Ng \etal first generates parse tree $T$ from the source code, then decomposes parse tree into subtrees $T' \subseteq T$ with respect to the functionality \eg imports are categorized together \cite{TBCFPD2012}. The similarity score is thus obtained by comparing trees with \emph{depth-first search} and summing the scores for all subtrees to form a similarity score between two documents. The similarity function between two documents can be expressed with the following definition where $simST$ is the similarity score between two subtrees obtained by comparing nodes and tokens 

\begin{equation}
    sim(d_i, d_j) = sim(T_i, T_j) = \dfrac{\sum\limits_{i, j}simST(T'_i, T'_j)}{10 \cdot |T'|} \cdot 100
\end{equation}

\noindent
Son \etal computes similarity value between two parse trees with a modified parse tree kernel \cite{AAPSCDPTK2013}. They define the kernel function $k$ via recursive function $C$ where $n$ is the node of a subtree $T'$. Function $C$ finds a maximal similarity between $n_i, n_j$ thus authors calls it as \emph{maximum node value}  

\begin{equation}
    k(T_i, T_j) = \sum\limits_{n_i \in T'_i} \sum\limits_{n_j \in T'_j} C(n_i, n_j)
\end{equation}

\noindent
The actual similarity between documents can be calculated via normalization \cite{AAPSCDPTK2013}

\begin{equation}
    sim(d_i, d_j) = \dfrac{k(T_i, T_j)}{\sqrt{k(T_i, T_i) \cdot k(T_j, T_j)}}
\end{equation}

% C(n_i, n_j) &= \lambda \prod \limits_{k}^{nc(n_i)} \left( 1 + \max\limits_{ch \in ch_{n_j}} C(ch_k(n_i), ch)\right)

\noindent
Asd

%   \cite{AIR2015} & \cite{Fu2017WASTKAW}


\newpage
\paragraph{Authorship identification}\mbox{}\\
a



%\begin{algorithm}[ht]
%\caption{See how easy it is to provide algorithms}
%\label{myFirstAlgorithm}
%\begin{algorithmic}
%\REQUIRE $a$
%\STATE $b = 0$
%\STATE $x \leftarrow 1:10$
%\FORALL{x}
%    \STATE $b = b+a$
%\ENDFOR
%\RETURN $b$
%\end{algorithmic}
%\end{algorithm}

\subsubsection{Accuracies}
\input{liter_review/accuracies.tex}

\newpage


\section{Methodology}
\subsection{Context}
% Course in Uni helsinki etc.
\subsection{Data}
\subsubsection{Preprocessing}
\subsubsection{}
\subsection{Research question}

\section{Results}

\section{Discussion}

\section{Conclusion}


%

% --- References ---
%
% bibtex is used to generate the bibliography. The babplain style
% will generate numeric references (for example [1]) appropriate for theoretical
% computer science. If you need alphanumeric references (e.g [Tur90]), use
%
%\bibliographystyle{babalpha-lf}
%
% instead.
\newpage
\bibliographystyle{babplain-lf}
\bibliography{references-fi}


% --- Appendices ---

% uncomment the following

% \newpage
% \appendix
% 
% \section{Esimerkkiliite}

\end{document}
