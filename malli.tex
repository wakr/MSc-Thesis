% --- Template for thesis / report with tktltiki2 class ---
% 
% last updated 2013/02/15 for tkltiki2 v1.02

\documentclass[english, grading]{tktltiki2}

% tktltiki2 automatically loads babel, so you can simply
% give the language parameter (e.g. finnish, swedish, english, british) as
% a parameter for the class: \documentclass[finnish]{tktltiki2}.
% The information on title and abstract is generated automatically depending on
% the language, see below if you need to change any of these manually.
% 
% Class options:
% - grading                 -- Print labels for grading information on the front page.
% - disablelastpagecounter  -- Disables the automatic generation of page number information
%                              in the abstract. See also \numberofpagesinformation{} command below.
%
% The class also respects the following options of article class:
%   10pt, 11pt, 12pt, final, draft, oneside, twoside,
%   openright, openany, onecolumn, twocolumn, leqno, fleqn
%
% The default font size is 11pt. The paper size used is A4, other sizes are not supported.
%
% rubber: module pdftex

% --- General packages ---

\PassOptionsToPackage{hyphens}{url}
\usepackage[hyphens]{url}
\usepackage[utf8]{inputenc}
\usepackage[T1]{fontenc}
\usepackage{lmodern}
\usepackage{microtype}
\usepackage{bbm} % indicator
\usepackage{amsfonts,amsmath,amssymb,amsthm,booktabs,color,enumitem,graphicx}
\usepackage[pdftex,hidelinks]{hyperref}
\usepackage{longtable}
\usepackage{listings}
\usepackage{diagbox}
\usepackage{array}


\usepackage{pgfplots}
\usepackage{pgfplotstable}
\pgfplotsset{compat=1.15}
\newlength\figureheight
\newlength\figurewidth

%tikz
\usetikzlibrary{shapes.geometric, arrows, patterns}
\tikzstyle{startstop} = [rectangle, rounded corners, minimum width=3cm, minimum height=1cm,text centered, draw=black]
\tikzstyle{io} = [trapezium, trapezium left angle=70, trapezium right angle=110, minimum width=3cm, minimum height=1cm, text centered, draw=black]
\tikzstyle{process} = [rectangle, minimum width=3cm, minimum height=1cm, text centered, draw=black]
\tikzstyle{decision} = [diamond, minimum width=3cm, minimum height=1cm, text centered, draw=black]
\tikzstyle{arrow} = [thick,->,>=stealth]

\usepackage{hhline}

\usepackage{parcolumns}
\usepackage{multicol}
\usepackage{multirow}
\usepackage{subcaption}

\usepackage{qtree}
\usepackage{newfloat}
\DeclareFloatingEnvironment[fileext=lod]{diagram}

\usepackage{dirtytalk}

\usepackage{algorithm}
\usepackage{algpseudocode}
%\usepackage[bottom]{footmisc}

% Automatically set the PDF metadata fields
\makeatletter
\AtBeginDocument{\hypersetup{pdftitle = {\@title}, pdfauthor = {\@author}}}
\makeatother

% --- Language-related settings ---
%
% these should be modified according to your language

% babelbib for non-english bibliography using bibtex
\usepackage[fixlanguage]{babelbib}
\selectbiblanguage{english}

% add bibliography to the table of contents
\usepackage[nottoc]{tocbibind}
% tocbibind renames the bibliography, use the following to change it back
\settocbibname{References}


% -- Handy shortcuts
\newcommand{\etal}{\textit{et al}. }
\newcommand{\ie}{\textit{i}.\textit{e}., }
\newcommand{\eg}{\textit{e}.\textit{g}. }
\newcommand{\cpp}{C\texttt{++}}
\newcommand{\bolditt}[1]{\mathbf{#1}}
\newcommand{\norm}[1]{\left\lVert#1\right\rVert}

\DeclareMathOperator*{\argmax}{argmax}
\DeclareMathOperator*{\argmin}{argmin}

% --- Theorem environment definitions ---
\newtheorem{lau}{Lause}
\newtheorem{lem}[lau]{Lemma}
\newtheorem{kor}[lau]{Korollaari}

\theoremstyle{definition}
\newtheorem{maar}[lau]{Definition}
\newtheorem{ong}{Ongelma}
\newtheorem{alg}[lau]{Algoritmi}
\newtheorem{esim}[lau]{Esimerkki}
\newtheorem{example}{Example}

% algorithmic
\algnewcommand\algorithmicinput{\textbf{Assume:}}
\algnewcommand\Assume{\item[\algorithmicinput]}

\theoremstyle{remark}
\newtheorem*{huom}{Huomautus}

\numberwithin{equation}{section} % equations as (chapter num. , i)

% --- tktltiki2 options ---
%
% The following commands define the information used to generate title and
% abstract pages. The following entries should be always specified:

\title{Automatic Software Plagiarism Detection}
\author{Kristian Wahlroos}
\date{\today}
\level{M.Sc. Thesis}
\abstract{Plagiarism is an act of copying where one doesn't rightfully credit the original source and the motivations behind plagiarism can vary from gaining economical advantage to even completing academic courses. Plagiarism exists in various domains, where people want to take credit from something they have worked on. These areas can include e.g. literature, art or software, which all have a meaning for an authorship.  

In this thesis, document similarity detection and authorship identification from source code are analyzed to build an automatic plagiarism detector. The term \textit{automatic} here refers to a system which requires as little as possible human intervention. The goal for our model is to point out possible plagiarism from a collection of documents, which in this thesis is specified as a collection of source code files written by various authors. Our data, which we will use to our statistical methods, consists of three datasets: (1) 33\,000 documents belonging to University of Helsinki's first programming course, (2) 15\,000 documents belonging to University of Helsinki's advanced programming course and (3) around 400 submissions for source code re-use competition. Statistical methods we apply in this thesis are inspired by the theory of search engines, and are data mining for the similarity detection and machine learning for the authorship identification.

The results show that...}

% The following can be used to specify keywords and classification of the paper:

\keywords{plagiarism; authorship identification; similarity detection}

% classification according to ACM Computing Classification System (http://www.acm.org/about/class/)
% This is probably mostly relevant for computer scientists
% uncomment the following; contents of \classification will be printed under the abstract with a title
%"ACM Computing Classification System (CCS):"
\classification{Information systems $\rightarrow$  Information retrieval $\rightarrow$  Retrieval tasks and goals $\rightarrow$  Near-duplicate and plagiarism detection\\
Information systems $\rightarrow$  Information retrieval $\rightarrow$  Retrieval tasks and goals $\rightarrow$ Clustering and classification\\
Information systems $\rightarrow$  Information systems applications $\rightarrow$  Data mining\\
Computing methodologies $\rightarrow$  Machine learning $\rightarrow$  Learning paradigms $\rightarrow$  Supervised learning\\
Computing methodologies $\rightarrow$  Machine learning $\rightarrow$  Learning paradigms $\rightarrow$  Unsupervised learning}

% If the automatic page number counting is not working as desired in your case,
% uncomment the following to manually set the number of pages displayed in the abstract page:
%
% \numberofpagesinformation{16 sivua + 10 sivua liitteissä}
%
% If you are not a computer scientist, you will want to uncomment the following by hand and specify
% your department, faculty and subject by hand:
%
% \faculty{Matemaattis-luonnontieteellinen}
% \department{Tietojenkäsittelytieteen laitos}
% \subject{Tietojenkäsittelytiede}
%
% If you are not from the University of Helsinki, then you will most likely want to set these also:
%
% \university{Helsingin Yliopisto}
% \universitylong{HELSINGIN YLIOPISTO --- HELSINGFORS UNIVERSITET --- UNIVERSITY OF HELSINKI} % displayed on the top of the abstract page
% \city{Helsinki}
%



% 10-15 pages abstract

\begin{document}

\lstdefinestyle{mystyle}{
    tabsize=2,
    breakatwhitespace=false,         
    breaklines=true,                 
    keepspaces=true,
    %numbers=left,
    showspaces=false,                
    showstringspaces=false,
    showtabs=false,
    numberstyle=\small,
    numbersep=8pt,
    columns=flexible,
    %framexleftmargin=15pt,
    xleftmargin=\parindent,
    basicstyle=\ttfamily\small,
    keywordstyle=\color{red}
}
 
\lstset{style=mystyle}


% --- Front matter ---

\frontmatter      % roman page numbering for front matter

\maketitle        % title page
\makeabstract     % abstract page

\tableofcontents  % table of contents

% --- Main matter ---

\mainmatter       % clear page, start arabic page numbering

\section{Introduction}

%What means term plagiarism?
%What is plagiarism?
%Why studied here --> 

Massive Online Courses (MOOCs) are a popular way to complete undergraduate courses offered by various institutes and universities. For example a course \emph{Circuits and Electronics} led by Massachusetts Institute of Technology and Harvard University, gathered around 155\,000 registered students from all over the world to a website called \emph{edX}\footnote{\url{https://www.edx.org/} Accessed 10th April 2018} in 2012 \cite{SLWCRFM2013}. The structure of \emph{Circuits and Electronics} consisted of two parts which are now common in majority of MOOCs: theory part and graded tasks which are offered weekly during the timespan of a course. 

The course \emph{Ohjelmoinnin MOOC} is an online programming course offered by University of Helsinki. It has a two-part structure; introduction and advanced course in Java programming language, where both are mandatory undergraduate-level courses including 14 weeks of total workload. During these weeks students follow the offered course material independently and submit their solutions to various programming tasks that are automatically tested and scored. If the participant is not a student in University of Helsinki, she can apply for a study right after completing the course and taking an exam, otherwise the student gains total of ten credits to her degree. As the nature of \emph{Ohjelmoinnin MOOC} is heavily score-based and students are free to choose their working hours without any major mandatory attendance, it can create a motivation to cheat among  students. Also the fact that there are over hundred students registered and many submissions sent by each student, makes it very hard for course staff to manually detect possible cheating. 

The word \emph{cheating} here refers to an act of plagiarism and one of the ways to define the verb \emph{plagiarize} is as \say{to steal and pass off (the ideas or words of another) as one's own}\footnote{\url{https://www.merriam-webster.com} Accessed 10th April 2018}, and the person conducting this act is called \emph{a plagiarist}. Source-code plagiarism on other hand, refers to the act of plagiarism that happens between software that is built from various source code documents. This kind of plagiarism can be also defined as \emph{source-code reuse}, which includes the following four facets \cite{TDSCP2008}: (1) copying others work without alterations, (2) copying and changing some parts of the code to fool a human inspector, (3) converting a solution from one language to another and (4) using code-generators to automatically create a solution. 


Source-code plagiarism in academia is considered as a serious offence and there often exists a zero tolerance for it \cite{TDSCP2008}. This is usually stressed at the start of courses and can lead to serious consequences ranging from rejecting the students current course registration to even suspension. Dick \etal points out that in some courses over 80\% of the students were found guilty of cheating if they were given a good enough opportunity for it \cite{Dick:2002:ASC:782941.783000}. Usual forms of cheating methods were found to be related to plagiarism: copying solutions from the web, sharing solutions with friends and excessive collaboration between students. 

The opinions about cheating motives varies between students and academics \cite{TDSCP2008}. Academics reported that cheating is due to three major factors: external pressure, the ease of sharing solutions and cultural differences. Students on the other hand, gave two major reasons under the study: time pressure and heavy workloads. Given that the MOOCs have time sensitive weekly assignments, the automatic scoring system, and freedom to complete the course wherever students want, can increase the motivation to cheat.

In this thesis we approach the problem of source code plagiarism detection with data mining and machine learning. By data mining we mean an approach that is able to use computers to find interesting patterns from the data, and by machine learning a statistical process which is able to make predictions using computers. For our proposed detection model, we first build two classifiers: identifying suspicious authors based on the similarity of documents and authorship identification that is able to predict the most likely author of a document. Using results from both of these classifiers, we propose a novel approach where the intersection of suspicious authors and candidate authors of a document is able to reveal possible cases of plagiarism. Suspicious authors are grouped together to reveal clusters of possible plagiarists, whereas authorship identification is used to detect if a writing style of an author matches her previous work. The intersection of these two models should thus minimize the amount of falsely accused people, as we can verify if the author is who she claims to be.  

Following three research questions are asked and answered in this study, which are all tied closely to the question \emph{How plagiarism can be automatically detected from source code documents?}

\begin{itemize}
    \item[Q1:] \emph{What kind of approaches exist to detect source code plagiarism?}
    \item[Q2:] \emph{What are the possible benefits of using code structure for plagiarism detection?}
    \item[Q3:] \emph{How can one reduce the amount of false accusations?}
\end{itemize}

\noindent
To answer these questions, we first conduct a literature review in which we  establish a categorization for techniques used in plagiarism detection. Then, we show how documents can be presented and retrieved in large-scale environment, and introduce the benefits of using the code structure within plagiarism detection. Finally, we evaluate the similarity detection and authorship identification individually and combine the best scoring models to see how false positives are affected, and how they are introduced in the model. 

Rest of this thesis is structured as follows: in Chapter 2 more detailed overview of source code plagiarism is given with a theory of classifiers, Chapter 3 presents the results of systematic literature review where we focus on data and methods applied in research, in Chapter 4 our method of using the result of two classifiers and the used real-life data sets are presented, Chapter 5 presents the results by comparing our method to two popular baselines. Chapter 6 discusses the results by answering previous research questions, discusses the shortcomings with our proposal and presents possible problems when automatic system is used to accuse students from plagiarism. 

\section{Background}

In this chapter we define the problem of plagiarism detection more formally, describe possible plagiarism strategies and give an overview to the similarity detection and authorship identification. We approach these latter two problems by first defining them, then showing how they tie closely to the domain of information retrieval, and finally give two real-life models. The first model is a probabilistic model able to predict the author based on the authors previous work, and the second model is a clustering algorithm able to group similar documents together. We start first by defining the problem of plagiarism detection.    

\newtheorem*{sc-plg}{Plagiarism detection}
\begin{sc-plg}
Given a set of documents $D = \{d_1, d_2, ..., d_n\}$ called as the corpus and a set of authors $A = \{a_1, a_2, ..., a_k\}$ who are writers of these documents, define a function $f$ that is able to classify which documents are plagiarized, and who are possible plagiarists from the set of authors $A$.
\end{sc-plg}

\noindent
The above formalization gives an overview of the problem that is studied in this thesis. Some aspects about the general problem have been simplified for this study, as for example we don't try to reveal the \emph{direction} of plagiarism, we ignore any possible data gathered from the creation process and we only consider authors inside a predefined set. This means that we try only to detect if possible plagiarism can be observed from the the collection of documents submitted by students. 

To get a better understanding of the details that are relevant to source-code plagiarism, instead of \eg detecting plagiarism from essays, we define some important themes and terms next. Starting from the definition of source code plagiarism, we show some common strategies of plagiarists and briefly introduce the underlying structure of a source code and existing tools to detect plagiarism.

\subsection{Source code plagiarism} \label{chap-bg-sc-plag}

Source code plagiarism refers to a plagiarism between source code files, which can happen in both academic programming courses as in software industry. Despite that in this thesis we focus on academic courses, both of these domains share a common problematic constraint which makes plagiarism detection often manually impossible. This constraint is simply the time constraint, creating a need for automatic detection tools as course administrators have limited hours to use for one course. 

In academia, the underlying motives behind source code plagiarism include concepts like \cite{PlagInProg1999}: ambiguity about what is considered as excessive collaboration between students, using other students work to gain grades and minimizing the work needed to complete the course. A study revealed three emerging plagiarism behaviors with take-home exams \cite{Hellas:2017:PTE:3059009.3059065}: help-seeking, collaboration and systematic cheating. Indicating that the most common type of plagiarism is accidental and done with other students from the same course.

Students committing plagiarism can have problems to define what they consider as source code plagiarism, and generally three common guidelines can be defined \cite{Pieterse2014DecodingCP}:

\begin{enumerate}
    \item[1)] Refactoring other students work, and submitting it as your own, is plagiarism
    \item[2)] If exercise templates are used, possible similarities between documents and templates are not plagiarism
    \item[3)] Submitting a direct copy of other students work is plagiarism
\end{enumerate}

\noindent
%Because of possible confusion about what is considered as plagiarism and the serious nature of plagiarism accusations in academia, it's preferable still to use some kind of human expert to judge to call out if a student has actually plagiarized someone \cite{Pieterse2014DecodingCP}. This allows the detection process to reveal candidates, which prunes heavily the amount of manual work needed.
Detecting 2. and 3. are straightforward; code templates can be filtered out from documents so that they contain only students own work and detecting direct copy can be easily found by using string matching techniques. However, the problem arises when students try to hide the plagiarism by mutating the directly copied document.

\paragraph{Plagiarism strategies}\mbox{}\\
Some common source code transformation techniques, often called as \emph{obfuscation strategies}, are targeted mainly towards two types of changes \cite{PlagInProg1999}: lexical and structural. Lexical changes doesn't require a deeper understanding of the logic and are doable with any \emph{integrated development environment} (IDE). Structural changes requires some understanding of the program logic, and includes modifications which change the layout of the source code but keeps the logic same. For example considering following clause with an operand \texttt{if(a == true)}. This can be written equally as \texttt{if(a == !false)}, keeping the logic same but mutating the lexical information.

Table \ref{tbl-plag-strat} shows some of the most common transformation targets in source codes. Given a source code from another student, plagiarist can apply above transformations and complicate the task for a human to spot plagiarism, or even confuse naïve methods. The motivation behind using these transformations is simple, plagiarists want to hide traces and thus, the detection method must be resilient against these strategies.

\newpage

\begin{table}[ht]
\centering
\caption{Common targets for transformations \cite{PlagInProg1999}. Lexical changes are superficial and easy to change, whereas structural changes require some understanding of the logic.}
\label{tbl-plag-strat}
\begin{tabular}{|l|l|} \hline
 \textbf{Lexical} & \textbf{Structural} \\ \hhline{|=|=|}
 Comments                    & Loops                          \\
 Formatting                  & Clauses                        \\
 Naming                      & Statement order                \\
                             & Operand order               \\ \hline
\end{tabular}
\end{table}



Plagiarism transformations defined in Table \ref{tbl-plag-strat} closely relate to an earlier study which characterized six levels of transformations \cite{Faidhi:1987:EAD:27319.27321}.

\begin{table}[ht]
\centering
\caption{Transformation levels.}
\label{tbl-plag-transf}
\scalebox{0.85}{
    \begin{tabular}{|c||c|p{5cm}|} \hline
     \textbf{Level of change} & \textbf{Target}  & \textbf{Example action}\\ \hhline{|=|=|=|}
     1 & Comments and indentation & Add extra spaces and newlines\\ \hline
     2 & Identifiers & Rename all variables\\ \hline
     3 & Declarations & Reorder functions\\ \hline
     4 & Modules & Merge functions\\ \hline
     5 & Statements & Use \texttt{for} instead of \texttt{while}\\ \hline
     6 & Logic & Change whole expressions\\ \hline
    
    \end{tabular}
    }
\end{table}

\noindent
Applying all of these transformations one after another from Table \ref{tbl-plag-transf}, makes the detection of plagiarism very difficult, as the plagiarized document diverges too much from the original document and hides most of the traces that could be used for detection. However, as the textual information changes, plagiarists still try to maintain the same logic between original and copied documents. This means that there still exists some kind of similarity, but now the similarity can not be found directly from the textual representation of a source code. Information about the logical structure is thus crucial and accessible when source code is parsed to a tree format.

\paragraph{Code structure}\mbox{}\\
Source code is a structured text, made of keywords and user-defined variables. To write a running program, one must know the rules \ie the \emph{grammar} of a language, which is usually represented as the order in which various keywords and variables must follow each others. A compiler is the core of a programming language and is able to transform a source code into a machine code. When rules must be interpreted by the compiler, it uses a \emph{parse tree} that is generated from the source code \cite{johnson1975yacc}. This parse tree captures the syntax and semantics of a source and the abstracted version of it is called as the \emph{abstract syntax tree}. 

Consider for example storing an integer value to a variable. The source code in JavaScript and its syntax tree is visible from the following diagram.

\begin{diagram}[ht]
\centering
\scalebox{1}{
\Tree[.VariableDeclaration [.Identifier "a" ] [.Literal "5" ] ]
}
\caption{Example syntax tree for the expression \texttt{var a = 5;}}
\label{diag-parse}
\end{diagram}

\noindent
Pruning the leaves of diagram \ref{diag-parse} leads to a more general expression that captures the logic of the source code, becoming resistant against most of the transformations given in Table \ref{tbl-plag-transf}. This gives the ability to detect similar \emph{structure} rather than similar \emph{tokens}, where latter is more vulnerable to simple transformations. For example changing the name of the identifier or the value of the literal won't affect the upper tree structure at all.

\paragraph{Tools}\mbox{}\\
Because plagiarism is kept as a serious offence in academia, a lot of various detection software has been made to detect it. Novak lists seven of the most well-known tools in a review \cite{RSCAD2016}: \emph{MOSS, JPlag, SPLaT, SIM, Marble, Plaggie} and \emph{Sherlock}. These tools can be classified into five different categories based on methods used: text, token, graph, tree and hybrid. Among these tools, the most common way to detect plagiarism is a five step approach: pre-process documents, tokenize documents, exclude templates, calculate similarities and find suspects using the similarity scores. Therefore, it's notable that each of these tools are trying to purely calculate the similarity between documents.

As an example, JPlag is tool targeted for Java code \cite{prechelt2002finding}. It works by utilizing the program structure, transforming the program code into sequence of tokens by traversing the parse tree of a program and using predefined token correspondence to form a token stream which represents the source code. To form similarity score between two programs, a string matching algorithm \emph{Greedy String Tiling} \cite{SSGST1993} is applied and the summed length of all matches is calculated. This results a similarity value between zero and one where the value one means that two programs are exact copies of themselves. Makers of JPlag also claim that their tool is resilient against most common obfuscations made by plagiarists \eg renaming and reordering. 

\subsection{Similarity detection} \label{chap-sd}

Similarity detection, or code clone detection, focuses directly on finding similar functionality from a set of source code documents. We can define it formally as following.

\newtheorem*{smd1}{Similarity detection}

\begin{smd1}
Given a set of source code documents $D = \{d_1,...,d_n\}$ called as the corpus, define normalized similarity function $sim: d_i, d_j \rightarrow [0, 1]$ where $1 \leq i, j \leq n$, such that $sim(d_i, d_j) = sim(d_j, d_i)$ and $sim(d_i, d_i) = 1$. In other words similarity score is same regardless of the order of two documents and similarity between same documents is maximal. With an optional threshold $\theta \in [0, 1]$ one can define the limit when two source codes are considered as too similar. With this definition, any pair of source code file $(d_i, d_j) \in D \times D$ can also be presented as a triplet $(d_i, d_j, s)$, where $s$ is the similarity value between documents. 
\end{smd1}

%\noindent
%The definition above is flexible enough to support multiple solutions for the same task, the only restriction being that one must define a function that is able to return a score that either captures the textual or functional similarity. Defined method should be as resilient as possible against transformations given in table \ref{tbl-plag-strat} and table \ref{tbl-plag-transf}, so that two logically similar programs will get a high score even if their textual information differs. Scoring here refers to the value given by similarity function, which can be thought as a distance between documents.

The approach one takes to define the similarity function, is ultimately based on how source code document is seen as a data \cite{Roy:2009:CEC:1530898.1531101}: document consisting of plain text, series of tokens, syntax tree, series of metrics or as a graph. If the document is seen purely as a sequences of characters, one can use naïve methods like string matching techniques to detect fragments of copy and paste, which requires no pre-processing. Other data formats require some kind of transformation or an extraction process. 


If one represents the source code simply as metrics, then there exists two major categorization for those metrics \cite{Roy:2009:CEC:1530898.1531101}: attribute-counting metrics and structure metrics. Attribute-counting refers to high-level features which can be extracted directly from the plain source code \eg line counts and the amount of whitespace, whereas structure metrics use the underlying structure of the source code to capture the low-level representation \cite{Verco:1996:SDS:369585.369598}.


The core process to detect similarity can be visualized to Figure \ref{fig-sd-flow}, which follows the general structure seen in many state of the art systems \cite{Roy:2009:CEC:1530898.1531101}.

\begin{figure}[!ht]
\centering
\vspace{0.5cm}
\scalebox{0.75}{
   \begin{tikzpicture}[node distance=2cm, baseline]

    \node (start) [io] {Corpus};
    \node (pro1) [process, right of=start, xshift=4.1cm] {Pre-process};
    \node (pro2) [process, right of=pro1, xshift=5cm] {Transform};
    \node (pro3) [process, below of=pro2] {Clone detection};
    \node (end) [io, below of=start] {Suspects};
    
    \draw [arrow] (start) --  node[anchor=south] {Raw source code} (pro1);
    \draw [arrow] (pro1) -- node[anchor=south] {Partition} (pro2);
    \draw [arrow] (pro2) -- node[anchor=east] {Intermediate representation} (pro3);
    \draw [arrow] (pro3) -- node[anchor=north] {Pairwise similarity} (end);
    
    \end{tikzpicture}
}
\caption{Similarity detection process for source code documents. Documents are transformed into intermediate representation so similarity scores can be calculated.}
\label{fig-sd-flow}
\end{figure}

\noindent
In Figure \ref{fig-sd-flow} after the corpus has been defined, pre-process stage takes as an input the unmodified source codes to perform two key tasks: to remove unnecessary segments and to determine the level of comparison granularity. The granularity one chooses can range from function-level to document-level, depending how accurately the results should pinpoint plagiarism. After the source code is partitioned, it is transformed into intermediate representation which consists of two parts: extraction and normalization. In extraction the data is modified so it is usable in similarity function, and covers things like parsing, tokenization or building control flow from the given code. In normalization one applies techniques which reduce the variation between documents \cite{Roy:2009:CEC:1530898.1531101}: comments and whitespace removal, uniforming user-defined identifiers and removing anything which is not crucial for the detection process.

However, the most key issue regarding to similarity detection are false-positives that can be handled by manual verification after the suspects are gathered \cite{ Roy:2009:CEC:1530898.1531101, Verco:1996:SDS:369585.369598}. This is often mandatory step as the detection tools simply try to find similarity between documents, but this similarity can be pure coincidence for example when the solution space for a given task is limited. 



\subsection{Authorship identification} \label{chap-ai}

Authorship identification deals with the issue of trying to name the author of a document given some previous work of the author. This problem can be seen as a classification task \cite{KRSUL1997233}, thus we can define authorship identification formally as following.

\newtheorem*{aui1}{Authorship identification}
\begin{aui1}
Given a set of documents $D$, a set of authors $A$ and a function $f: D \rightarrow A$ that identifies the writer by assigning every source code document $d \in D$ to one author $a \in A$. Estimate $f$ with $\hat{f}$, a classifier that treats every document as a feature vector $\bolditt{x}$ where $x_i \in \mathbb{R}$, and every known class as a vector $\bolditt{y}$ where $y_i \in \{0, 1\}$. The binary value represents boolean value if the $i$th author is the predicted author for a given document, which means that if dimension of the vector $\bolditt{y}$ is $\mathbb{R}^n$, then there are $n$ authors $|A| = n$. The predicted author $\hat{y}$ can be thus expressed with $\hat{f}(\bolditt{x}) = \hat{\bolditt{y}}, \hat{y} \in \hat{\bolditt{y}}$.
\end{aui1}

Therefore, the classifier should be able to discriminate between writing styles of different authors, which in programming refers to the habits \ie programming style. Habits are restricted by the grammar of the chosen programming language, but commonly refers to everything that is controllable by the author \eg how one names variables or uses spacing. As a problem it differs from the similarity detection where we want to find maximum equivalency between programs, because now we are more interested of detecting the unique programming style.  

One way to represent programming style is by using software metrics \cite{KRSUL1997233}. Software metrics can be put into roughly three categories: layout being fragile metrics which are easily transformed by the IDE, style which are non-fragile metrics related to layout and lastly structure which can capture experience and ability of the programmer. Because source code can be also thought as a text written with a specific language, natural language processing (NLP) techniques can be applied, thus we can form a five-level categorization for stylistic features called \emph{stylometrics features}  \cite{Stamatatos:2009:SMA:1527090.1527102}. Categorization for stylometric features is visualized in Table \ref{tbl-ai-stylomet}, where semantic features are the most difficult to form as they require understanding deeper meaning of the written source code.

\newpage

\begin{table}[ht]
\centering
\caption{Five-levels of stylometrics features. More external information is required on each level.}
\label{tbl-ai-stylomet}
\begin{tabular}{|c|c|} \hline
\textbf{Category}             & \textbf{Feature examples} \\ \hline
Character            & Character subsequences, types, compression            \\
Lexical              & Token statistics, word sequences                   \\
Syntactic            & Errors, expression usage, keywords, parse tree        \\
Application-specific & Indentation, language-specific constructs \\ 
Semantic             & Synonyms, functional dependencies                \\\hline
\end{tabular}
\end{table}

In natural language authorship analysis, statistical methods are often being used \cite{Stamatatos:2009:SMA:1527090.1527102}, and more specifically machine learning to find reoccurring patterns that are able to distinguish between writing styles. The training of these statistical models can happen in two ways: via profile-based or via instance-based. In profile-based approach, all documents that are presented as observed data per author are concatenated into one file. In instance-based learning, each text is used as an individual data point. If we know that each document belongs to one author, then the problem can be thought as an instance of \emph{a multiclass classification} \cite{Stamatatos:2009:SMA:1527090.1527102}, where author pool of size $n$ can be encoded into binary vector $\bolditt{y} = [y_1, y_2, \cdots, y_n], y_i \in \{0, 1\}$ and only one value of the vector $\bolditt{y}$ can be one.


%Despite the choice of, one is able to reduce the problem into multiclass classification \cite{Stamatatos:2009:SMA:1527090.1527102}, where author pool of size $n$ can be encoded into binary vector $\bolditt{y} = [y_1, y_2, \cdots, y_n], y_i \in \{0, 1\}$. With this representation if the $i$th candidate is the actual author, the value of that field will be one and the rest are zeroes. 


\subsection{Information retrieval} \label{chap-IR}

Before we can apply any statistical models to our problem, we need a way of properly handling the collection of documents. One way is to think the problem of plagiarism detection as retrieving certain kinds of documents from a collection. Therefore we next introduce some topics from the \emph{information retrieval}.

Information retrieval (IR) is a collection of strategies of finding documents from large collections \cite{Manning:2008:IIR:1394399}. These documents are often represented as unstructured text and possible methods covers topic like: clustering documents to find similar documents, classifying documents based on their content, ranking text for query search and building search engines. In this thesis as we mainly focus on finding similarities between documents and how to classify the author, we disregard some of the query-based focus of IR and use techniques that are relevant to plagiarism study \ie how document can be represented for statistical models and how distance between two documents can be calculated.




\subsubsection{Document representation} \label{chap-IR-document-repr}

As we want some form of numerical way \ie vector to represent one document, we use following IR-related concepts to express the documents in our corpus: \emph{vector space model} which captures algebraically the representation of the document and a \emph{weighting scheme} which helps to give more importance to specific terms.


\paragraph{Vector space model}
%Tokenization

One form of vector space model is called \emph{a binary term-document incidence matrix} \cite{Manning:2008:IIR:1394399}, which represents documents as columns of the matrix and terms as the rows. Terms are gathered by tokenization procedure which divides single document into units that are often \emph{words} of the document, but can also use adjacent words. 

Let $\bolditt{M}_{n \times k}$ be this matrix having $n$ terms and $k$ documents, then the value of $\bolditt{M}_{d, t}$ \ie term $t$ appearing in document $d$, is 1 if it appears at all and zero otherwise. Table \ref{tbl-binmatr} shows example matrix build from programs in Appendix \ref{appendix:programs}.  

\begin{table}[ht]
\centering
\caption{Example of a binary term-document incidence matrix for three sample programs in Appendix \ref{appendix:programs}.}
\label{tbl-binmatr}
\begin{tabular}{|c|c|c|c|} \hline
      \backslashbox{\bf Term}{\bf Document} & A & B & C \\ \hline
\texttt{public} & 1 & 1 & 1 \\
\texttt{sum}   & 0 & 1 & 0 \\
\texttt{double} & 1 & 1 & 0 \\
$\vdots$ & $\vdots$ & $\vdots$ & $\vdots$ \\
\texttt{b} & 1 & 1 & 1 \\ \hline
\end{tabular}
\end{table}

\noindent
Taking a transpose of the values in Table \ref{tbl-binmatr} gives a document-term matrix, where one row represents now document having $n$ dimensions, reserving one dimension for each term occurrence. For example the representation of document A is the vector $\bolditt{a} = [1, 0, 1, \cdots, 1]$. This is referred as the \emph{bag of words} model, because it treats every word as independent event, losing the information about ordering of the words and adjacent words \cite{Manning:2008:IIR:1394399}. 

%Unigrams can be thus expressed as a product of term probabilities 

%\begin{equation}
%    P(t_1, t_2, t_3, t_4) = P(t_1)P(t_2)P(t_3)P(t_4)
%\end{equation}

A binary term-document incidence matrix is however very naïve, giving the same value despite the times term appears in a document. The solution is to use a  method called \emph{term weighting}, which is able to assign non-binary value for terms. 

\paragraph{Term weighting schemes}

One simple scheme is called \emph{term frequency}, which is the occurrence of term $t$ in document $d$ denoted by $tf_{t, d}$ \cite{Manning:2008:IIR:1394399}. Let $f_{t, d}$ denote the raw frequency count, then term frequency can be given as $tf_{t, d} = f_{t, d}$ and normalized by dividing the raw frequency with the total frequency over every term in document

\begin{equation} \label{eq-tf}
    tf_{t, d} = \dfrac{f_{t, d}}{\sum \limits_{t' \in d} f_{t', d}}
\end{equation}

To scale down the most frequently appearing terms, one can use \emph{inverse document frequency} which boosts the weights of rare occurring terms \cite{Manning:2008:IIR:1394399}. Inverse document frequency is defined as

\begin{equation}
    idf_t = \log \dfrac{N}{df_t}
\end{equation}

\noindent
Where $N$ is the total amount of documents and $df_{t}$ is the count of documents that contains the term $t$.

Now by taking the product of the term frequency and inverse document frequency (TF-IDF), we get a weight for each term appearing in a document

\begin{equation}
    tf{\text -}idf_{t, d} = tf_{t, d} \cdot idf_{t}
\end{equation}

By using TF-IDF weighting scheme, we are able to discriminate between documents despite their length and diminish the problem of frequently appearing terms that are introduced often in programming. Terms appear frequently in programming because the language is very structured and defined by a finite amount of keywords. 

\subsubsection{Document similarity} \label{chap-bg-sim}

As we have a way of expressing a document as a vector, we are interested to be able to calculate similarity between two documents, which is crucial part for similarity detection. One way of doing this is by using \emph{cosine similarity}.

\paragraph{Cosine similarity}

Cosine similarity measures the similarity between two documents by calculating the cosine of the angle between the document vector representations \cite{Manning:2008:IIR:1394399}. 

Let $\bolditt{x}, \bolditt{y}$ be these vectors for documents $d_1, d_2$, then

\begin{equation} \label{eq-cosine-orig}
    sim(d_1, d_2) = \cos(\theta) = \dfrac{\bolditt{x} \boldsymbol{\cdot} \bolditt{y}}
                          {\norm{\bolditt{x}}_2 \norm{\bolditt{y}}_2} = 
                          \dfrac{\sum \limits_{i=1}^n x_i y_i}
                                {\sqrt{\sum \limits_{i=1}^n {x}_i^2} \sqrt{\sum \limits_{i=1}^n y_i^2}}
\end{equation}

\noindent
The dot-product is normalized in Equation \ref{eq-cosine-orig} with Euclidean norm and because weights derived from TF-IDF are non-negative, the cosine similarity gets values between zero and one \cite{Manning:2008:IIR:1394399}. Values closer to one indicate high content similarity, whereas closer to zero indicate dissimilarity. Complement of the cosine similarity called \emph{cosine distance} can be calculated by subtracting the similarity from one \ie $d = 1 - \cos(\theta)$.

\subsubsection{Retrieval metrics}

Having a way to retrieve candidate documents for possible plagiarism creates a need to justify how well the retrieval process is performing. For evaluating the retrieval method in a binary case, three important metrics have been defined \cite{Manning:2008:IIR:1394399}: precision, recall and $F_1$-score. To express these metrics more clearly, we use a confusion matrix which has four fields: true positive (TP), true negative (TN), false negative (FN) and false positive (FP). These four fields are visualized in Table \ref{tbl-confmatr-orig}.


\begin{table}[ht]
\centering
\caption{Confusion matrix which helps to visualize the error in retrieval \cite{Manning:2008:IIR:1394399}.}
\label{tbl-confmatr-orig}
\begin{tabular}{c|c|c}
          & \bf Relevant & \bf Irrelevant \\ \hline
\bf Retrieved & TP      & FP        \\
\bf Rejected  & FN      & TN       
\end{tabular}
\end{table}

Precision, recall and $F_1$-score can be all defined by using the cells of the confusion matrix. Both precision and recall count the rate of true positives to falsely retrieved documents, and balancing between these values requires often knowledge about the domain. If precision is preferred to be high, model is able to correctly retrieve greater portion of correct positive cases and if recall is preferred, model is able to retrieve high portion of relevant documents. Formally, precision and recall is defined as

\begin{align}
    \text{Precision } &= \dfrac{TP}{TP + FP} = \dfrac{|\text{relevant} \cap \text{retrieved}|}{|\text{retrieved}|}\\
    \text{Recall } &= \dfrac{TP}{TP + FN} = \dfrac{|\text{relevant} \cap \text{retrieved}|}{|\text{relevant}|}
\end{align}

$F_1$-score combines precision and recall, giving an average between precision and recall. It's defined as

\begin{equation}
    F_1 = 2 \cdot \dfrac{\text{Precision} \cdot \text{Recall}}{\text{Precision} + \text{Recall}}
\end{equation}

All of the above metrics help to evaluate how well the retrieval process is performing but require some form of process where one decides if a document is relevant or not. If the data is classified and we want to retrieve documents belonging to certain class, then the process can is automatically evaluable. Otherwise some sort of human expertise is needed for the evaluation.

\subsection{Document classification} \label{chap-bg-classification}

We can formulate the document classification problem as $\gamma: \mathbb{X} \rightarrow \mathbb{C}$ \cite{Manning:2008:IIR:1394399}, approximating a function $\gamma$ that maps data $\bolditt{x} \in \mathbb{X}$ to class $c \in \mathbb{C}$. For example \emph{a binary classifier} would choose between a positive and a negative class $\mathbb{C} = \{+, -\}$, whereas \emph{a multiclass classifier} chooses between multiple classes $\mathbb{C} = \{c_1, c_2, \cdots, c_n\}$ for a given document.

To be able to classify documents algorithmically into some predefined classes, the classifier must learn a way to predict outputs from inputs \cite{hastie_09_elements-of.statistical-learning}. That is, given some $d$-dimensional data $\bolditt{x} \in \mathbb{R}^d$, the classifier $\gamma$ must predict the response variable $y$ which represents the class. To make the prediction, the classifier is supported with some observed data as a training data represented in matrix format $\bolditt{X}_{n \times d}$ and predefined response variables in a column vector $\bolditt{y}_{n \times 1}$ \cite{hastie_09_elements-of.statistical-learning}. This so called \emph{training} of the model refers that the classifier can use some part of the total data to tune its internal parameters, so when a data point outside the training set is given, the classifier is able to give prediction for it based on the data it has seen already. This kind of setting is also called as \emph{supervised learning} as we have some data that guides the process, and the analogue for supervised learning can be thought as learning with a teacher \cite{hastie_09_elements-of.statistical-learning}.

In this case as we know the output $y$ of each observed point $\bolditt{x}$, we want to have similar data to be predicted with the same output. The prediction the algorithm gives can be noted as $\hat{y}$ for $\bolditt{x}$, and because this value is just a prediction, evaluation is needed for the algorithm to be able to chance its learning into right direction. This evaluation happens by penalizing wrong predictions with a loss function $L: \hat{\theta} \times \theta \rightarrow \mathbb{R}$ \cite{hastie_09_elements-of.statistical-learning}, where $\hat{\theta}$ is the prediction that the classifier gives and $\theta$ the value we want the prediction to be. For example a loss function able to penalize categorical predictions, is called \emph{0-1 loss} and formulated as $L(\hat{y}, y) = I(\hat{y} \neq y)$, where $I$ is the indicator function. 


%IR pgs. 259 - 273, Element. stat pgs. 37, 21

\paragraph{Naïve Bayes}

%A Comparison of Event Models for Naive Bayes Text Classification

Naïve Bayes is a probabilistic model, which is often used as a baseline model in text classification \cite{acemNBtc2001}. It applies the \emph{Bayes' theorem} to estimate the parameters of the classifier \ie conditional probabilities with respect to data. Bayes' theorem is generally given for events $A, B$ as 

\begin{equation} \label{eq-naive-bayes}
    P(A \mid B) = \dfrac{P(A \cap B)}{P(B)} = \dfrac{P(A)P(B \mid A)}{P(B)}
\end{equation}

\noindent
Where $P(B)$ can be expressed by the law of total probability
\begin{equation}
    P(B) = P(B \mid A)P(A) + P(B \mid \lnot A)P(\lnot A)
\end{equation}

When the area of interest is classification, we denote the probabilities of events $A,B$ as the \emph{prior} and \emph{likelihood}. Prior being in this case the probability of a class appearing in data $P(y=c), \; c \in \mathbb{C}$ and likelihood the likeliness of a document belonging to a class $P(\bolditt{x} \mid y=c)$. Equation \ref{eq-naive-bayes} can be rewritten as \cite{Zhang04theoptimality, acemNBtc2001}

\begin{equation} \label{eq-naive-bayes-class}
    P(y \mid \bolditt{x}) = \dfrac{P(y)P(\bolditt{x} \mid y)}{P(\bolditt{x})} = \dfrac{P(y)P(\bolditt{x} \mid y)}{\sum \limits_{c \in \mathbb{C}} P(y=c)P(\bolditt{x} \mid y=c)}
\end{equation}

In Equation \ref{eq-naive-bayes-class}, the denominator remains constant because $\bolditt{x}$ is kept unchanged as it's the sum over every known class, and therefore the Equation \ref{eq-naive-bayes-class} is proportional to the product between prior and likelihood \cite{Manning:2008:IIR:1394399}

\begin{equation}
    P(y \mid \bolditt{x}) \propto P(y)P(\bolditt{x} \mid y)
\end{equation}

Because the underlying real distribution is unknown, prior and likelihood needs to be estimated from the training data. Estimated prior can be calculated from the relative frequency \ie number of samples belonging to the class $c$ divided by the total number of observations \cite{Manning:2008:IIR:1394399}.

\begin{equation}
    \hat{P}(c) = \frac{\#c}{|\mathbb{X}|}
\end{equation}

\noindent
 To calculate the estimated likelihood $\hat{P}(\bolditt{x} \mid y)$, one uses the assumption that features represented in $\bolditt{x}$ are conditionally independent with respect to each other. This assumption simplifies the likelihood by using the chain rule \cite{Manning:2008:IIR:1394399}

\begin{equation}
    \hat{P}(\bolditt{x} \mid y) = \hat{P}(x_1, x_2, \cdots, x_n \mid y) = \prod \limits_i^n \hat{P}(x_i \mid y)
\end{equation}

To assign data into a class, the most probable class is chosen \cite{Zhang04theoptimality, acemNBtc2001, Manning:2008:IIR:1394399}. This is referred to also as \emph{maximum a posteriori} (MAP), and the final class assignment \ie the result of the classifier $\gamma$, is expressible as

\begin{equation} \label{eq-mapp}
    \hat{y} = c_{\text{map}} = \argmax \limits_{c \in \mathbb{C}} \hat{P}(c) \prod \limits_i^n \hat{P}(x_i \mid c)
\end{equation}

\noindent
This means that the most likely class for a data point is the class which maximizes the posterior, which again is proportional to calculating product between prior and posterior. All values for a 

A variant of Naive Bayes called \emph{Multinomial Naïve Bayes}, is able to form the likelihood by assuming underlying multinomial distribution \cite{acemNBtc2001}. Given the problem of document classification and the feature vector $\bolditt{x}$ represented as term frequencies of vocabulary $\mathbb{V}$, the conditional probability of Equation \ref{eq-mapp} can be given in similar way as term-frequency function in Equation \ref{eq-tf}. Contrast to it we can add a smoothing called \emph{Laplace smoothing}, to eliminate the problem with terms appearing zero times \cite{Manning:2008:IIR:1394399}, which can happen because the vocabulary $\mathbb{V}$ is formed from training set. The smoothed version of the conditional probability using frequencies is given as

\begin{equation} \label{eq-laplace}
    \hat{P}(x \mid c) = \hat{P}(t \mid c) = \dfrac{f_{t, c} + 1}{\sum \limits_{t' \in \mathbb{V}} (f_{t', c} + 1)} 
\end{equation}

\noindent
In Equation \ref{eq-laplace} $f_{t, c}$ is the frequency of term $t \in \mathbb{V}$ appearing in class $c \in \mathbb{C}$, so the conditional probability of a data point given a class is estimable from the smoothed relative frequency of the term $t$ that the point $x$ represents. 

It has been shown that TF-IDF weighting scheme improves the classification results even as TF-IDF weights are non-discrete like raw term frequencies are \cite{Kibriya:2004:MNB:2146834.2146882}. This means that all documents can be efficiently represented as vectors of TF-IDF weights for Multinomial Naive Bayes.


\subsection{Document clustering}

Document clustering is a process that is able to group the set of documents, so that their similarities is maximized \ie documents belonging to the same group are as similar as possible. This is relatively easy task for a human to do manually for a small set of documents, but in order to perform this task automatically in large scale we resort to \emph{unsupervised learning}. 

Unsupervised learning can divide the observed data \ie documents, into subgroups called \emph{clusters} \cite{hastie_09_elements-of.statistical-learning}. The main difference to supervised learning (classification) is that when the data is represented as a sequence $X = (\bolditt{x}_1, \bolditt{x}_2, \cdots, \bolditt{x}_n)$, where $\bolditt{x}_i \in \mathbb{R}^d$ is $d$-dimensional feature vector which represents the $i$th document, we don't have the sequence of response variables $\bolditt{y} = (y_1, y_2, \cdots, y_n)$ to guide the process. Thus there is no loss function which is dependent from the true classes of the data, which leads to the situation where distribution of the data determines the classes \cite{Manning:2008:IIR:1394399}. The performance of the unsupervised model can be therefore very subjective, requiring some kind of prior domain knowledge \cite{hastie_09_elements-of.statistical-learning}. 

As an example, the Figure \ref{fig-clust-example} visualizes two-dimensional data generated from three separate distributions. 

\begin{figure}[ht]
\centering
\setlength\figureheight{7cm}
\setlength\figurewidth{7cm}
\input{plots/example_datadistr_clust.tikz}
\caption{Data points centered around three distributions with respect to the means and variances. Three separable clusters are visible.} \label{fig-clust-example}
\end{figure}

\noindent
Because we know how the data was generated in Figure \ref{fig-clust-example}, we are able by prior knowledge and by visually to divide the space exactly into three regions. However, if the data would be more uniformly distributed and one could not say the exact amount of regions, then this task requires more knowledge about how two data points are able to have similar location. When considering for example plagiarism between documents, we are highly interest of cases where two or more documents are too similar to each others. 

%If Figure \ref{fig-clust-example} would represent documents in two-dimensional plane, and the clustering process would determine which documents belong to the same group, then the process of dividing data points into clusters would produce much more clusters than three.

The normalized similarity value $s \in [0, 1]$, or respectively distance value $d = 1 - s$, is defined before the clustering algorithm is executed \cite{hastie_09_elements-of.statistical-learning}, and it ultimately controls what kind of cluster are being formed. Distances between data points can be precomputed into matrix of documents $\bolditt{M}_{d \times d}$, where $\bolditt{M}_{i, j}$ is the similarity, or distance value between two documents $d_i$ and $d_j$. 

We next give a brief formalization for the problem of clustering and then introduce two different unsupervised clustering algorithms: \emph{K-means clustering} and \emph{DBSCAN}.

\newtheorem*{docclus}{Document clustering}
\begin{docclus}
Given a set of datapoints $X = \{\bolditt{x}_1, \bolditt{x}_2, \cdots, \bolditt{x}_n\},  \; \bolditt{x}_i \in \mathbb{R}^d$ which represents the documents, define assignment $\gamma: X \rightarrow \{1, \cdots, k\}$ where $k$ is the total amount of clusters \cite{Manning:2008:IIR:1394399}. The set of clusters can be notated by $\Omega = \{\omega_1, \omega_2, \cdots, \omega_k\}$ and each document belongs to some cluster $\forall d \in \omega$. 
\end{docclus}


\paragraph{K-means clustering}

K-means clustering requires the $k$ parameter to be predefined and it assumes there exists a \emph{centroid} \ie a mean point, for every cluster. These \emph{cluster centroids} are notated as $C = \{\boldsymbol{\mu}_1, \boldsymbol{\mu}_2, \cdots, \boldsymbol{\mu}_k\}, \; \boldsymbol{\mu}_i \in \mathbb{R}^d$ \cite{Manning:2008:IIR:1394399}. To assign a data point to a cluster, one calculates the \emph{squared Euclidean distance} from a point to a centroid $\norm{\bolditt{x}_i - \boldsymbol{\mu}}^2$, and minimizes this distance. In other words data point is assigned to the same cluster as the nearest centroid. The K-means algorithm works iteratively by updating the cluster assignments for each point and calculating new centroids until the algorithm converges. Convergence can be decided in multiple ways and one of those is that no new assignments has been done when all data points are iterated. 


Algorithm \ref{alg-kmeans} shows the pseudocode for K-means. In it, centroids are first chosen randomly from the set of data points with \textproc{InitCentroids}-function. Then iteratively until there are no further updates to centroids $C$, $k$ clusters are first initialized as empty sets. Next cluster assignments are calculated from the set of data points with respect to Euclidean distance to the nearest cluster. After every loop, new centroids are calculated by taking the mean of assigned data points per cluster in \textproc{UpdateCentroids}. The return value of the K-means will be $k$ centroids which represent the middle points of a cluster, and cluster assignments $\Omega$ indicating which data point belongs to which cluster.

\clearpage

\begin{algorithm}[ht]
\caption{K-means algorithm \cite{Manning:2008:IIR:1394399}}
\label{alg-kmeans}
\begin{algorithmic}

\Require Set of datapoints $X$
\Require Amount of clusters $k$
\Procedure{K-means}{$X, k$}
   \State $C  \leftarrow $ \Call{InitCentroids}{$X,k$}
   \While{stop criterion has not been met}
       \For{$i=1$ to $k$}
            \State $\omega_i \gets \{\} $
       \EndFor
        \For{$j=1$ to $|X|$}
            \State $l \gets \argmin_{l} \norm{\bolditt{x}_j - \boldsymbol{\mu}_l}^2$
            \State $\omega_l \gets \omega_l \cup \bolditt{x}_j$
        \EndFor
       \State $C \gets $ \Call{UpdateCentroids}{$\Omega$}
   \EndWhile
\State \textbf{return} $C, \Omega$
\EndProcedure

\end{algorithmic}
\end{algorithm}

\noindent 
Visualization of the clustering result using K-means for the same data as in Figure \ref{fig-clust-example} is seen in Figure \ref{fig-kmeans-example}.

\begin{figure}[!h]
\centering
\setlength\figureheight{7cm}
\setlength\figurewidth{7cm}
\input{plots/example_kmeans.tikz}
\caption{Result of K-means clustering after converge. Crosses indicate the cluster centroids, other colors cluster assignments. Parameter $k$ is set to 3 so three different clusters have been discovered by the algorithm.} \label{fig-kmeans-example}
\end{figure}

%https://www.youtube.com/watch?v=0MQEt10e4NM

The drawback with K-means clustering is that one must specify the parameter $k$ before the clustering \cite{hastie_09_elements-of.statistical-learning}. When for example detecting similar documents, there is no indication beforehand that how many documents should be grouped together, and therefore pre-estimating number of clusters can be very difficult. To overcome this issue, one can utilize the density of the data points rather than direct distance between them.

\paragraph{DBSCAN}

Density-based spatial clustering of applications with noise (DBSCAN) can produce clustering by using only the density information, label some data points as noise, produce arbitrary sized clusters and use any distance function \cite{Ester:1996:DAD:3001460.3001507}. It requires two parameters $\varepsilon$ which controls the neighbour search radius, and $MinPts$ which defines the minimum number of points needed to form a cluster. 

To form a cluster, point $q$ must be reachable from point $p$ \ie there must exist a path $p \leadsto q$ which fulfills both $\varepsilon$ and $MinPts$ parameters. To form this path, some points are labeled as core points satisfying parameters simultaneously, and some as border points which have at least one core point in its $\varepsilon$-range. If a data points is neither above, it is labeled as noise.


The pseudocode for DBSCAN is given in Algorithm \ref{alg-dbscan}, where \textproc{DiscoverNeighbours} is a recursive function that finds the neighbourhood space by forming the radius based on the distance function, and retrieves all reachable points restricted by the $\varepsilon$-range. The algorithm is able to form the amount of clusters itself and requires no pre-defined amount of clusters, which is a major benefit for plagiarism detection.

\begin{algorithm}[ht]
\caption{DBSCAN algorithm \cite{Ester:1996:DAD:3001460.3001507, Schubert:2017:DRR:3129336.3068335}}
\label{alg-dbscan}
\begin{algorithmic}

\Require Set of datapoints $X$
\Require Distance radius $\varepsilon$
\Require Minimum neighbour count $MinPts$
\Require Distance function $dist: X \times X \rightarrow \mathbb{R}$
\Procedure{DBSCAN}{$X, \varepsilon, MinPts, dist$}
   \State $k \gets 0$
   \For{$i = 1$ to $|X|$}
    \State $N \gets $ \Call{DiscoverNeighbours}{$X, dist, \bolditt{x}_i ,\varepsilon, MinPts$}
    \If{$\bolditt{x}_i$ is a core point}
        \State $k \gets k + 1$
        \State $\omega_k \gets N \cup \bolditt{x}_i$
    \Else
        \State $\bolditt{x}_i$ is noise
    \EndIf
   \EndFor
   \State \textbf{return} $\{\omega_1, \omega_2, \cdots, \omega_k\}$
\EndProcedure

\end{algorithmic}
\end{algorithm}


Using parameters $\varepsilon = 0.5, MinPts = 15$ and setting distance function as Euclidean distance, DBSCAN learns more denser clusters than K-means and is able to label some data as noise. This noise is visible in Figure \ref{fig-dbscan-example} as black data crosses. 

\begin{figure}[ht]
\centering
\setlength\figureheight{7cm}
\setlength\figurewidth{7cm}
\input{plots/example_dbscan.tikz}
\caption{Result of DBSCAN by setting parameters $\varepsilon = 0.5, MinPts = 15$. Black crosses refers to noise as these points are too far away from core points which forms the three clusters.} \label{fig-dbscan-example}
\end{figure}

\newpage

\noindent
If data points in Figure \ref{fig-dbscan-example} would represent documents in Euclidean space, we can interpret noise as documents which are too dissimilar to any other document and require no further attention. Other points are classified into three dense clusters represented by three different colors. 

\section{Literature Survey} \label{chap-liter-review}
Systematic literature review was conducted to gain information about the current state of source code plagiarism study in academia. The database that was utilized to query research papers is called \emph{Scopus\footnote{\url{https://www.scopus.com/}}}, which is a web service containing peer-reviewed scientific literature. The service itself holds links to papers which are published under for example \emph{ACM (Association for Computing Machinery)} and \emph{IEEE (Institute of Electrical and Electronics Engineers)}, both of these being major computer science institutions. 

Following subchapters describe how the review was conducted and what kind of results were found. 

\subsection{Review process}

Querying Scopus can be done in a similar way as querying databases in SQL-like languages. The used query inside Scopus was following
\begin{verbatim}
TITLE-ABS-KEY (("plagiarism" OR "authorship identification")  
                AND "source code") 
AND  (LIMIT-TO (SUBJAREA,"COMP"))
\end{verbatim}

\noindent
This query translates to finding papers which title, abstract or keywords contains the word \emph{plagiarism} or \emph{authorship identification} and at least the term \emph{source code}. The reason to choose these keywords was to find papers which study the problem of plagiarism finding from source code either in general terms, or by utilizing authorship identification techniques. Next, the query limits the area of study to computer science to focus on plagiarism studies which utilize techniques found from computer science.

After the initial search, the goal of the second step was to limit the amount of papers. This was done by excluding all papers that were any of the following types: a review of certain aspect of source code plagiarism e.g. student motives behind plagiarism, an improvement to some pre-existing algorithm\footnote{In this context meaning algorithmic speedup}, plugin to online learning management systems, application to competition where the used method wasn't explained, study that used either byte-level information or information gathered during running the program, hashing techniques\footnote{Using the size of compression as a metric}, system review which didn't address the method and theses. 

Beside these attributes, included papers needed to also test their proposed method in some way and the amount of documents in experiment phase needed to be larger than two. The reason for including this as a limiting factor, was to gather studies that used test sets to evaluate the performance of their model in terms of accuracy, as this allows to compare used techniques more critically. 


\subsection{Review results}

The total number of papers gathered by querying Scopus in the first part of the literature review was 187, and the date when the query was done was 7th of February 2018. The distribution of paper per year can be seen in the following plot.

\begin{figure}[h]
\centering
\includegraphics[width=\textwidth]{plots/Rplot.png}
\caption{Results of the first query to Scopus}
\end{figure}

\noindent
This set was filtered by terms described in the previous chapter, and the total number of papers inspected more carefully in this systematic literature review is 32. 

From the set of 32 studies, we look answers for following questions: \emph{how plagiarism can be detected from source code}, \emph{what are possible features that can be derived from source code} and \emph{how can one identify the author of a given source code}. We start first by grouping the papers by their themes to see what kind of different approaches there are to deal with the problem of plagiarism detection. After the initial classification, following aspects are identified from the studies: data, methods and test results. 


\subsubsection{Approaches}

The most high level division between papers could be done in a similar fashion as was used by the
query; dividing papers either to be about the detection of plagiarism or identifying the author of a given source code. However, during the literature review it was found that it's more clearer to make a division between similarity detection and identifying the authorship. This high-level division can be seen from the following table.

\begin{table}[ht]
    \caption{Papers divided into two high-level categories}
    \label{table-highcateq}
    \centering
    \begin{tabular}{ | c | c | }
        
        \hline
        {\bf Similarity detection} & {\bf Authorship identification} \\ \hline
    
        \cite{AFAPLI2015, LICD2010, AASCPD2012} & \cite{SCAANN2017, ABEC2014, CAPSCAP2014}   \\
        \cite{Heblikar2015NormalizationBS, USCR2014, AIR2015} &  \cite{SCANG2007, EJPFSAI2004, ACSBPD2012}\\
        \cite{OTIOLSS2015, BUAA2009, ramirez2015high} &  \cite{APASCAI2007, UCMHGAAI2007, ESHPFSCAC2008}\\
        \cite{Ohmann2015, TBCFPD2012, Fu2017WASTKAW} &  \cite{AIRTSCAA2009, TSUDIJSCAI2011, DNNSCAI2013} \\
        \cite{ASTMLPD2013, AAPSCDPTK2013, CPDPPD2013}    & \cite{SCAIUFL2013, SDNAIJSP2015, AISC2017} \\
        \cite{PACASCD2005, RCISCP2017} &  \\ \hline
        {\bf Number of papers} & {\bf Number of papers} \\ \hline
        17 & 15 \\ \hline
    \end{tabular}
\end{table}

\noindent
Even though papers divide quite evenly in table \ref{table-highcateq}, these high-level groups are still too large, and thus for the sake of clarity, we divide both into subgroups.

Similarity detection in itself can be further divided into at least two general categories based on the current tools \cite{RSCAD2016}: attribute and structure. Then naturally, as authorship identification uses features derived directly from the source code, we can use the same classification to authorship identification studies. However, based on the literature review, there are more finer categorizations that define the studies better based on the features they use, and thus we propose the following categories and their abbreviations: \emph{attribute counting (AC)}, \emph{segment matching (SM)}, \emph{n-gram (NG-STR)}, \emph{tree-based (AST-STR)} and lastly \emph{hybrid approaches (HYB-STR)}. If category has no studies under it, we leave the category out from the upcoming tables. These categories can be summarized briefly as following and are similar to categories identified from other similarity detection studies by Ali \etal \cite{OCPOCP2011}. 

\paragraph{Attribute counting}
Studies utilizing countable statistics, often referred as \emph{metrics}, that are gathered from source codes. This includes features like amount of words per line, number of lines per source code and number of keywords.

\paragraph{Segment matching}
Considers two source codes as two strings and finding maximum match between them i.e. longest common subsequence. These problems are also known as string matching problems, where one of the most famous algorithms is \emph{Greedy String Tiling} introduced early in \cite{SSGST1993}. We also categorize string similarity measures to this category like string edit distances.

\paragraph{$N$-gram}
Treating the source code as a string and splitting it via sliding window where the window size is the value of $n$ and the window traverses on either word or character level. This forms the vocabulary of the source code which is then transformed into occurrences of particular terms that are present, thus ultimately creating a vector representation of the source code. For example the statement \texttt{int a = 2} could be transformed into following word level 2-tuples using two as the value of $n$ (bigram). The first value of the following tuples is the $n$-gram extracted and the second value is the frequency: (\texttt{int a}, 1), (\texttt{a =}, 1), (\texttt{= 2}, 1). 

\paragraph{Tree-based methods}
Constructing a tree presentation from the source code, that captures the structure. The generation of a tree presentation usually requires some kind of parser because it's language specific feature. The inspection of a generated tree can be done via tree traversal methods for example using recursive functions. 

\paragraph{Hybrid methods}
Combine the usage of AST-structure with $n$-gram representation. For example it can be a method which traverses abstract syntax tree, prints it and generates $n$-gram representation from the output.
\\\\
The grouping of similarity detection papers can be seen from following table, where it's clear that most of the papers deal with similarity detection by utilizing structural features, indicated by the STR-ending, and many studies prefers to use $n$-gram representation of the source code.

\begin{table}[ht]
    \caption{Subgroups and sizes of similarity detection studies}
    \label{table-sdstudies}
    \centering
    \begin{tabular}{ | c | c | c | c | c |}
        
        \hline
        {\bf AC} & {\bf SM} & {\bf NG-STR} & {\bf AST-STR} & {\bf HYB-STR} \\ \hline
        \cite{PACASCD2005} & 
        \cite{LICD2010, ASTMLPD2013} & 
        \cite{AASCPD2012, USCR2014, AFAPLI2015} & 
        \cite{TBCFPD2012, AAPSCDPTK2013, AIR2015} & 
        \cite{BUAA2009, CPDPPD2013, RCISCP2017} \\
        & 
        & 
        \cite{Heblikar2015NormalizationBS, Ohmann2015, OTIOLSS2015} & 
        \cite{Fu2017WASTKAW} &
        \\
        & & \cite{ramirez2015high} &  & \\ \hline
        {\bf \#AC} & {\bf \#SM} & \multicolumn{3}{c |}{\bf \#STR} \\ \hline
        1 & 2 & \multicolumn{3}{c |}{14}
        \\ \hline
    \end{tabular}
\end{table}

When inspecting the division of authorship studies, we can see the division in table \ref{table-aistudies} is more evenly distributed contrast to similarity detection studies. More studies seems to utilize countable attributes from source codes and many also prefers to utilize $n$-grams, which is quite obvious when one considers that these methods are able to capture the writing style of an author from high-level features. For example authors can name the identifiers how they like, introduce comments and use various stylistic techniques when they write source code. 

\begin{table}[ht]
    \caption{Subgroups and sizes of authorship identification studies}
    \label{table-aistudies}
    \centering
    \begin{tabular}{ | c | c | c | c |}
        
        \hline
        {\bf AC} & {\bf NG-STR} & {\bf AST-STR} & {\bf HYB-STR} \\ \hline
        \cite{EJPFSAI2004, UCMHGAAI2007, APASCAI2007} & \cite{SCANG2007, ESHPFSCAC2008, AIRTSCAA2009} & \cite{SCAANN2017} & \cite{SDNAIJSP2015, AISC2017}\\ 
        \cite{ACSBPD2012, SCAIUFL2013, DNNSCAI2013} & \cite{TSUDIJSCAI2011, CAPSCAP2014, ABEC2014} & &\\ \hline
        {\bf \#AC} & \multicolumn{3}{c |}{\bf \#STR} \\ \hline
        6 & \multicolumn{3}{c |}{9}
        \\ \hline
    \end{tabular}
\end{table}

\newpage

Both of these results seems to show that utilizing structure is popular in both high-level classes, but quite dominant in similarity detection. However both groups of studies seems to show high popularity on $n$-gram methods, which is able to capture both individual style of the author and structural preferences. % source?


\subsection{Descriptive statistics} \label{subsec-liter-data}
The data used in testing phase of 32 gathered articles is presented next, where the focus is for the similarity detection on following attributes: number of total documents, is there any synthetic data used and the average number of lines of code (Avg. LOC). For the authorship identification we focus on features like documents per author and number of possible authors. The term \emph{document} here refers to the number of source code file samples per author. We summarize the findings from data utilizing the categorization that was made earlier.

\paragraph{Similarity Detection}\mbox{}\\
Attribute counting study by Moussiades and Vakali in \cite{PACASCD2005} uses two real data sets written in C\texttt{++}. They contain programming assignments and a forged set of programs. The first data set contains 24 programs having an average of 247 lines of code per submission, the second set is 51 programs having an average of 178 lines per source code. The forged data set is two modified versions from one program, trying to deliberately confuse state-of-the-art detectors.

Segment matching study by Brixtel et al. used three corpora on their evaluation and are written in Haskell, Python and C \cite{LICD2010}. Haskell corpus had 13 documents averaging 400 lines per each, Python 15 documents averaging 150 lines per each and C 19 documents averaging 250 lines per source code. Study by Zhang and Liu used 12 programs written in C that all reflected different plagiarism strategies \cite{ASTMLPD2013}. 

Studies utilizing $n$-grams are summarized into following table.

\begin{table}[ht]
\centering
\caption{Data used in similarity detection studies utilizing $n$-grams}
\label{table-ng-str-data}
\begin{tabular}{|c|c|c|c|c|c|c|c|}
          \hline
          \backslashbox{\bf Feature}{\bf Paper} & \cite{AASCPD2012} & \cite{USCR2014} & \cite{AFAPLI2015} & \cite{Heblikar2015NormalizationBS} & \cite{Ohmann2015} & \cite{OTIOLSS2015} & \cite{ramirez2015high} \\ \hline
\bf Documents  &  179  & 5302   & 191  & 1356  & 2935  & 5408  & 1277   \\ \hline
\bf Synthetic &  No  & No  &  No  & No  & No  &  No & No  \\ \hline
\bf Avg. LOC & NA  & NA  & NA  & NA & NA  & 63.7  & NA  \\ \hline
\end{tabular}
\end{table}

\noindent
It's visible from the table \ref{table-ng-str-data} that there are now a lot more documents used in experimentation and surprisingly synthetic data is not used at all. This is due to the usage of student submissions and competition data sets like \emph{Google Code Jam} submissions, which was for example utilized by Flores \etal in \cite{USCR2014}. 


\begin{table}[ht]
\centering
\caption{Data used in similarity detection studies utilizing abstract syntax tree}
\label{table-ast-str-data}
\begin{tabular}{|c|c|c|c|c|}
          \hline
          \backslashbox{\bf Feature}{\bf Paper} & \cite{TBCFPD2012} & \cite{AAPSCDPTK2013} & \cite{AIR2015} & \cite{Fu2017WASTKAW}\\ \hline
\bf Documents & 121 & 555 & NA & 22\,214  \\ \hline
\bf Synthetic & NA & No  & NA & Yes\\ \hline
\bf Avg. LOC & NA & 305.7 & NA & 20\\ \hline
\end{tabular}
\end{table}

\noindent
One can see from the table \ref{table-ast-str-data} that a study done by Fu \etal in \cite{Fu2017WASTKAW} has a large number of documents, and this due to two facts: they reported the size as pairs of documents and they used a generator to form a lot of forged documents from a small set of 10 original submissions. Ganguly and Jones in \cite{AIR2015} don't explicitly report the statistics of their data set, but refers to a competition test set called \emph{SOurce COde Re-use} (SOCO). This competition offers a set of C and Java files which contains known cases of cross-lingual plagiarism \cite{saez2014pan}. The train set size of SOCO is 338 files. 

Finally, hybrid study by Xiong \etal utilizes 40 assignments gathered from students \cite{BUAA2009}, Muddu \etal uses 5054 original files that they mutate to introduce copied code \cite{CPDPPD2013} and Ganguly \etal uses both train and test set of the SOCO competition, totaling around 12\,000 files \cite{RCISCP2017}. 


\paragraph{Authorship identification}\mbox{}\\
Usage of data in studies dealing with the problem of identifying the author and utilizing attribute counting are summarized to the following table, where we now turn the focus on the amount of candidate authors and documents per author reported in studies.

In table \ref{table-ai-ac-str-data}, one can see that there are two same data sets used in \cite{SCAIUFL2013, DNNSCAI2013}. This set was collected from \emph{SourceForge}\footnote{\url{https://sourceforge.net/}} projects and there are around 61 to 377 files per author. Rest of the attribute counting studies prefers to use \eg submissions gathered from students, as it's an easy way to gather tagged source code files.  

\begin{table}[ht]
\centering
\caption{Data used in authorship studies utilizing attribute counting}
\label{table-ai-ac-str-data}
\scalebox{0.9}{
    \begin{tabular}{|c|c|c|c|c|c|c|c|}
              \hline
              \backslashbox{\bf Feature}{\bf Paper} & \cite{EJPFSAI2004} & \cite{UCMHGAAI2007} & \cite{APASCAI2007}  & \cite{ACSBPD2012} & \cite{SCAIUFL2013} & \cite{DNNSCAI2013}\\ \hline
    \bf Authors  & 46 & 20 & 8  & 120 & 10 & 10 \\ \hline
    \bf Documents per author  & NA & 3 & 3  & NA & 61-377 & 61-377\\ \hline
    \bf Synthetic  & No & No & No & No & No & No\\ \hline
    \end{tabular}
}
\end{table}


Next, data sets from the second popular method $n$-grams used in authorship identification, are summarized into following table.

\begin{table}[ht]
\centering
\caption{Data used in authorship studies utilizing $n$-grams}
\label{table-ai-ng-str-data}
    \begin{tabular}{|c|c|c|c|c|c|c|}
              \hline
              \backslashbox{\bf Feature}{\bf Paper} & \cite{SCANG2007} & \cite{ESHPFSCAC2008} & \cite{AIRTSCAA2009} & \cite{TSUDIJSCAI2011} & \cite{CAPSCAP2014} & \cite{ABEC2014}\\ \hline
    \bf Authors  & 100 & 8 & 100 & 8 & 30 & 30\\ \hline
    \bf Documents per author  & 14 & 2 & 14-26 & 2 & NA & NA\\ \hline
    \bf Synthetic  & No & No & No & No & No & No\\ \hline
    \end{tabular}
\end{table}

\noindent
There exists three different data sets used by three different authors in table \ref{table-ai-ng-str-data}: Burrows \etal in \cite{SCANG2007, AIRTSCAA2009} used data set gathered from students C programming assignments, Frantzeskou \etal in \cite{ESHPFSCAC2008, TSUDIJSCAI2011} used open-source programs written in Java and Tennyson \etal in \cite{CAPSCAP2014, ABEC2014} used programs written in \cpp and Java which mixture of were open-source, sample and textbook programs.

The only study that mainly used abstract syntax tree in their authorship study is by Alsulami \etal in \cite{SCAANN2017}. They used \emph{Google Code Jam} to gather 700 Python source code files belonging to 70 programmers averaging around 10 submissions per author. 

Finally, the data used in two hybrid studies are summarized. Wisse and Veenman used repositories from version control website called \emph{GitHub} \cite{SDNAIJSP2015}. The largest author pool they had while testing was 30. Zhang \etal had the data set also gathered from websites like \emph{GitHub} in their study \cite{AISC2017}. Their largest data set with respect to the author size, was imbalanced set of 503 programs belonging to 53 authors. 

\paragraph{Summary}\mbox{}\\
When looking the data usage of plagiarism study as a whole, one can see that almost all studies use data that is non-synthetic \ie use real-life data, that can be gathered for example from students course submissions or from competitions like SOCO. In similarity detection studies the median of the amount of source codes used is 447 and very few studies reported the average lines of code, which is a bit problematic as it can be easier to find plagiarism from a small set of program lines than from larger programs. In authorship attribution the median of possible authors in studies is 30 and the documents per author ranges from two to as high as 377.

% tarkista mediaani

\subsection{Methods} \label{chap-liter-review-methods}
In this chapter we turn the focus to the actual methods used in various studies. We use the same 
classification as a baseline for studies that was made earlier. The math used in studies is generalized to match the style of this paper, which means that a document is represented as $d$, matrices are bold and upper-cased $\bolditt{A}$ and vectors are bold but lower-cased $\bolditt{a}$. 

\paragraph{Similarity detection}\mbox{}\\
As a recap, the problem of similarity detection can be described formally as following.

\newtheorem*{smd}{Similarity detection}

\begin{smd}
Given a set of source code documents $D = \{d_1,...,d_n\}$, define similarity function $sim: d_i, d_j \rightarrow [0, 1]$ such that $sim(d_i, d_j) = sim(d_j, d_i)$ and $sim(d_i, d_i) = 1$, with a optional threshold $\theta \in [0, 1]$ that defines the limit where two source codes are considered as too similar. With this definition, any pair of source code file $(d_i, d_j) \in D \times D$ can also be presented as a triplet $(d_i, d_j, s)$, where $i \neq j$ and $s$ is the similarity value between documents. 
\end{smd}

The attribute counting study by Moussiades and Vakali uses a graph clustering on top of pair-wise similarities calculated using the Jaccard coefficient \cite{PACASCD2005}. Authors use following form of Jaccard coeficcient in their study where $T$ is the indexed set of substitute keywords per source code 

\begin{equation}\label{jacc_eqn}
    sim(d_1, d_2) = \dfrac{|T(d_1) \cap T(d_2)|}{|T(d_1) \cup T(d_2)|}
\end{equation}
\noindent
% refer to plag. attack
The indexed set can be built considering language dependent keywords \eg \texttt{while, for, false and true} in \cpp, and marking their position with respect to the occurrences of same keywords previously. However, authors claim that to generalize the set more, substitution keywords should be used. This means that for example all occurrences of \texttt{for}- and \texttt{while} -loops should be counted together, which helps to protect against plagiarism attack. The graph clustering algorithm Moussiades and Vakali uses is called \emph{WMajorClust} which works by presenting all pairs of source codes as non-directed graph $G = (V, E)$ where the set of vertices $V$ represents the source codes while the set of edges $E$ are weighted by equation \ref{jacc_eqn}. We can also express the definition of $E$ by Moussiades and Vakali with following constraints

\begin{equation}\label{jacc_edges_eqn}
         E = \Big\{ \{ d_i, d_j, sim(d_i, d_j)\} \, | \, (d_i, d_j) \in D \times D \land sim(d_i, d_j) \geq \theta \Big\}
\end{equation}

\noindent
%chapter ref
In equation \ref{jacc_edges_eqn}, $\theta$ is a user-defined parameter and works as a minimum threshold value that separates non-plagiarized source codes from plagiarized ones \ie two source codes will not share an edge if their similarity is below $\theta$.

Segment matching study by Brixtel \etal presents their algorithm, which builds from three major steps \cite{LICD2010}: pre-filtering, segmentation and document distance calculation. Their pre-filtering is to normalize the source code in a way, that every keyword and parameter definitions is transformed into a single symbol. As a segmentation, authors split the source code by lines forming set of segments $S_k$ presenting the partitioned set of a single source code. Similarity calculation happens by first forming distance matrix $\bolditt{M}$ between two source codes $d_1, d_2$ and then comparing all pairs of segments $(s_i^1, s_j^2) \in S_1 \times S_2$ where $S_k = (s_1^k, ..., s_n^k)$, with \emph{Levenshtein edit distance}. Distance matrix $\bolditt{M}$ is then transformed into noise reduction matrix $\bolditt{H}$ by finding the maximal matching between segmentations. Finally, $\bolditt{H}$ is filtered into a matrix $\bolditt{P}$ by convolution and utilizing a threshold\footnote{Authors used $\theta = 0.7$}. With the matrix $\bolditt{P}$, distance between two pairs of documents can be calculated by Brixtel \etal as 

\begin{equation}
    sim(d_1, d_2) = 1 - \dfrac{1}{\min(|S_1|, |S_2|)}\sum_{i, j} 1 - \bolditt{P}_{(i, j)}
\end{equation}

\noindent
Zhang and Liu utilize AST-tree and their core method is mainly constructed from two methods \cite{ASTMLPD2013}: forming the AST-representation and similarity calculation. Their AST-representation is done by traversing the parsed AST-tree and turning it into textual format by printing the nodes, and similarity calculation is computed using \emph{Smith Waterman algorithm} that finds the optimal matching between two strings $S_1, S_2$. Zhang et Liu gives the formula for similarity calculation between two source codes as

\begin{equation}
    sim(d_1, d_2) = \dfrac{2 \cdot \text{ SLength}(d_1, d_2)}{|S_1| + |S_2|}
\end{equation}
\noindent
Where SLength is the length of maximal matching string obtained via  \emph{Smith Waterman algorithm}, and $|S_k|$ represents the character length of one segment. 


$N$-gram studies take a different approach. Cosma and Joy uses \emph{Latent Semantic Analysis} to find suspicious documents \cite{AASCPD2012}. They first preprocess the documents by removing \eg short terms and comments. Then all documents are first transformed into a term-by-file matrix $\bolditt{A}$, where each document is represented as a occurrences of each unique term, which is same as forming the unigrams of a document. Values of $\bolditt{A}$ are weighted, and then $\bolditt{A}$ is decomposed via \emph{singular value decomposition} into $\bolditt{A} = \bolditt{U}\mathbf{\Sigma}\bolditt{V}^\intercal$ where $\bolditt{U}$ represents terms by dimension, $\mathbf{\Sigma}$ singular values and $\bolditt{V}$ files by dimensions. The dimensionality reduction is performed for all these matrices by considering only the first 30 columns. Finally, the similarity between a query vector $\bolditt{q}$ representing term frequency of document $d_i$, and document $d_j$ represented as a column $\bolditt{a}_j$ of matrix $\bolditt{A}$ is calculated by \emph{cosine similarity} \cite{AASCPD2012}

\begin{equation}\label{cosine_sim_eqn}
    sim(\bolditt{q}, d_j) = \cos \Theta_j = \dfrac{\bolditt{a}_j^\intercal \bolditt{q}}{\norm{\bolditt{a}_j}_2 \norm{\bolditt{q}}_2} = \dfrac{\bolditt{a}_j \boldsymbol{\cdot} \bolditt{q}}{\sqrt{\sum \limits_{i} \bolditt{a}_{(j, i)}^2} \sqrt{\sum \limits_{i} \bolditt{q}_i^2}}
\end{equation}

\noindent
Acampora and Cosma \cite{AFAPLI2015} continues on same style as Cosma and Joy \cite{AASCPD2012}, first preprocessing the documents by lowercasing and removing comments, syntactical tokens and short terms. Then using singular value decomposition with weighting to form three matrices from the corpus of source codes. For the reduced matrix $\bolditt{V}$ however, they perform a \emph{Fuzzy C-Means} clustering algorithm, which is tuned with \emph{ANFIS} learning algorithm to optimize the hyperparameters of Fuzzy C-means \cite{AFAPLI2015}. The process returns a membership degree $\mu_{i, k}$ per document, indicating how close $i$th document is to $k$th cluster. 
\noindent
Flores \etal \cite{USCR2014} uses similar preprocessing approach to Cosma and Joy. They first process the documents by lower-casing them and removing repeated character, tabs with spaces. Then transform the documents into $3$-grams and weighting them by using a \emph{term frequency}. Finally, similarity is calculated using cosine similarity where $t$ is one of the 3-grams and $tf$ is the term frequency function \cite{USCR2014}. Formally this can be calculated in a same way as in equation \ref{cosine_sim_eqn} between two documents as

\begin{equation}
    sim(d_i, d_j) = \dfrac{\sum\limits_{t \in d_i \cap d_j} tf(t, d_i) tf(t, d_j) }
                          {\sqrt{\sum\limits_{t \in d_i} tf(t, d_i)^2 \sum\limits_{t \in d_j} tf(t, d_j)^2}}
\end{equation}

\noindent
Heblikar \etal \cite{Heblikar2015NormalizationBS} preprocesses also their documents by lower-casing, pruning repeated whitespace and removing single symbols. They then normalize the documents by considering most frequent terms, renaming similar terms under same symbols and ultimately filtering them completely out from the source codes. For detection phase, they use same approach as Flores \etal did in \cite{USCR2014} but use both 1-grams and 2-grams with \emph{term frequency - inverse document frequency} (tf-idf) weighting. Interestingly, also Ramírez-de-la-Cruz \etal in \cite{OTIOLSS2015} and Ramírez-de-la-Cruz \etal in \cite{ramirez2015high} decides to use cosine similarity and Jaccard coefficient. The only major difference being, that Ramírez-de-la-Cruz \etal uses additional structural and stylistic features, forming total combination of eight various similarity measurements \cite{OTIOLSS2015}. Where as Ramírez-de-la-Cruz \etal in \cite{ramirez2015high} uses cosine similarity with character 3-grams to calculate five different similarities: lexical, stylistic, comments, text\footnote{Referring here as any string passed in as an argument of a function} and structural. Lastly, Ohmann and Rahal proposes density-based clustering to form clusters of similar documents \cite{Ohmann2015}. Their similarity approach follows closely to other studies presented above: filtering and normalization as preprocessing, data format as word $n$-grams and similarity values gained by using cosine similarity. 

Tree-based studies mostly relies on calculating similarity between two tree structures $T_i, T_j$ obtained from the original documents $d_i, d_j$ by parsing them. For example Ng \etal first generates parse tree $T$ from the source code, then decomposes parse tree into subtrees $T' \subseteq T$ with respect to the functionality \eg imports are categorized together \cite{TBCFPD2012}. The similarity score is thus obtained by comparing trees with \emph{depth-first search} and summing the scores for all subtrees to form a similarity score between two documents. The similarity function between two documents can be expressed with the following definition where $simST$ is the similarity score between two subtrees obtained by comparing nodes and tokens 

\begin{equation}
    sim(d_i, d_j) = sim(T_i, T_j) = \dfrac{\sum\limits_{i, j}simST(T'_i, T'_j)}{10 \cdot |T'|} \cdot 100
\end{equation}

\noindent
Son \etal computes similarity value between two parse trees with a modified parse tree kernel \cite{AAPSCDPTK2013}. They define the kernel function $k$ via recursive function $C$ where $n$ is the node of a subtree $T'$. Function $C$ finds a maximal similarity between $n_i, n_j$ thus authors calls it as \emph{maximum node value}  

\begin{equation}
    k(T_i, T_j) = \sum\limits_{n_i \in T'_i} \sum\limits_{n_j \in T'_j} C(n_i, n_j)
\end{equation}

\noindent
The actual similarity between documents can be calculated via normalization \cite{AAPSCDPTK2013}

\begin{equation}
    sim(d_i, d_j) = \dfrac{k(T_i, T_j)}{\sqrt{k(T_i, T_i) \cdot k(T_j, T_j)}}
\end{equation}

% C(n_i, n_j) &= \lambda \prod \limits_{k}^{nc(n_i)} \left( 1 + \max\limits_{ch \in ch_{n_j}} C(ch_k(n_i), ch)\right)

\noindent
Asd

%   \cite{AIR2015} & \cite{Fu2017WASTKAW}


\newpage
\paragraph{Authorship identification}\mbox{}\\
a



%\begin{algorithm}[ht]
%\caption{See how easy it is to provide algorithms}
%\label{myFirstAlgorithm}
%\begin{algorithmic}
%\REQUIRE $a$
%\STATE $b = 0$
%\STATE $x \leftarrow 1:10$
%\FORALL{x}
%    \STATE $b = b+a$
%\ENDFOR
%\RETURN $b$
%\end{algorithmic}
%\end{algorithm}

%\subsubsection{Accuracies}
%\input{liter_review/accuracies.tex}


\section{Research Design}

We base our model in this thesis to a claim that building a two phase model can reduce the amount of false-positives found in source code plagiarism detection. False-positives are problematic, as it would mean an innocent author would be considered as a possible plagiarist, thus having too sensitive models introduces a lot of extra work. These two phases of our model are similarity detection and authorship identification, where both of them are able to define a set of authors; similarity detection reveals suspicious authors based on the file similarity and authorship identification predicts candidate authors for a document $d$. 

%pre-process documents, tokenize documents, exclude templates, calculate similarities and finding suspects using these similarity scores.

Both of our models are inspired by other studies presented in the literature review and combine the high-level approach used in many tools \cite{RSCAD2016}: preprocess, normalize, calculate metrics and predict. The generalization of the proposed model is given below.

\begin{algorithm}[ht]
\caption{Detecting plagiarism between a set of source code files}
\label{alg-toplvl}
\begin{algorithmic}

\Require Set of authors $A$
\Require Set of source code files $D$ written by various authors $\forall a \in A$
\Procedure{PLGdetect}{$A, D$}
   \State $D'\gets normalize(D)$
   \State $D'_{train}, D'_{test} \gets split(D')$
   \State $\mathcal{M} \gets train(A, D'_{train})$
   \State $A_{auth} \gets \mathcal{M}(D'_{test})$
   \State $A_{susp} \gets detectSim(A, D'_{test})$
   \State \textbf{return} $A_{auth} \cap A_{susp}$
\EndProcedure

\end{algorithmic}
\end{algorithm}

\noindent
In algorithm \ref{alg-toplvl}, the source codes are first normalized to reduce the noise caused by \eg comments and repeated whitespace. We train a supervised learning model $\mathcal{M}$ that is able to classify the author of a document. We retrieve the top $n$ predictions called a set of candidate authors $A_{auth}$ for a document $d$. Similarity detection is calculated for the same set of documents to find suspicious authors $A_{susp}$. The word \emph{suspicious} refers here to too similar documents that are written by different authors. Finally, intersection between candidate authors and suspicious authors is taken, to provide a set of possible plagiarists. Both parts of our model are validated against real-life tools; plagiarism detection is evaluated against JPlag \cite{prechelt2002finding} which has been used as a baseline model in a SOCO competition \cite{saez2014pan}, and authorship identification against SCAP method.

\subsection{Assumptions}

We mainly focus on academia and especially to programming courses that are offered by 
universities. Following four assumptions are defined to simplify the problem of plagiarism
detection by allowing us to concern only plagiarism that happens in a closed environment and within a closed set of documents. 

\paragraph{In-class plagiarism} Plagiarism has occured only inside a 
specific course implementation. Let $\mathcal{P}(A)$ be a powerset of students within offered courses in university. We are only interested about a set of students referred as authors $A$ attended in a specific course $c$ \ie a subset $A_c \subseteq \mathcal{P}(A), A_c \neq \emptyset$. The corpus $D_c$ is built by gathering every submission done by students $\forall a \in A_c$ and a set of documents belonging to individual student is defined as $D_a = \{d \mid d \in D_c, a = auth(d)\}$. 


\paragraph{Exercise focus} 
Let $E_c = \{e_1, e_2, ..., e_n\}$ be a set of exercises for a course $c$, then submissions for a single exercise is represented by a subset $D_{c,e} \subseteq D_c$. With this assumption, we focus the plagiarism detection to submissions done to a single exercise at a time \ie plagiarism can happen only between submissions to a single exercise.

\paragraph{Single author} 
Every source code $d \in D_c$ is assumed to have a single author $a = auth(d), a \in A_c$. This allows us to assume that every source code submissions is done as a individual work, and all results that suggests otherwise implies about the case of excessive collaboration. 

\paragraph{Plagiarism direction} 
Let a file $d_i$ be plagiarized from $d_j$ \ie $d_i \xrightarrow{plag} d_j$, we treat this as same as the opposite direction $d_i \xleftarrow{plag} d_j$, making the direction of plagiarism unimportant. Thus the goal of our model is to give a set suspicious authors given the file and the claimed author.

\paragraph{Expert interference}
We believe that no system can be accurate enough to autonomously accuse students about plagiarism. However, this is doable when some form of human judgment is added to the model. In principal this means that the model can make predictions about cases of plagiarisms which we call \emph{suspects}, but the human expert must make the \emph{allegation} of plagiarism based on the results and after questioning the students.  

\subsection{Data set}

Our model is aimed to the traditional MOOC setting which is for example used by  undergraduate-level programming courses \emph{Introduction to Programming} (OHPE) and \emph{Advanced course in Programming} (OHJA) in University of Helsinki. We use all three real-life data sets; students submissions done to both of latter courses during the implementation in fall 2016 and a train data from SOCO task from 2014. All source code files are written in Java programming language. 

To implement our model, we first use SOCO to train and evaluate our similarity detection model, then train and test authorship identification with OHPE and OHJA. Our proposed model is built based on these results and plagiarism is detected individually for both courses. The reason to use train set of SOCO for similarity detection, is simply that it's the only data set that contains fully labeled cases of plagiarism, but unfortunately contains only one file per author. OHPE and OHJA on the other hand, contains multiple files per author making author identification possible, but only a few \emph{known} cases of plagiarism. Therefore we make use of both sets and consider our model to be successful if it finds at least every known case of plagiarism from OHPE and OHJA.

% TODO: add references from laptop
\paragraph{Course overview}\mbox{}\\
As courses, OHPE and OHJA shares the same structure; students first register to automatic scoring system called \emph{Test My Code} (TMC) which also distributes the exercises as an plugin to \emph{NetBeans IDE}, then independently work during seven weeks by completing programming exercises within deadlines. Students earn one point per exercise depending if all tests were successfully passed and complete an exam at end of the course, which is a programming exam that ultimately decides if a student has at least learned the minimum level required. There are no mandatory lectures, thus students are able to earn credits by working individually without any physical attendance. Also the exam in fall 2016 was a home exam, meaning that students were able to do it individually wherever they wanted to. 

\paragraph{SOCO overview}\mbox{}\\
Source code reuse (SOCO) data is from a 2014 competition \emph{PAN@FIRE}, where two sets were given to detect monolingual source code re-use \cite{saez2014pan}. SOCO2014 offered a train and a test set for competitors, which contained files written in \cpp\, and Java by various authors. The train set contains the source code files and annotations which are made by three experts flagging which pairs are considered as plagiarism. Competitors were then asked to retrieve which pairs are plagiarized. For example pair $(d_i, d_j)$ refers that there exists plagiarism between these two files, and because the direction was completely ignored, it was sufficient to retrieve just the predicted pairs.

SOCO contains mainly submissions to a single exercise and couple of documents, that are transformed from C to Java. As only the plagiarized file pairs are annotated and SOCO has been used successfully used in other studies \cite{AIR2015, RCISCP2017, OTIOLSS2015, USCR2014}, we make a simplifying assumption that the train set of SOCO contains one file per one unique author and that all submissions are submitted for the same task. This won't affect negatively the performance of our proposed model, as similarity detection is not affected at all if there exists multiple tasks within a corpus. 



\paragraph{Corpus statistics}\mbox{}\\
We are going to focus to Java language, therefore we only use the Java-specific part of SOCO training set, but fully utilize OHPE and OHJA data sets due to a fact that they only contain Java files. Number of non-transformative steps has been made beforehand to form the upcoming corpora: 1) leave SOCO as it is, 2) add exams to both OHPE and OHJA, and 3) concatenate submission containing multiple files into one file. This allows us to handle only one file per submission and we also get the benefit of having exam submissions, which is something where plagiarism is absolutely not allowed. 

%As OHPE and OHJA are both real-life courses, we also include the exam which in OHPE is made out of four tasks and in OHJA out of three tasks. 

Descriptive statistics for all three collections without any textual preprocessing is given in table \ref{tbl-corporastats}, where ten different metrics are reported: number of total authors, exercises and documents; does the corpus contains synthetic data; means for documents per author, character count, lines of code (LOC) and expressions\footnote{We assume countable expressions to be the ones ending in a semicolon}; and lastly minimum and maximum line counts. We can see from the table \ref{tbl-corporastats}, that SOCO has the smallest amount of authors but the tasks are more complex indicated by the largest LOC, amount of expressions and character count. When comparing OHPE to OHJA, OHPE has relatively smaller submissions than OHJA, which is mostly due to OHPE having easier tasks due to being the introductory course where students are not expected to know anything about programming beforehand. OHPE also has the most largest document-to-author ratio (106) compared to SOCO (1) and OHJA (56), making it the most richest data set when it comes to having a large amount of submissions per author.  Comparing to other corpora presented in chapter \ref{subsec-liter-data}, our OHPE corpus is one of the largest with OHJA. They both have over four times as many authors than any of the corpora used in other studies.

\begin{table}[ht]
\centering
\caption{Descriptive statistics for the unprocessed corpora. Bold values represents maximum per metric.}
\label{tbl-corporastats}
\begin{tabular}{|c|c|c|c|} \hline
\backslashbox{\bf Feature}{\bf Corpus}   & SOCO & OHPE & OHJA\\  \hline
\textbf{Authors}         & 259 & \textbf{316} & 270   \\  \hline
\textbf{Exercises}       & 1 & \textbf{151} & 92     \\  \hline
\textbf{Documents}       & 259 & \textbf{33\,363} & 15\,196    \\  \hline
\textbf{Documents per author $\mu$} & 1 & \textbf{106} & 56\\ \hline
\textbf{Synthetic}       & Partly & No & No \\  \hline
\textbf{LOC $\min$}         & \textbf{12} & 1 & 1      \\  \hline
\textbf{LOC $\mu$}        & \textbf{149} & 44 & 109     \\  \hline
\textbf{LOC $\max$}         & \textbf{1696} & 679 & 637   \\  \hline
\textbf{Expression $\mu$}       & \textbf{63} & 17 & 38 \\ \hline
\textbf{Character $\mu$} & \textbf{3898} & 1139 & 2794   \\  \hline
\end{tabular}
\end{table}

\newpage

A problem however arises when average line count with respect to the exercises is visualized for both OHPE and OHJA. Figure \ref{fig-hists} visualizes this by histograms, where both bin sizes are set to 50. 


\begin{figure}[!h]
\centering
\captionsetup[subfigure]{justification=centering}

\begin{subfigure}{\textwidth}
    \setlength\figureheight{5cm}
    \setlength\figurewidth{\textwidth}
    \input{plots/ohpe_avgloc.tikz}
    \label{fig-ohpeavgloc}
\end{subfigure}

\begin{subfigure}{\textwidth}
  \setlength\figureheight{5cm}
    \setlength\figurewidth{\textwidth}
    \input{plots/ohja_avgloc.tikz}
    \label{fig-ohjaavgloc}
\end{subfigure}

\caption[Two histograms for corpora]{Histograms showing average line of count per exercise for OHPE (top) and OHJA (below)}
\label{fig-hists}
\end{figure}

\noindent
From figure \ref{fig-hists} we see that majority of the submissions for OHPE has under 50 lines of code. This creates an issue for plagiarism detection, as there exists tasks where the submission can only contain a few dozen lines meaning, that the similarities between solutions will be naturally high. 

Consider for example three programs presented in appendix \ref{appendix:programs}. Because the solution to the given task is very limited, the submissions will inevitably have similar solutions, questioning the very fact that can there even exist plagiarism in short tasks. To overcome this problem we filter out all tasks where average line count falls to first quartile \ie under the 25th percentile, which is 21.0 for OHPE and 54.5 for OHJA. We call these filtered out tasks as \emph{trivial tasks} \ie exercises which solution space is very restricted. The intuition behind our filtering is that one cannot simply find plagiarism from too simple exercises which includes given tasks like: printing \say{Hello World} (1st task of OHPE), calculating how many seconds are in a year (6th task of OHPE) or doing simple string concatenation (2nd task of OHJA). 


\newpage

\subsection{Document representation}


\begin{figure}[!h]
\centering
\setlength\figureheight{5cm}
\setlength\figurewidth{8cm}
\input{plots/fig.tikz}

\caption{TEST} \label{fig:M1}
\end{figure}

\subsection{Normalization}

\subsection{Similarity detection}

%use t-SNE for visualization!! density is lost https://stats.stackexchange.com/questions/263539/k-means-clustering-on-the-output-of-t-sne?utm_medium=organic&utm_source=google_rich_qa&utm_campaign=google_rich_qa

\subsection{Authorship identification}



\subsection{Evaluation metrics}

\section{Results}

Following chapters describe the results we gathered during the evaluation of our models. All results are generated using Python version 3.6.0\footnote{\url{https://www.python.org/} Accessed 14th May 2018} and scikit-learn version 0.19.1\footnote{\url{http://scikit-learn.org/stable/} Accessed 14th May 2018}. 

As explained in the Chapter \ref{chap-method-evaluation}, we first evaluate both models individually and lastly combine the results to create a final prediction which is evaluated by a human expert. Our similarity detection is trained with SOCO data set and authorship identification with OHPE and OHJA without using the exams. A summary of these exam tasks is given below.

\begin{table}[ht]
\centering
\caption{Submission count and average line count for exam tasks. A refers to OHPE and B for OHJA.}
\label{tbl-exam-data}
\begin{tabular}{|c||c|c|c|c|c|c|c|}
\hline
\bf Task        & 1.A & 2.A & 3.A & 4.A & 1.B & 2.B & 3.B \\ \hline
\bf Submissions & 244 & 242 & 227 & 240 & 200 & 198 & 197 \\ \hline
\bf Avg. LOC    & 37  & 39  & 47  & 110 & 160 & 86  & 150 \\ \hline
\end{tabular}
\end{table}

\noindent
It's clear from the Table \ref{tbl-exam-data} that OHJA's tasks are more longer than OHPE's. Some of the tasks of OHPE's exam have a very low average line count that creates a challenge for the detection.

\subsection{Document similarity} \label{chap-sd-result}
% macro-averaged results

We start evaluating our similarity detection by tuning the hyperparameters $n$ for $n$-gram length and $\varepsilon$ for the epsilon-range \ie minimum distance to other documents. Results are gained by turning all documents into binary vector based on the SOCO labels \ie vector $\bolditt{y}$ where $y_i = 1$ and $y_j = 1$ if $i$th and $j$th documents are reported as plagiarized pairs. Our predictions are then compared to this golden standard.

Table \ref{tbl-sd-socot-fone} shows averaged $F_1$-score, weighted by label counts, for the SOCO-T data. One can see from it that the $F_1$-score is highest when $n \in [4, 7]$ and $\varepsilon \in [0.4, 0.6]$. However, allowing 40-50\% dissimilarity between documents means that there is a high chance for false-positives, especially when submissions are relatively short and the task is well-defined like in OHPE and OHJA, meaning that the solution space for a given task can be limited. Therefore to avoid over-fitting similarity detection to SOCO's training data we use also two test sets of SOCO.   

\newpage

\begin{table}[ht]
\centering
\caption{Average $F_1$-score for $n$-gram length and $\varepsilon$-range for SOCO-T containing 115 cases of plagiarism. The smaller the $\varepsilon$-range is, the more similar documents have to be. $F_1$-scores close or over 0.8 are bolded.}
\label{tbl-sd-socot-fone}
\scalebox{0.75}{
    \def\arraystretch{1.5}
    \begin{tabular}{|c||c|c|c|c|c|c|c|c|c|c|} \hline
    \backslashbox{\bf Epsilon}{\bf $N$-gram} & 1 & 2 & 3 & 4 & 5 & 6 & 7 & 8 & 9 & 10 \\ \hline
    0.1 & 0.31  & 0.69  & 0.63  & 0.60  & 0.59  & 0.56 & 0.55  & 0.55   & 0.52  & 0.52   \\ \hline
    
    0.2 & 0.28  & 0.59  & 0.73  & 0.66  & 0.63  &  0.62  & 0.60  & 0.59  & 0.56  &  0.55 \\\hline
    
    0.3 &  0.27  & 0.43  & \bf 0.78 & 0.73  & 0.70  & 0.67  & 0.64 & 0.63 & 0.59  & 0.58   \\ \hline
    
    0.4 & 0.27  & 0.31  & 0.72  & \bf 0.81  & \bf 0.78  & 0.72   &  0.71  & 0.69  & 0.65  & 0.64  \\ \hline
    
    0.5 & 0.27  & 0.29  & 0.57  & \bf 0.80  & \bf 0.81  & \bf 0.80  & \bf 0.81  & \bf 0.78 &  0.77   & 0.74   \\ \hline
    
    0.6 & 0.27  & 0.27  & 0.39  & 0.71  & \bf 0.83  & \bf 0.89  & \bf 0.90  &  \bf 0.86  & \bf 0.85 & \bf 0.85   \\ \hline
    
    \end{tabular}
}
\end{table}




\begin{table}[ht]
\centering
\caption{Precision with respect to plagiarized class, ranging various $n$-gram lengths and $\varepsilon$-ranges for SOCO-T. Values close or over 0.9 are bolded.}
\label{tbl-sd-soco-prec}
\scalebox{0.75}{
    \def\arraystretch{1.5}
    \begin{tabular}{|c||c|c|c|c|c|c|c|c|c|c|} \hline
    \backslashbox{\bf Epsilon}{\bf $n$-gram} 
        & 1 & 2 & 3 & 4 & 5 & 6 & 7 & 8 & 9 & 10 \\ \hline
    0.1 & 0.45  & 0.77  & \bf 1.00  & \bf 1.00  & \bf 1.00  & \bf 1.00 & \bf 1.00  & \bf 1.00   & \bf 1.00  & \bf 1.00   \\ \hline
    
    0.2 & 0.45  & 0.53  & \bf 0.98  & \bf 1.00  & \bf 1.00  &  \bf 1.00  & \bf 1.00  & \bf 1.00  & \bf 1.00  &  \bf 1.00 \\\hline
    
    0.3 &  0.44  & 0.48  &  0.83 & \bf 1.00  & \bf 1.00  & \bf 1.00  & \bf 1.00 & \bf 1.00 & \bf 1.00  & \bf 1.00  \\ \hline
    
    0.4 & 0.44  & 0.45  & 0.63  & \bf 0.87  & \bf 0.97  & \bf 0.98  &  \bf 0.98  & \bf 1.00  & \bf 1.00 & \bf 1.00  \\ \hline
    
    0.5 & 0.44  & 0.45  & 0.54  & 0.75  & \bf 0.90  &  \bf 0.92  &  \bf 0.97  & \bf 0.98 &  \bf 1.00   & \bf 1.00   \\ \hline
    
    0.6 & 0.44  & 0.44  & 0.47  & 0.62  & 0.77  & \bf 0.87  & \bf 0.94  & \bf 0.93  & \bf 0.95 & \bf 0.96   \\ \hline
    
    \end{tabular}
}
\end{table}

\noindent
We see from the Table \ref{tbl-sd-soco-prec}, that as we grow the number of $n$-grams, the precision starts converging to 1.00. Having a high precision means that the set of retrieved documents contains high number of true positives, as we have effectively minimized the amount of false positives, and no document is falsely accused of plagiarism. This happens because longer $n$-grams grow the size of vocabulary $\mathbb{V}$, thus making already dissimilar documents even more dissimilar and allowing the threshold to grow. The most smallest $n$-gram having a near perfect precision over plagiarized class is when $n=3$ and $\varepsilon \in [0.1, 0.2]$. This kind of high similarity value ranging between 80-99\% is also used in other studies \cite{AASCPD2012, OTIOLSS2015, Heblikar2015NormalizationBS, BUAA2009}. 

One sees from the following table that the $F_1$-score starts to deteriorate in all cases, when no plagiarism occurs in a set of documents. One must either have a high similarity threshold or increase the $n$-gram length to get a high $F_1$-score, because having a low threshold quickly introduces false positives. The model thus becomes too sensitive and retrieves documents where similarity has occurred naturally, adding work for the human expert who must go through the detected pairs and label them again. 


\begin{table}[ht]
\centering
\caption{$F_1$-score for SOCO-C1, which contains no cases of plagiarism. False-positives are  introduced as the threshold gets lower.}
\label{tbl-sd-sococ1-fone}
\scalebox{0.75}{
    \def\arraystretch{1.5}
    \begin{tabular}{|c||c|c|c|c|c|c|c|c|c|c|}
    \hline
    \backslashbox{\bf Epsilon}{\bf $n$-gram}    & 1    & 2    & 3    & 4    & 5    & 6    & 7    & 8    & 9    & 10   \\ \hline
    0.1 & 0.24 & \bf 0.94 & \bf 0.99 & \bf 0.99 & \bf 0.99 & \bf 0.99 & \bf 0.99 & \bf 0.99 & \bf 0.99 & \bf 0.99 \\ \hline
    0.2 & 0.11 & 0.56 & \bf 0.98 & \bf 0.99 & \bf 0.99 & \bf 0.99 & \bf 0.99 & \bf 0.99 & \bf 0.99 & \bf 0.99 \\ \hline
    0.3 & 0.06 & 0.38 & \bf 0.95 & \bf 0.99 & \bf 0.98 & \bf 0.99 & \bf 0.99 & \bf 0.99 & \bf 0.99 & \bf 0.99 \\ \hline
    0.4 & 0.03 & 0.20  & \bf 0.87 & \bf 0.98 & \bf 0.98 & \bf 0.98 & \bf 0.98 & \bf 0.98 & \bf 0.98 & \bf 0.98 \\ \hline
    0.5 & 0.03 & 0.16 & 0.59 & \bf 0.95 & \bf 0.98 & \bf 0.98 & \bf 0.98 & \bf 0.98 & \bf 0.98 & \bf 0.98 \\ \hline
    0.6 & 0.02 & 0.08 & 0.29 & \bf 0.88 & \bf 0.96 & \bf 0.98 & \bf 0.98 & \bf 0.98 & \bf 0.98 & \bf 0.98 \\ \hline
    \end{tabular}
}
\end{table}


\begin{table}[ht]
\centering
\caption{$F_1$-score for SOCO-C2, which contains 28 cases of plagiarism.}
\label{tbl-sd-sococ2-fone}
\scalebox{0.75}{
   \def\arraystretch{1.5}
   \begin{tabular}{|c||c|c|c|c|c|c|c|c|c|c|}
    \hline
     \backslashbox{\bf Epsilon}{\bf $n$-gram}     & 1    & 2    & 3    & 4    & 5    & 6    & 7    & 8    & 9    & 10   \\ \hline
    0.1 & 0.34 & \bf 0.92 & \bf 1.00 & \bf 1.00 & \bf 1.00 & \bf 1.00 & \bf 1.00 & \bf 1.00 & \bf 1.00 & \bf 1.00 \\ \hline
    0.2 & 0.27 & 0.57 & \bf 1.00 & \bf 1.00 & \bf 1.00 & \bf 1.00 & \bf 1.00 & \bf 1.00 & \bf 1.00 & \bf 1.00 \\ \hline
    0.3 & 0.20 & 0.38 & \bf 0.92 & \bf 1.00 & \bf 1.00 & \bf 1.00 & \bf 1.00 & \bf 1.00 & \bf 1.00 & \bf 1.00 \\ \hline
    0.4 & 0.15 & 0.31 & 0.75 & \bf 0.97 & \bf 0.97 & \bf 0.99 & \bf 0.99 & \bf 0.99 & \bf 0.99 & \bf 1.00 \\ \hline
    0.5 & 0.15 & 0.27 & 0.47 & \bf 0.91 & \bf 0.97 & \bf 0.97 & \bf 0.97 & \bf 0.97 & \bf 0.99 & \bf 0.99 \\ \hline
    0.6 & 0.15 & 0.22 & 0.33 & 0.78 & \bf 0.92 & \bf 0.97 & \bf 0.97 & \bf 0.97 & \bf 0.97 & \bf 0.97 \\ \hline
    \end{tabular}
}
\end{table}

\noindent
As in Table \ref{tbl-sd-sococ1-fone}, Table \ref{tbl-sd-sococ2-fone} shows that having $n=3$ with similarity threshold being around 80\%, yields one of the highest $F_1$-score with the lowest $n$ used. 

Taking the best scoring models over all scores for each $n$-gram and excluding $n \geq 8$ as they aren't improving the performance compared to $n=7$, we end up with five model candidates A ($n=3, \varepsilon=0.2$), B ($n=4, \varepsilon=0.4$), C ($n=5, \varepsilon=0.5$), D ($n=6, \varepsilon=0.6$), E ($n=7, \varepsilon=0.6$). As we compare these five models by their relative size of the largest cluster across all formed clusters of OHPE's exam tasks, we see in Figure \ref{fig-sd-clust-size} that in majority of the cases model A's largest cluster has the lowest relative size. In Figure \ref{fig-sd-clust-size:a} models B and D have relative size close to 1.0, meaning that the largest cluster contains almost every single retrieved document. The most probable cause for this is that all submissions share natural similarity, caused by the restricted task description or that the solution space for a given task might be very limited. Thus this kind of super cluster can contain a lot of false positives in form of similar documents which are not necessarily plagiarized, but rather correct similar solutions for the task.

\newpage

\begin{figure}[ht] 
  \begin{subfigure}[b]{0.5\linewidth}
    \centering
    \setlength\figureheight{3.5cm}
    \setlength\figurewidth{\textwidth}
    \input{plots/result/SD/ohpe_sd_cluster_size1.tikz}
    \caption{OHPE 1st exam task.} 
    \label{fig-sd-clust-size:a} 
    \vspace{4ex}
  \end{subfigure}%% 
  \begin{subfigure}[b]{0.5\linewidth}
    \centering
    \setlength\figureheight{3.5cm}
    \setlength\figurewidth{\textwidth}
    \input{plots/result/SD/ohpe_sd_cluster_size2.tikz}
    \caption{OHPE 2nd exam task.} 
    \label{fig-sd-clust-size:b} 
    \vspace{4ex}
  \end{subfigure}
   \begin{subfigure}[b]{0.5\linewidth}
    \centering
    \setlength\figureheight{3.5cm}
    \setlength\figurewidth{\textwidth}
    \input{plots/result/SD/ohpe_sd_cluster_size3.tikz}
    \caption{OHPE 3rd exam task.} 
    \label{fig-sd-clust-size:c} 
    \vspace{4ex}
  \end{subfigure}%% 
  \begin{subfigure}[b]{0.5\linewidth}
    \centering
    \setlength\figureheight{3.5cm}
    \setlength\figurewidth{\textwidth}
    \input{plots/result/SD/ohpe_sd_cluster_size4.tikz}
    \caption{OHPE 4th exam task.} 
    \label{fig-sd-clust-size:d} 
    \vspace{4ex}
  \end{subfigure} 
\caption{Relative size of the largest cluster in OHPE.}
\label{fig-sd-clust-size}
\end{figure}



\begin{figure}[!h] 
  \centering
  \begin{subfigure}[b]{0.5\linewidth}
    \setlength\figureheight{3.5cm}
    \setlength\figurewidth{\textwidth}
    \input{plots/result/SD/ohja_sd_cluster_size1.tikz}
    \caption{OHJA 1st exam task.} 
    \label{fig-sd-clust-size-ohja:a} 
    \vspace{4ex}
  \end{subfigure}%
  \begin{subfigure}[b]{0.5\linewidth}
    \setlength\figureheight{3.5cm}
    \setlength\figurewidth{\textwidth}
    \input{plots/result/SD/ohja_sd_cluster_size2.tikz}
    \caption{OHJA 2nd exam task.} 
    \label{fig-sd-clust-size-ohja:b} 
    \vspace{4ex}
  \end{subfigure}
   \begin{subfigure}[b]{0.5\linewidth}
    \setlength\figureheight{3.5cm}
    \setlength\figurewidth{\textwidth}
    \input{plots/result/SD/ohja_sd_cluster_size3.tikz}
    \caption{OHJA 3rd exam task.} 
    \label{fig-sd-clust-size-ohja:c} 
    %\vspace{4ex}
  \end{subfigure}
\caption{Relative size of the largest cluster in OHJA.}
\label{fig-sd-clust-size-ohja}
\end{figure}

Figure \ref{fig-sd-clust-size-ohja} shows same results for OHJA, which is an advanced course where a lot more programming skills are required from the students. This allows the tasks to be more difficult and longer, and as we see, the relative size has gone down in all cases compared to OHJA. Cluster sizes are now a lot smaller as exam tasks are more open-ended and more demanding. In other words, exam tasks have a range of multiple solutions and ways to do them, which minimizes the natural similarity between documents. This can be seen as a trend where none of the models now suffer from forming clusters containing majority of the retrieved documents, as the size is around 0.5 (50\%) at maximum. 


Retrieving majority of the documents is not optimal for plagiarism detection as these documents often needs to be manually inspected at the end. When inspecting the retrieval rate \ie 

\begin{equation}
    \text{Retrieval rate} = \dfrac{\#\text{Documents retrieved}}{\#\text{Documents in total}}
\end{equation}

\noindent
We can see that this rate should not get values near 1.0, as this would mean that all documents are detected as plagiarism, which is very unlikely in cases with hundreds of documents. In other words, models with a very high retrieval rates are almost guaranteed to have a high number of false positives unless there are only a small set of documents which are all plagiarized from one another. 

The retrieval rates of all models can be seen from the Figure \ref{fig-sd-retrieval-rate}.

\begin{figure}[ht] 
  \centering
  \begin{subfigure}[b]{0.8\linewidth}
    \centering
    \setlength\figureheight{5cm}
    \setlength\figurewidth{\textwidth}
    \input{plots/result/SD/ohpe_retrieval_rate.tikz}
    \caption{OHPE's exam tasks and the retrieval rates. Most models consider around 60-70\% of all document to be over the similarity threshold.} 
    \label{fig-sd-retrieval-rate:a} 
    \vspace{1ex}
  \end{subfigure}
  
  \begin{subfigure}[b]{0.8\linewidth}
    \centering
    \setlength\figureheight{5cm}
    \setlength\figurewidth{\textwidth}
    \input{plots/result/SD/ohja_retrieval_rate.tikz}
    \caption{OHJA's exam tasks and the retrieval rates. Models have similar retrieval rates and the rate of retrieval is low.} 
    \label{fig-sd-retrieval-rate:b} 
  \end{subfigure}
\caption{Retrieval rates of all models across every exam task. The model A keeps the lowest retrieval rate overall.}
\label{fig-sd-retrieval-rate}
\end{figure}

\noindent
The Figure \ref{fig-sd-retrieval-rate:a} shows how the the model A keeps the retrieval rate lowest around 50\%, meaning that half of the documents contain too much similarities between each others, and as said before it's very unlike so many documents are plagiarized. When comparing model A to other models, they claim the partition to be even higher, which is certainly not true. Interestingly in Figure \ref{fig-sd-retrieval-rate:b}, the models agree quite well, only having some level of disagreement with first exam exercise of OHJA. The second exercise shows good agreement, as all models have near 5\% retrieval rate.

Results on similarity detection show that tuning the two parameters $n$ and $\varepsilon$ is very data dependent as choosing the best performing combination might lead to very different results for other data sets. In our case, we choose the model A ($n=3, \varepsilon=0.2$) for the final evaluation, because that model had a decent $F_1$-score in SOCO-T, the precision for SOCO-T was nearly perfect, $F_1$ for both SOCO-C1 and SOCO-C2 were near 1.00. The model A also kept the largest cluster relatively small compared to other models and the retrieval rate for both OHPE and OHJA was the lowest, implying it could maintain a low rate of false positives. Keeping the rate of false positives minimal is more valuable us than retrieving every single plagiarism case, so we allow the model's detection rate to suffer with the benefit of having a high precision.  

To get perspective of how well our chosen model compares to the state of the art Java plagiarism detection tools, we first run JPlag detection for OHPE's and OHJA's exam tasks, then run our model for the same set of exercises and finally report the Jaccard similarity between the set of detected documents. For the JPlag, we use its default parameters and collect all document pairs where reported similarity score is above the same threshold as our model's $\varepsilon$-range, which in practice this means all documents where the reported similarity is over 80\% are collected. Following tables show results for both OHPE and OHJA with five metrics: documents detected by JPlag, documents detected by our chosen model, size of the intersection between the set of detected documents, number of unique documents retrieved in total and the Jaccard similarity score.

\begin{table}[ht]
\centering
\caption{Retrieval metrics for model A compared to JPlag with OHPE's exam tasks.}
\begin{tabular}{|c||c|c|c|c|}
\hline
\bf Exercise & 1. & 2. & 3. & 4. \\ \hline
\bf JPlag - Documents retrieved & 127 & 134 & 106 & 156 \\ \hline
\bf Model A - Documents retrieved & 109 & 130 & 111 & 114\\ \hline
\bf Common documents & 98  & 109 & 95 & 102\\ \hline
\bf Unique documents & 138 & 155 & 122 & 168\\ \hline
\bf Jaccard similarity    & 0.71  & 0.70  & 0.78  & 0.61  \\ \hline
\end{tabular}
\label{tbl-jacc-sd-ohpe}
\end{table}

\noindent
Table \ref{tbl-jacc-sd-ohpe} shows how our model agrees quite well with JPlag, as around 100 documents per exam task are shared. But even with the state of the art tool like JPlag one retrieves a lot of documents with a high threshold like 80\% for OHPE, as the retrieval rate with JPlag for all OHPE's tasks is around 50\%. This implies that even JPlag introduces false positives for restricted tasks, and minimizing false positives is a problem for every plagiarism detection tool.   

\begin{table}[ht]
\centering
\caption{Retrieval metrics for model A compared to JPlag with OHJA's exam tasks. JPlag retrieves just a few documents when using 80\% threshold.}
\begin{tabular}{|c||c|c|c|c|}
\hline
\bf Exercise & 1. & 2. & 3. \\ \hline
\bf JPlag - Documents retrieved & 2 & 2 & 0  \\ \hline
\bf Model A - Documents retrieved & 15 & 9 & 9 \\ \hline
\bf Common documents & 2  & 2 & 0\\ \hline
\bf Unique documents & 15 & 9 & 9\\ \hline
\bf Jaccard similarity    & 0.13  & 0.22  & 0.00  \\ \hline
\end{tabular}
\label{tbl-jacc-sd-ohja}
\end{table}

\noindent
The retrieval rate for OHJA's tasks for all our model candidates was very low, and this same result is reflected in Table \ref{tbl-jacc-sd-ohja} where JPlag retrieves only two documents or no documents at all. It seems that our model retrieves more documents than JPlag, but without a human interference it's impossible to say which one of the models is more correct. However, the retrieval from tasks 1. and 2. share the same two documents that JPlag detected, meaning that our model performs similar to JPlag but the scoring it produces is more consistent which can be seen when we inspect the third task where the level of agreement was the lowest. 

As we inspect every pair our model retrieved from OHJA's third task and compare the similarity scores to JPlag, we get five unique document pairs which are denoted here as $p_i, i \in [0, 5]$, formed by a total of nine documents. The results are visible in Figure \ref{fig-jplag-sd-ohja3}.

\begin{figure}[ht]
    \centering
    \setlength\figureheight{5cm}
    \setlength\figurewidth{0.8\textwidth}
    \input{plots/result/SD/model_a_vs_jplag_ohja3.tikz}
    \caption{The difference between JPlag's reported similarity value and our model for OHJA's final exam task.}
    \label{fig-jplag-sd-ohja3}
\end{figure}

\noindent
Figure \ref{fig-jplag-sd-ohja3} visualizes how our model keeps the similarity score near 80\% for every pair, whereas JPlag's score varies. The most similar scores are with pairs $p_1$ and $p_4$, where the difference is around 0.1 compared to our model. In other cases, it seems that JPlag can produce more specific results, because the comparing process differs from ours. We use the whole used vocabulary to produce the similarity score, whereas JPlag forms the score by string matching the token streams. 

We have now trained and evaluated our similarity detection model. The model we chose uses $n$-gram length of three, and retrieves any document where the calculated similarity value is above the 80\% threshold, which is reflected as $\varepsilon$-range of 0.2 in our clustering method. Our model was compared to JPlag and the retrieved documents were mostly the same, but there were some variance in number of documents retrieved. In following chapter, we train and evaluate the second model, the authorship identification.







\subsection{Authorship identification}
Our authorship identification model requires only one parameter to be tuned, the length
of character-level $n$-grams to be extracted. We tune this parameter based on the average $F_1$-score and accuracy
over seven split points for both OHPE and OHJA. For every split the final exercise is left out as an test data, 80\% of the remaining used for training and 20\% for validation. 

Course feedback is filtered out:

\begin{table}[ht]
\centering
\caption{(OHPE  1-7, OHJA 8-14) based on last exercise. Profile size is document count per student rounded to nearest integer.}
\label{lbl-result-ai-ohpe-ohja-stat}
\scalebox{0.65}{
    \begin{tabular}{|c||c|c|c|c|c|c|c||c|c|c|c|c|c|c|}
    \hline
    \bf Week         & \bf 1. & \bf 2. & \bf 3. & \bf 4. & \bf 5. & \bf 6. & \bf 7. & \bf 8. & \bf 9. & \bf 10. & \bf 11. & \bf 12. & \bf 13. & \bf 14. \\ \hline
    \bf Students     & 230  & 239  &  189 & 174  & 127  & 138  & 53  & 144  & 114  & 137   & 90   & 111   & 121   &  113  \\ \hline
    \bf AVG Profile size & 24  & 40  & 64  & 76  & 85  & 94  & 102  & 11  & 21  &  30  &  36  &  43  & 50   & 53    \\ \hline
    \end{tabular}
}
\end{table}


Results are shown below.


\begin{table}[ht]
\centering
\caption{Macro-averaged $F_1$-score (OHPE) for validation data}
\label{lbl-result-ai-f1-ohpe}
\begin{tabular}{|c|c|c|c|c|c|c|c|c|} \hline
\backslashbox{\bf $n$-gram}{\bf Week}  & 1 & 2 & 3 & 4 & 5 & 6 & 7 \\ \hline
4     &     0.01 & 0.03  & 0.03  & 0.04  & 0.04  & 0.04  & 0.04    \\ \hline
6     &  0.02    & 0.04  & 0.05  & 0.05  & 0.05  & 0.05  & 0.05    \\ \hline
8     & 0.02     & 0.04  & 0.05  & 0.06  & 0.06  & 0.06  & 0.06    \\ \hline
10    &  0.02    & 0.05  & 0.06  & 0.06  & 0.07  & 0.07  & 0.07     \\ \hline
12    & 0.02     & 0.05  & 0.06  & 0.06  & 0.07  & 0.07  & 0.07     \\ \hline
14    & 0.02     & 0.05  & 0.06  & 0.07  & 0.07  & 0.07  & 0.07    \\ \hline
\end{tabular}
\end{table}


Because the result were so poor, we limited the amount of students. Using the last week where students have the largest profile, we plot the accurace for various author sizes.

\newpage

\begin{figure}[ht]
\centering
\setlength\figureheight{7cm}
\setlength\figurewidth{\textwidth}
\input{plots/ohpeohja_ai_ng14.tikz}
\caption{Accuracy on validation set when different size of author pools are being used. Accuracy deteriorates when more than two authors are present. (ng=14, char, 7 weeks))} \label{fig-ohpeohja-ai-ng14}
\end{figure}

From Figure \ref{fig-ohpeohja-ai-ng14} we see that the model completely fails to predict the author when the size of authors is increased. 



\begin{table}[ht]
\centering
\caption{Test}
\label{asdasd}
\begin{tabular}{|c|c|c|c|c|c|c|c|}
          & NG & 4 & 6 & 8 & 10 & 12 & 14 \\
In set of &    &   &   &   &    &    &    \\
5         &    &   &   &   &    &    &    \\
10        &    &   &   &   &    &    &    \\
20        &    &   &   &   &    &    &    \\
25        &    &   &   &   &    &    &   
\end{tabular}
\end{table}

\begin{table}[ht]
\centering
\caption{Macro-averaged $F_1$-score (OHJA) for validation data}
\label{lbl-result-ai-f1-ohja}
\begin{tabular}{|c|c|c|c|c|c|c|c|c|} \hline
\backslashbox{\bf $n$-gram}{\bf Week}  & 1 & 2 & 3 & 4 & 5 & 6 & 7 \\ \hline
4     &      &   &   &   &   &   &     \\ \hline
6     &      &   &   &   &   &   &     \\ \hline
8     &      &   &   &   &   &   &     \\ \hline
10    &      &   &   &   &   &   &      \\ \hline
12    &      &   &   &   &   &   &      \\ \hline
14    &      &   &   &   &   &   &     \\ \hline
\end{tabular}
\end{table}




\newpage


\begin{table}[ht]
\centering
\caption{My caption}
\label{lbl-result-ai-best-model}
\begin{tabular}{|c|c|c|c|c|c|c|c|} \hline
Week & 1 & 2 & 3 & 4 & 5 & 6 & 7 \\
ACC  &   &   &   &   &   &   &  \\ \hline
\end{tabular}
\end{table}

\subsection{PLGDetect}
Because the Multinomial Naïve Bayes and the SCAP evaluated poorly with our data sets, we decide not to use authorship identification for the final results as even reducing the amount of authors would diminish the possibility of finding any plagiarists as random sampling would leave some students out of the detection. This is a drawback for our approach and we discuss the implications at the discussion. However, our similarity detection model evaluated well and can be still used for exploring and detecting the possible plagiarists. What we can't do is to restrict efficiently the amount of false detections by using the authorship identification model.

Before we can discuss the final results, we must consider an issue with the retrieval rate of our similarity detection. Looking from the Table \ref{tbl-jacc-sd-ohpe} and Table \ref{tbl-jacc-sd-ohja} there are around 500 total documents retrieved, which is too many documents for the human expert to go through in reasonable time. To overcome this issue the we select only a subset of the exam tasks reducing the amount of documents to 144. These are OHPE's third exam task (3.A) and all of the exam tasks of OHJA's (1.B, 2.B, 3.B). A brief description of each selected task is given below.

\paragraph{3.A (OHPE)} Students were required to fill a method to find the most common number from the Java's ArrayList structure. The methods name, return value and parameters were given as a template. 

\paragraph{1.B (OHJA)} Students were required to make a text interface for adding books with name and year information. The outline of the text interface was given for the students. After the initial adding phase, added books were printed in wanted order.

\paragraph{2.B (OHJA)} This task measured how well students are able to manipulate text data. The task required to have a small text interface to read a text file, censor every occurrence of a given word and write the results to a new text file. This exercise had a hint, which recommended to use a specific Java class to read and write text files. 

\paragraph{3.B (OHJA)} Task required to create a text interface to emulate a simple storage management software. The actions that had to be implemented were adding, listing, searching, removing items and exiting the interface. A small piece of code was given as a hint for this exercise.

\mbox{}\\
\noindent
In all exam tasks, also the scoring and example output was given for the students, so that they could mimic the wanted functionality of these programs. The reason behind this was to guide the student into right direction and also to be able to automatically score the submissions.

To see the difference between these tasks, descriptive statistics about them is given in Table \ref{tbl-plagdet-desc-stat}. It shows how OHPE differs from OHJA, as its task is quite constrained having only around 50 lines to get a correct answer. OHPE also creates a lot more clusters, as the similarities between OHJA's submissions are more varied.


\begin{table}[ht]
\centering
\caption{Results before the evaluation by the human expert. These results are produced by our similarity detection model which uses parameters $n=3$ for the $n$-gram length and $\varepsilon=0.2$ for the maximum allowed distance between the documents, which reflects that the documents have to score over 80\% similarity in order to cluster them together.}
\begin{tabular}{|c||c|c|c|c|}
\hline
\bf Task                & 3.A & 1.B & 2.B & 3.C  \\ \hline
\bf Number of submissions & 227 & 200 & 198 & 197 \\ \hline
\bf Average line count         & 47   & 160    & 85   & 150     \\ \hline
\bf Documents retrieved & 111 & 15 & 9 & 9 \\ \hline
\bf Clusters emerged & 15 & 5 & 3 & 4 \\ \hline
\end{tabular}
\label{tbl-plagdet-desc-stat}
\end{table}

\noindent
As the final result, we first show the pair-level detection results and then the more general result, which shows the precision with respect to documents considered containing plagiarism. For each of these tasks we inspect every cluster and the true and false positives in them, where the results are given by our human expert who has manually gone through detected documents. Results for each task is given in following figures, where we show the frequencies of retrieved pairs compared to true positives. Note that this format is more fine grained than what we have used before as earlier we have reported only the number of documents detected, and that we had to prune the first cluster of OHPE's third task, as it contained nearly 410 pairs. Pruning was done by keeping only the pairs where the cosine similarity was 1.0.

\newpage


\begin{figure}[ht] 
    \setlength\figureheight{7cm}
    \setlength\figurewidth{\textwidth}
    \input{plots/result/AI/plgdet_cluster_tpfp_ohpe_3.tikz}
    \caption{Detected and true pairs of 3.A OHPE. False positives in the first cluster were mostly correct submissions which were similar to model solution. Fourth cluster contained almost empty submissions and sixth cluster similarly wrong solutions with two highly suspicious authors.}
    \label{fig-plgdet-res3a}
\end{figure}

\begin{figure}[!h] 
    \setlength\figureheight{6cm}
    \setlength\figurewidth{\textwidth}
    \input{plots/result/plgdet/plgdet_cluster_tpfp_ohja_1.tikz}
    \caption{Detected and true pairs of 1.B OHJA. Most of the pairs were reported to be close to model solution without any signs of plagiarism. However, there were two pairs which were flagged for further attention.}
     \label{fig-plgdet-res1b}
\end{figure}

\newpage

\begin{figure}[ht] 
    \setlength\figureheight{6cm}
    \setlength\figurewidth{\textwidth}
    \input{plots/result/plgdet/plgdet_cluster_tpfp_ohja_2.tikz}
    \caption{Detected and true pairs of 2.B OHJA. All of the detected pairs in this task were false positives. However, two non-paired authors were flagged for further attention.}
    \label{fig-plgdet-res2b}
\end{figure}

\begin{figure}[!h] 
    \setlength\figureheight{6cm}
    \setlength\figurewidth{\textwidth}
    \input{plots/result/plgdet/plgdet_cluster_tpfp_ohja_3.tikz}
    \caption{Detected and true pairs of 3.B OHJA. Three pairs were flagged for further attention, but as difficult cases.}
    \label{fig-plgdet-res3b}
\end{figure}

\noindent
In Figures \ref{fig-plgdet-res3a}, \ref{fig-plgdet-res1b} and \ref{fig-plgdet-res3b}, we see that our approach is able to retrieve suspicious documents. As reported by the human expert, most of true positives contain direct copies and renaming of the variables. However, there exist false positives as seen in Figure \ref{fig-plgdet-res2b} where most of these false positives are caused by natural similarity between the submissions. The human expert reported also that in most of the cases one can't say for sure that the document pair is plagiarism. Therefore, the reported pairs are flagged if they are considered as suspicious and would require further information \eg other submissions done by the pair of authors. In the table below, one sees the document level results of false and true positives with the level of precision for each task.

\newpage


\begin{table}[ht]
\centering
\caption{Document-level results of our plagiarism detection. There are false positives introduced to our detection results.}
\begin{tabular}{|c|c|c|c|c|}
\hline
\bf Task      & 3.A   & 1.B   & 2.B & 3.B   \\ \hline
\bf True Positives        & 30   & 4    & 0  & 6    \\ \hline
\bf False Positives        & 26   & 11   & 9  & 3    \\ \hline
\bf Precision & 0.54 & 0.27 & 0.00  & 0.67 \\ \hline
\end{tabular}
\label{tbl-plgdet-final-res}
\end{table}

\noindent
The low precision in Table \ref{tbl-plgdet-final-res} shows how our model fails to limit the amount of false positives, which can be mostly due to the fact that we had to use only the similarity detection part of our approach. As seen before, all of the submissions for OHPE and OHJA contain a high level of natural similarity, which introduces many false positives even with as high threshold as 80\%. To help the work of our human expert, we had to prune the first cluster of OHPE's third task. In reality there would be near 400 detected document pairs, which are clearly all false positives due to the restricted solution space of the task.


After the human expert evaluated the detected documents, the five plagiarists caught in 2016 were revealed to us. Our model was able to retrieve documents belonging for all of these authors in OHPE's third task. 



\section{Discussion}
Following three research questions are asked and answered in this study, which are all tied closely to the question \emph{How plagiarism can be automatically detected?}

\begin{itemize}
    \item[Q1:] \emph{What kind of approaches exist to detect source code plagiarism?}
    \item[Q2:] \emph{What are the possible benefits of using code structure for plagiarism detection?}
    \item[Q3:] \emph{How one can reduce the amount of false accusations?}
\end{itemize}



% pair programming
% worked example

% why similarity
% why ai left out


\section{Conclusion}
































%

% --- References ---
%
% bibtex is used to generate the bibliography. The babplain style
% will generate numeric references (for example [1]) appropriate for theoretical
% computer science. If you need alphanumeric references (e.g [Tur90]), use
%
%\bibliographystyle{babalpha-lf}
%
% instead.
\newpage
\bibliographystyle{babplain-lf}
\bibliography{references-fi}


% --- Appendices ---

% uncomment the following

\newpage
\appendix
% 

\section{Sample programs} \label{appendix:programs}

These three functionally similar source codes belong to three imaginary authors A, B and C. They are used throughout the study as examples. The task for all is to create a program that calculates mean between three numbers: 5, 10, 2.

\begin{lstlisting}[language=Java, caption=Java example belonging to author A]
public class A{

     public static void main(String[] args){
        int a = 5;
        int b = 10;
        int c = 2;
        double d = (a + b + c)/(double)3;
        System.out.println(d);
     }
}
\end{lstlisting}

\begin{lstlisting}[language=Java, caption=Java example belonging to author B]
public class B{

     public static void main(String[] b){
        int sum = 5 + 10 + 2;
        double res = sum / 3.0;
        System.out.println(res);
     }
}
\end{lstlisting}

\begin{lstlisting}[language=Java, caption=Java example belonging to author C]
public class C{
     public static void main(String[] b){
        System.out.println((5 + 10 + 2)/3.0);
     }
}
\end{lstlisting}

\newpage

\section{Token list} \label{appendix:token-list}

\begin{table}[ht]
\centering
\caption{Token list for Java.}
\label{tbl-token-list}
\def\arraystretch{1.5}
\scalebox{0.75}{
    \begin{tabular}{c l p{7cm} l} 
     &  \bf Token  & \bf Equivalency          & \bf Example    \\ \hline
    1 & IMPORT & Import declaration & \texttt{import java.awt.*;} \\ \hline
    2 & PACKAGE & Package declaration & \texttt{package foo;} \\ \hline
    3 & VARDEF & Variable declaration & \texttt{int a;} \\ \hline
    4 & CLASS\{      &  Enter class declaration   &  \texttt{public class A\{} \\ \hline
    5 & CATCH\{ & Enter catch clause & \texttt{try \{\underline{catch (...)\{}\} \}} \\ \hline
    6 & INCLASS\{ & Statement inside a class & - \\ \hline
    7 & ENUM\{ & Enter enum declaration & \texttt{public enum Day \{} \\ \hline
    8 & APPLY & Method call, Explicit constructor invocation, Generic invocation  & \texttt{System.out.print(...);} \\ \hline
    9 & NEWCLASS & Create object & \texttt{new A(...);} \\ \hline
    10 & NEWARRAY & Create array object & \texttt{new int[5];} \\ \hline
    11 & TRY\{ & Enter try declaration & \texttt{try \{ } \\ \hline
    12 & INTERF\{ & Enter interface declaration & \texttt{interface Foo \{} \\ \hline
    13 & METHOD\{ & Enter method declaration & \texttt{void foo(int a) \{} \\ \hline
    14 & VOID & Void method & \texttt{void main(String[] args)} \\ \hline
    15 & CASE & Case in switch statement & \texttt{case MONDAY:} \\ \hline
    16 & CONSTR\{ & Enter constructor declaration & \texttt{public A(int a, int b) \{} \\ \hline
    17 & ARRINIT\{ & Enter array initialization & \texttt{new int[] \{1, 2\};} \\ \hline
    18 & ASSIGN & Variable assignment & \texttt{a += 5;} \\ \hline
    19 & COND & Conditional expression & \texttt{(a > b) ? a : b;} \\ \hline
    20 & LOOP\{ & Enter for, while, do statement & \texttt{for(...) \{}\\ \hline
    21 & IF\{ & Enter if clause & \texttt{if(...) \{} \\ \hline
    22 & THROW & Throw statement & \texttt{throw new Exception();} \\ \hline
    23 & BREAK & Break statement in loop & \texttt{break;} \\ \hline
    24 & CONTINUE & Continue statement in loop & \texttt{continue;} \\ \hline
    25 & RETURN & Return statement & \texttt{return a + b;} \\ \hline
    26 & SWITCH\{ & Enter switch statement & \texttt{switch(...) \{} \\ \hline
    \end{tabular}
}
\end{table}

\noindent
Table \ref{tbl-token-list} shows the token list used to transform a parse tree into a continuous string of tokens. Every token with ending bracket also has a reserved token when exiting the statement.



\end{document}
