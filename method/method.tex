
In this thesis, we explore the possibility of using a two-phase model that combines both similarity detection and authorship identification, and hypothesize that such model increases the precision of the plagiarism detection. It is noticeable that none of the studies that were discovered during the literature review in Chapter \ref{chap-liter-review} combine these two approaches when making the decision of possible plagiarism. Therefore our goal is to introduce a new approach which make the use of both tasks submitted by students to create an author profile for each student, and document clustering to retrieve similar documents. Building an author profile happens in our setting naturally because academic courses are often offered as weekly exercise sets which are supplemented by lessons, meaning that the accuracy of the author identification should get better as more data for a student is produced each week. 



We base our approach to a two phase model which could reduce the amount of false-positives found in source code plagiarism detection. False-positives are problematic as it means an innocent author is considered to be a possible plagiarist, and therefore having too sensitive model introduces extra work. The two phases of our model are similarity detection and authorship identification, where both of them are able to define a set of authors; similarity detection reveals suspicious authors based on the file similarity and authorship identification predicts candidate authors for a given document $d$. 

%pre-process documents, tokenize documents, exclude templates, calculate similarities and finding suspects using these similarity scores.

Both of our models are based on other studies presented in the literature review and combine the high-level approach used in many tools \cite{RSCAD2016}: preprocess, normalize, evaluate and predict. For similarity detection we use lower level features that capture the structure and are resilient against transformations introduced in Chapter \ref{chap-liter-review-methods}, and for authorship identification we use higher level features which can capture the style of an author. The generalization of the proposed model is given below.

\begin{algorithm}[ht]
\caption{Detecting plagiarism between a set of source code files.}
\label{alg-toplvl}
\begin{algorithmic}

\Require Set of authors $A$
\Require Set of source code files $D$ written by various authors $\forall a \in A$
\Require Index of the exercise of interest $i \in \mathbb{N}$
\Require Length of word level $n$-grams $n_w \in \mathbb{N}$
\Require Length of character level $n$-grams $n_c \in \mathbb{N}$
\Require Minimum rate of similarity $\varepsilon \in [0, 1]$
\Procedure{PLGdetect}{$A, D, i, n_w, n_c, \varepsilon$}
   \State $D'\gets$ \Call{normalize}{$D$}
   \State $A_{susp} \gets$ \Call{detectSim}{$A$, $D'_i, n_w, \varepsilon$}
   \State $A_{auth} \gets$ \Call{trainAndPredictAuthor}{$A, D', i, n_c$}
   \State \textbf{return} $A_{auth} \cap A_{susp}$
\EndProcedure

\end{algorithmic}
\end{algorithm}

\noindent
Algorithm \ref{alg-toplvl} requires six parameters so it can fully function and the most import ones are the collection of documents $D$, which all are assigned respectively to authors inside a set $A$ \ie all source codes are submitted by a known author. Remaining four parameters ($i, n_w, n_c, \varepsilon$) can be defined freely but in this thesis we estimate the latter three by running a series of tests and choosing the best performing values. Therefore the only parameter we can't estimate is of course the index of interest, which in best case iterates over every exercise. However, for the sake of clarity we later choose only a subset of tasks \eg the most challenging tasks, to be under inspection.   

The flow of Algorithm \ref{alg-toplvl} is following. Source code files are first normalized for similarity and author detection separately. Then similarity is detected for a collection of documents belonging under same exercise with the function \textproc{DetectSim}, which forms a group of suspicious authors noted as $A_{susp}$. The similarity detection process is controlled by the parameter $\varepsilon$ which acts as a threshold for the detection. For example $\varepsilon = 1.0$ means that documents must be exact copies in order to group them together. The function \textproc{trainAndPredictAuthor} trains our authorship identification model with previous documents that the author has written, and then predicts who are the most likely authors from the set $A$ for $i$th exercise noted as $A_{auth}$. 

Our final result is the intersection between sets $A_{susp}$ and $A_{auth}$, the results of similarity detection and authorship identification. Our intuition behind this can be shown with the following example. 

\begin{example}
Let there be three authors $a,b,c \in A$ and three exercises under detection $d_a, d_b, d_c \in D$, where $D$ contains also previous submissions for each author. Let there also exist a similarity detection phase able to cluster perfectly when document similarity is over the threshold $\varepsilon$, and authorship identification model trained with 100\% accuracy so that the expected error when classifying any of the documents $d_a, d_b, d_c$ as is minimal. 

If the clustering result is that $\omega_1 = \{d_a,d_b\}$ and $\omega_2 = \{d_c\}$ implying authors $a$ and $b$ are suspects as they share too much structural similarity, and the identification predicts following authors: $\hat{f}(d_a) = a, \hat{f}(d_b) = a$ and $\hat{f}(d_c) = c$. Then we have verified that authors $a$ and $b$ have a high chance for a case of plagiarism as their submissions are too similar and the style of a document send by author $b$ matches more the style of the author $a$. We claim that $a$ has probably shared the document to $b$, but because in this thesis we completely leave out the direction of plagiarism, both cases should be reviewed equally by a human expert \ie we treat sharing as equally serious offense as copying.
\end{example}




\subsection{Assumptions}

We mainly focus on academia and especially to programming courses that are offered by 
universities. Following five assumptions are defined to simplify the problem of plagiarism
detection by allowing us to focus only on plagiarism that happens in a closed environment and within a closed set of documents. 

\paragraph{In-class plagiarism} Plagiarism has occured only inside a 
specific course implementation. Let $\mathcal{P}(A)$ be a powerset of students within offered courses in university. We are only interested about a set of students referred as authors $A$ attended in a specific course $c$ \ie a subset $A_c \subseteq \mathcal{P}(A), A_c \neq \emptyset$. The corpus $D_c$ is built by gathering every submission done by students $\forall a \in A_c$ and a set of documents belonging to individual student is defined as $D_a = \{d \mid d \in D_c, a = auth(d)\}$. 


\paragraph{Exercise focus} 
Let $E_c = \{e_1, e_2, ..., e_n\}$ be a set of exercises for a course $c$, then submissions for a single exercise is represented by a subset $D_{c,e} \subseteq D_c$. With this assumption, we focus the plagiarism detection to submissions done to a single exercise at a time \ie plagiarism can happen only between submissions to a single exercise.

\paragraph{Single author} 
Every source code $d \in D_c$ is assumed to have a single author $a = auth(d), a \in A_c$. This allows us to assume that every source code submissions is done as a individual work, and all results that suggests otherwise implies about the case of excessive collaboration \ie plagiarism. 

\paragraph{Plagiarism direction} 
Let a file $d_i$ be plagiarized from $d_j$: $d_i \xrightarrow{plag} d_j$. We treat this as same as the opposite direction $d_i \xleftarrow{plag} d_j$, making the direction of plagiarism unimportant. This means that we treat both cases sharing and copying, as an act of supporting plagiarism. 

\paragraph{Expert interference}
We believe that no system can be accurate enough to autonomously accuse students about plagiarism. However, this is doable when some form of human judgment is added to the model. In principal this means that the model can make predictions about cases of plagiarisms which we call \emph{suspects}, but the human expert must make the \emph{allegation} of plagiarism based on the results and after questioning the students. Having guidelines about what is considered as plagiarism and how such cases are handled\footnote{University of Helsinki's guidelines: \url{https://blogs.helsinki.fi/alakopsaa/?lang=en} Accessed 9th May 2018}, helps both students and teachers to understand what the institution means when it accuses of plagiarism. 

\subsection{Data set}

Our model is aimed to the traditional MOOC setting which is for example used by  undergraduate-level programming courses \emph{Introduction to Programming} (OHPE) and \emph{Advanced course in Programming} (OHJA) in University of Helsinki. We use three authentic data sets; students submissions done to both of latter courses during the implementation in fall 2016 and a train data from SOCO task from 2014. All source code files are written in Java programming language. 

Both OHPE and OHJA includes proven cases of plagiarism, however to avoid any bias, more specific information about them is kept hidden as a golden standard by the courses administrative staff Arto Hellas until the final evaluation of our model. SOCO dataset on the other hand, contains prelabeled document pairs that have conducted plagiarism \cite{saez2014pan}.

To implement our model, we first use SOCO to train and evaluate our similarity detection model, then train and test authorship identification with OHPE and OHJA. Our proposed model is built based on these results and plagiarism is detected individually for both courses. The reason to use train set of SOCO for similarity detection, is simply that it's the only data set that contains fully labeled cases of plagiarism, but unfortunately contains only one file per author. OHPE and OHJA on the other hand, contains multiple files per author making author identification possible, but only a few known cases of plagiarism. Therefore we make use of both sets and consider our model to be successful if it has a high precision, minimizing the amount of false-positives. 

\paragraph{Course overview}\mbox{}\\
OHPE and OHJA shares the same structure; students first register to automatic scoring system called \emph{Test My Code} (TMC) \cite{Vihavainen:2013:SSL:2462476.2462501} which also distributes the exercises as an plugin to \emph{NetBeans IDE}, then independently work during seven weeks by completing programming exercises within deadlines \cite{Vihavainen:2012:MSM:2380552.2380603}. Both of these courses follow \emph{Extreme Apprenticeship method} \cite{Vihavainen:2011:EAM:1953163.1953196}; theoretical material is available online for students, students learn by doing \ie there exists mandatory programming exercises, weekly exercise sessions are available for those who require assistance, instructors can give feedback and students are able to track their process. 

Students earn points from exercises depending if all tests were successfully passed via TMC, and complete an exam at end of the course which is a programming exam that ultimately decides if a student has learned the minimum level required. The exam in fall 2016 was a home exam, meaning that students were able to do it individually wherever they wanted to. There are also no mandatory lectures, thus students are able to earn these credits by working individually without any physical attendance. 

\paragraph{SOCO overview}\mbox{}\\
Source code reuse (SOCO) data is from a 2014 competition \emph{PAN@FIRE}, where two sets were given to detect monolingual source code re-use \cite{saez2014pan}. SOCO2014 offered a train and a test set for competitors, which contained files written in \cpp\, and Java by various authors. The train set contains the source code files and annotations which are made by three experts flagging which pairs are considered as plagiarism. Competitors were then asked to retrieve which pairs are plagiarized. For example pair $(d_i, d_j)$ refers that there exists plagiarism between these two files, and because the direction was completely ignored, it was sufficient to retrieve just the predicted pairs.

SOCO contains mainly submissions to a single exercise and couple of documents, that are transformed from C to Java. As only the plagiarized file pairs are annotated and SOCO has been used successfully used in other studies \cite{AIR2015, RCISCP2017, OTIOLSS2015, USCR2014}, we make a simplifying assumption that the train set of SOCO contains one file per one unique author and that all submissions are submitted for the same task. This won't affect negatively the performance of our proposed model, as similarity detection is not affected at all if there exists multiple tasks within a corpus. 



\paragraph{Corpus statistics}\mbox{}\\
We are going to focus to Java language, therefore we only use the Java-specific part of SOCO training set, but fully utilize OHPE and OHJA data sets due to a fact that they only contain Java files. Number of non-transformative steps has been made beforehand to form the upcoming corpora: 1) leave SOCO as it is, 2) add exams to both OHPE and OHJA, and 3) concatenate submission containing multiple files into one file. This allows us to assume only one file per submission and we also get the benefit of having exam submissions, which is something where plagiarism is absolutely not allowed. 

%As OHPE and OHJA are both real-life courses, we also include the exam which in OHPE is made out of four tasks and in OHJA out of three tasks. 

Descriptive statistics for all three collections without any textual preprocessing is given in table \ref{tbl-corporastats}, where ten different metrics are reported: number of total authors, exercises and documents; does the corpus contains synthetic data; means for documents per author, character count, lines of code (LOC) and expressions\footnote{We assume countable expressions to be the ones ending in a semicolon}; and lastly minimum and maximum line counts. We can see from the table \ref{tbl-corporastats}, that SOCO has the smallest amount of authors but the tasks are more complex indicated by the largest LOC, amount of expressions and character count. 

When comparing OHPE to OHJA, OHPE has relatively smaller submissions than OHJA, which is mostly due to OHPE having easier tasks due to being the introductory course where students are not expected to know anything about programming beforehand. OHPE also has the most largest document-to-author ratio (106) compared to SOCO (1) and OHJA (56), making it the most richest data set when it comes to having a large amount of submissions per author.  Comparing to other corpora presented in chapter \ref{subsec-liter-data}, our OHPE corpus is one of the largest with OHJA. They both have over four times as many authors than any of the corpora used in other studies.

\begin{table}[!h]
\centering
\caption{Descriptive statistics for the unprocessed corpora. Bold values represents maximum value per metric.}
\label{tbl-corporastats}
\begin{tabular}{|l||l|l|l|} \hline
\backslashbox{\bf Metric}{\bf Corpus}   & SOCO & OHPE & OHJA\\  \hhline{|=|=|=|=|}
\textbf{Authors}         & 259 & \textbf{316} & 270   \\  \hline
\textbf{Exercises}       & 1 & \textbf{151} & 92     \\  \hline
\textbf{Documents}       & 259 & \textbf{33\,363} & 15\,196    \\  \hline
\textbf{Average documents per author} & 1 & \textbf{106} & 56\\ \hline
\textbf{Synthetic}       & Partly & No & No \\  \hline
\textbf{LOC $\min$}         & \textbf{12} & 1 & 1      \\  \hline
\textbf{LOC AVG.}        & \textbf{149} & 44 & 109     \\  \hline
\textbf{LOC $\max$}         & \textbf{1696} & 679 & 637   \\  \hline
\textbf{Expression AVG.}       & \textbf{63} & 17 & 38 \\ \hline
\textbf{Character AVG.} & \textbf{3898} & 1139 & 2794   \\  \hline
\end{tabular}
\end{table}



A problem however arises when average line count with respect to the exercises is visualized for both OHPE and OHJA. Figure \ref{fig-hists} visualizes this by histograms, where both bin sizes are set to 50. 


\begin{figure}[!h]
\centering
\captionsetup[subfigure]{justification=centering}

\begin{subfigure}{\textwidth}
    \setlength\figureheight{4cm}
    \setlength\figurewidth{\textwidth}
    \input{plots/ohpe_avgloc.tikz}
    \label{fig-ohpeavgloc}
\end{subfigure}

\begin{subfigure}{\textwidth}
  \setlength\figureheight{4cm}
    \setlength\figurewidth{\textwidth}
    \input{plots/ohja_avgloc.tikz}
    \label{fig-ohjaavgloc}
\end{subfigure}

\caption[Two histograms for corpora]{Histograms showing average line of count per exercise for OHPE (top) and OHJA (below). OHJA has more evenly distributed submissions, where as OHPEs submissions are mostly under 100 lines in length.}
\label{fig-hists}
\end{figure}

\noindent
From Figure \ref{fig-hists} we see that majority of the submissions for OHPE has under 100 lines of code. This creates an issue for plagiarism detection, as there exists tasks where the submission can only contain a few dozen lines meaning, that the similarities between solutions will be naturally high. 

To overcome this problem, we target our detection to the most challenging exercises per week which we assume is the final exercise of the week as they are mostly the longest tasks. The data supports this claim, as the mean length of every weeks last exercise submissions is seen in Table \ref{tbl-OHPE-last-week}.

\begin{table}[ht]
\centering
\caption{Average line count for submission of the final exercise of each week for OHPE. The only outlier is the last weeks exercise.}
\label{tbl-OHPE-last-week}
\begin{tabular}{l|c|c|c|c|c|c|c}
\bf Week        & 1.  & 2.  & 3.   & 4.  & 5.   & 6.   & 7.   \\ \hline
\bf Average LOC & 71 & 66 & 149 & 95 & 146 & 206 & 123 \\ \hline
\bf No. longest & 2nd  & 1st & 1st  & 1st & 1st   & 1st   & 4th  
\end{tabular}
\end{table}

\noindent
In Table \ref{tbl-OHPE-last-week} we don't take in account pair exercises which are meant to be done with another student, and we leave them completely out from the corpus as they violate our single author assumption. Also they are quite sparse; OHPE contains 12 pair programming tasks out of 151 as OHJA 10 out of 92. 



\subsection{Document normalization}

We utilize same approaches as studies reviewed in Chapter \ref{chap-liter-review-methods} to minimize the variance between documents. The benefit of normalization mainly is, that it reduces the vocabulary size by unifying language structures which are unimportant. However, with normalization we can also emphasize certain aspects. In case of similarity detection we want to preserve as much structural information as possible, and in case of authorship identification the students author profile must be captured. This means we can ignore all stylistic preferences in similarity detection and all structural information in authorship identification, because they share different goals.

For similarity detection we transform every document into a token stream by first parsing the program with a parser and turning it into abstract syntax tree, then traversing the structure to get the stream as a string format. This method allows to capture the higher-level structure of the program, and still allows to handle it as a text. Also, it works against obfuscation strategies which were stated in Tables \ref{tbl-plag-strat}, \ref{tbl-plag-transf} in Chapter \ref{chap-bg-sc-plag}, by ignoring certain aspects. For example the parses will ignore all white spaces, comments, identifier names and standardizes loop names. This means it works against levels 1,2 and 5 of Table \ref{tbl-plag-transf}. The parser itself only works with Java and is heavily inspired by the one used in JPlag \cite{prechelt2002finding}. The complete list of tokens is seen in Appendix \ref{appendix:token-list}, which shows also the equivalencies to generate certain tokens. For example all loop constructs generate a single token "LOOP\{" to indicate start of the loop. This normalizes the documents to preserve the underlying logic behind them, so even if the plagiarists would use a different looping construct like while instead of for, two programs would still share a common token stream. 

Table \ref{tbl-token-stream} shows the corresponding token stream for program A in Appendix \ref{appendix:programs}, where one can see how much information is discarded from the source code as we only keep the crucial structural information.

\begin{table}[ht]
\centering
\caption{Token stream generated from the example source code in Appendix \ref{appendix:programs}.}
\label{tbl-token-stream}
\begin{tabular}{|l|l|} \hline
\bf Original source code & \bf Token stream \\ \hline
\begin{lstlisting}
public class A{

     public static void main(String[] args){
        int a = 5;
        int b = 10;
        int c = 2;
        double d = (a + b + c)/(double)3;
        System.out.println(d);
     }
}
\end{lstlisting}                     &

\begin{lstlisting}
CLASS{  
VOID    
METHOD{ 
VARDEF 
ASSIGN  
VARDEF 
ASSIGN  
VARDEF 
ASSIGN  
VARDEF 
ASSIGN  
APPLY   
}METHOD 
}CLASS
\end{lstlisting}
\\ \hline      
\end{tabular}
\end{table}

\newpage

For the authorship identification, normalization method we apply uses the same idea as in \cite{AIRTSCAA2009, SCANG2007}. We discard all comments and normalize literal values to remove any possible notion of the original author, like unique student number or name in comments. The purpose behind normalization for authorship identification is therefore to leave the original document as intact as possible, maintaining the preferences that the programmer might have for \eg variable naming or spacing. An example of the normalization procedure is given in Table \ref{tbl-ai-normalization} for the same program used in Table \ref{tbl-token-stream}, where one can see that all numerical values have been transformed under a single dollar token \texttt{\$}.

\begin{table}[ht]
\centering
\caption{The result of normalization procedure for authorship identification.}
\label{tbl-ai-normalization}
\begin{tabular}{|l|} \hline
\bf Normalized code \\ \hline
\begin{lstlisting}
public class A{

     public static void main(String[] args){
        int a = $;
        int b = $;
        int c = $;
        double d = (a + b + c)/(double)$; 
        System.out.println(d);
     }
}
\end{lstlisting} 

\\ \hline      
\end{tabular}
\end{table}

\subsection{Document representation}

To represent every documents as vector, we use information retrieval techniques introduced in Chapter \ref{chap-IR}. Plagiarism detection is therefore done first by converting document into vector space model after the normalization of documents. In both similarity detection and authorship identification documents terms are first extracted, which in our case means all possible $n$-grams with respect to vocabulary $\mathbb{V}$. The only difference being that in similarity detection the vocabulary is formed using every document as a token stream, where as authorship identification uses only part of the complete data to form the available vocabulary. 

To overcome the problem with varying document length and frequently appearing terms, we apply TF-IDF weighting introduced in Chapter \ref{chap-IR-document-repr}. Table \ref{tbl-ngram-sd} shows example of the term extraction for similarity detection using word level 2-grams for program A in Appendix \ref{appendix:programs}. All TF-IDF weights have been normalized using Euclidean norm 

\begin{equation}
    \dfrac{\bolditt{x}}
          {\sqrt{\sum \limits_i^{|\mathbb{V}|} x_i^2}}
\end{equation}

% show example calculation

\begin{table}[ht]
\centering
\caption{Similarity detection term extraction for document A. Terms are word-level 2-grams extracted from the token stream, whereas TF-IDF weights have been normalized and values rounded at two decimal places.}
\label{tbl-ngram-sd}
\begin{tabular}{l|c|c}
\bf Term & \bf Raw frequency & \bf TF-IDF weight \\ \hline
    \texttt{APPLY \}METHOD} & 1 & 0.14\\
    \texttt{ASSIGN APPLY} & 1 & 0.18\\
    \texttt{ASSIGN VARDEF} & 3 & 0.55\\
    \texttt{CLASS\{ VOID} & 1 & 0.14\\
    \texttt{METHOD\{ APPLY} & 0 & 0.00\\
    \texttt{METHOD\{ VARDEF} & 1 & 0.18\\
    \texttt{VARDEF ASSIGN} & 4 & 0.74\\
    \texttt{VOID METHOD\{} & 1 & 0.14\\
    \texttt{\}METHOD \}CLASS} & 1 & 0.14\\
\end{tabular}
\end{table}

\noindent
Example how the calculation is done in Table \ref{tbl-ngram-sd} is given below.
\begin{example}
To get the value 0.18 for a term \texttt{ASSIGN APPLY} in document A one sees first that the value of $tf$ is 1 from Table \ref{tbl-ngram-sd}. The $idf$ is formed by dividing number of documents with the number of total term frequency over all documents, and taking a logarithm \ie $idf = \log(N/df) = \log((1+3)/(1+2)) + 1 \approx 1.29$. Note that we add extra ones to avoid division with zero and to diminish the effect of terms appearing only in training set. Now $tf{\text -}idf$ is simply $tf \cdot idf = 1 \cdot 1.29 = 1.29$. Finally, after calculating non-normalized weight for each term, we can derive the value 0.18 dividing $1.29$ with the Euclidean norm over the weights which gives $tf{\text -}idf_{norm} = 1.29 / \! \norm{\bolditt{w}}_2 = 1.29 / 6.98 = 0.18$
\end{example}

The vocabulary $\mathbb{V}$ that forms the set of possible tokens in Table \ref{tbl-ngram-sd}, is the union between every token appearing in three example documents \ie $\mathbb{V} = \bigcup_{i=1}^{3} V_i$ where $V_i = \{t_1, t_2, \cdots, t_n\}$. Therefore some terms may appear zero like the term \texttt{METHOD\{ APPLY} for document A in Table \ref{tbl-ngram-sd}, as it exists only in the token stream of document C. However, this problem with terms appearing zero times is only present in authorship identification, where the main corpus is the training set that is needed to train the model. Therefore even though there can be some terms that appears only in training set and not in test set, the smoothing we apply also in Equation \ref{eq-laplace} prevents the complete product to become zero. Terms like \texttt{ASSIGN VARDEF} and \texttt{VARDEF ASSIGN} have a high weight as they mostly appear in document A, implying that document A has more variable assignments than document B or C, which is true when one looks at the raw source code documents. 

With our approach, we can now represent document as a vector of weights \eg document A as $\bolditt{x} = [0.14, 0.18, \cdots, 0.14, 0.14]$, and the dimension of $\bolditt{x}$ is the size of vocabulary $\mathbb{V}$. The visualization of these three programs as vectors of weights is seen in Figure \ref{fig-tfidf}, where it is clear that program C is the outlier whereas A and B share more similarities between each others.

\begin{figure}[!h]
\centering
\setlength\figureheight{7cm}
\setlength\figurewidth{10cm}
\input{plots/tfidf.tikz}

\caption{Three sample programs from Appendix \ref{appendix:programs} visualized in two dimensions. TF-IDF weights have been calculated from the token streams.} \label{fig-tfidf}
\end{figure}



\subsection{Similarity detection}


After submissions for a given exercise have been normalized to a token stream, we apply the \textproc{DetectSim} function of the Algorithm \ref{alg-toplvl}, which retrieves set of authors we call suspicious authors. These authors share a lot of structural similarity to each others within a given task, implying that there is a high chance that plagiarism might be occurred within this set. 

The \textproc{DetectSim} in other words is the similarity detection method of our study, where we first calculate the vector similarity to form a distance matrix $\bolditt{M}$. In this matrix $\bolditt{M}_{i,j}$ implicates the similarity between documents $d_i$ and $d_j$. Then, we calculate the similarity by using cosine similarity introduced in Chapter \ref{chap-bg-sim}, which was also extensively utilized by other studies in Chapter \ref{chap-liter-review-methods}. Lastly, we apply DBSCAN clustering to the values in similarity matrix $\bolditt{M}$ to form a groups of suspicious authors. The pseudocode for the \textproc{DetectSim} function can be seen in Algorithm \ref{alg-detectSim}.

\begin{algorithm}[ht]
\caption{Detecting suspicious authors.}
\label{alg-detectSim}
\begin{algorithmic}

\Require Set of authors $A$
\Require Set of documents $D$ belonging to authors $a \in A$
\Require Every document $d \in D$ is represented as a token stream
\Require Preferred length of word level $n$-grams $n \in \mathbb{N}$
\Require Minimum rate of similarity $\varepsilon \in [0, 1]$
\Assume $MinPts \gets 2$
\Procedure{DetectSim}{$A, D, n, \varepsilon$}
   \State $\bolditt{X} \gets$ \Call{ExctractNgrams}{$D, n$}
   \State $\bolditt{W} \gets$ \Call{TFIDF}{$\bolditt{X}$}
   \State $\bolditt{M} \gets$ \Call{COS}{$\bolditt{W}$}
   \State $\Omega \gets$ \Call{DBSCAN}{$\bolditt{M}, \varepsilon, MinPts$}
   \State \textbf{return} $\Omega$
\EndProcedure
\end{algorithmic}
\end{algorithm}

\noindent
The Algorithm \ref{alg-detectSim} is dependant from two parameters: length of $n$-grams and similarity threshold $\varepsilon$. These two hyperparameters are tuned with SOCO data set before the final evaluation. The overall flow of operations is following: first we extract all word level $n$-grams and turn the documents into raw term frequencies, then terms are weighted using TF-IDF and cosine similarity is calculated between every document. Finally DBSCAN clustering algorithm is used to form clusters of similar documents. Because we know every author of each document and we assume single authorship, these clusters are identical to clusters of authors. 

Note that the value of $MinPts$ is assumed to be value 2, as only two documents is needed to form a cluster of suspicious documents. This refers to real life situation where two students have shared source code between each others. 

\subsection{Authorship identification}

The second method we apply, is the author identification from a collection of source codes. Like in similarity detection, we apply this to one exercise at a time but as this model requires training, we define the training set to be all documents that the author has previously written. For example when considering a course which consist of seven weeks and the final exam, we can use all seven weeks per author to train the model \ie try to capture the preferred style of an author, and then predict for a random sample from a collection of exam submission that who is the most likely author.  

The algorithm for authorship identification we use is heavily based on the probabilistic model Naïve Bayes from Chapter \ref{chap-bg-classification}. We utilize $n$-grams which was a popular method among other studies in Chapter \ref{chap-liter-review-methods}, because it captures preferences that the author might have when writing a program by using character-level information. That is also why we don't apply a lot of normalization for authorship identification, as this information would be lost if too much transformation would be applied. The pseudocode for our authorship identification is seen in Algorithm \ref{alg-ai}.

% multi label?

\begin{algorithm}[ht]
\caption{Detecting author candidates for a source code.}
\label{alg-ai}
\begin{algorithmic}

\Require Set of authors $A$
\Require Set of documents $D$ belonging to authors $A$
\Require Index of the exercise under detection $i \in \mathbb{N}$
\Require Length of character level $n$-grams $n \in \mathbb{N}$
\Procedure{TrainAndPredictAuthor}{$A, D, i, n$}
   \State $\bolditt{X} \gets$ \Call{ExctractNgrams}{$D, n$}
   \State $\bolditt{W} \gets$ \Call{TFIDF}{$\bolditt{X}$}
   \State $\bolditt{W}_{train}, \bolditt{y}_{train}, \bolditt{W}_{test}, \bolditt{y}_{test} \gets$ \Call{Split}{$\bolditt{W}, A, i$}
   \State $NB \gets$ \Call{TrainNaïveBayes}{$\bolditt{W}_{train}, \bolditt{y}_{train}$}
   \State $A_{auth} \gets$ \Call{Predict}{$NB, \bolditt{W}_{test}$}
   \State \textbf{return} $A_{auth}$
\EndProcedure
\end{algorithmic}
\end{algorithm}

\noindent
The authorship identification in Algorithm \ref{alg-ai} is dependent from one hyperparameter, which is the length of character level $n$-grams. The value for it will be tuned using both data sets OHPE and OHJA, and choosing the value which performs best on average. 

The remaining flow of Algorithm \ref{alg-ai} is following. After the weight matrix $\bolditt{W}$ has been formed we split the data into training and test sets with appropriate classes $\bolditt{y}$, which indicates the authorship assignments. The split is done by treating the $i$th exercise as a test set, and everything before it as a training data. For example if the interest is the exam, which can be thought as the final task of the course, there are 135 exercises before it in OHPE and 79 in OHJA\footnote{Values 135 and 79 are after pair programming tasks are filtered out from both sets.} that can be used to capture the individual style of an author. The appropriate training data is given to the Naïve Bayes algorithm in \textproc{TrainNaïveBayes}, which theoretical background is given in Chapter \ref{chap-bg-classification}. The training of the Naïve Bayes algorithm allows therefore to estimate the probabilistic parameters inside the model, and the function \textproc{Predict} being the maximum a posteriori probability (MAP) estimate, is able make the author prediction.

% ohpe exam 4 parts, 135 tasks before, removed pair programming
% ohja exam 3 parts, 79 tasks before, removed pair and course feedback





%\begin{figure}[!h]
%\centering
%\setlength\figureheight{5cm}
%\setlength\figurewidth{8cm}
%\input{plots/fig.tikz}
%
%\caption{TEST} \label{fig:M1}
%\end{figure}

%use t-SNE for visualization!! density is lost https://stats.stackexchange.com/questions/263539/k-means-clustering-on-the-output-of-t-sne?utm_medium=organic&utm_source=google_rich_qa&utm_campaign=google_rich_qa


\subsection{Evaluation} \label{chap-method-evaluation}

We first introduce a metric called \emph{accuracy}, which can be used in both binary and multiclass evaluation. Accuracy score, simply being the fraction between correct predictions and the total number of predictions, can be defined by using the confusion matrix given in Chapter \ref{chap-bg-sim} as

\begin{equation}
    ACC = \dfrac{TP + TN}{TP + TN + FP + FN}
\end{equation}

Both models of our approach are first evaluated against the data. This means that the similarity detection part uses SOCO to tune its parameters ($n$-gram and $\varepsilon$) and evaluate the performance of our model. Evaluation happens by reporting average precision and $F_1$-metric of document retrieval, and we mainly focus on the amount of correctly classified documents retrieved. After the hyperparameters for the similarity detection have been tuned, we compare it by calculating the agreement to the state of the art software plagiarism detection called JPlag \cite{prechelt2002finding}. The agreement with respect to the JPlag is based on the Jaccard similarity, which was given in Chapter \ref{chap-liter-review-methods} between two sets and we expect Jaccard similarity to be close to one, as our methods should get similar results as the JPlag. However, we can't say for sure that did JPlag retrieve all possible cases of plagiarism, thus we don't have direct access to true classes without going through every possible document in OHPE/OHJA. This means we can't calculate precision nor recall for the plagiarism detection, and we must resort to human judgement to base our final evaluation. 

%$A, B$ as

%\begin{equation}
%    J(A,B) = sim(A,B) = \dfrac{|A \cap B|}{|A \cup B|}
%\end{equation}





When evaluating the authorship identification, our classification problem is no longer binary. It's a multiclass classification problem, and in order to use $F_1$-score, it must be redefined. The multiclass-version of the $F_1$-score, which treats all classes equally, is called \emph{macro-averaged $F_1$} \cite{SOKOLOVA2009427}. It's defined as 

\begin{equation}
    F_M = 2 \cdot \dfrac{\text{Precision}_M \cdot \text{Recall}_M}{\text{Precision}_M + \text{Recall}_M}
\end{equation}

\noindent
Where $\text{Precision}_M$ and $\text{Recall}_M$ are averaged over every class as

\begin{equation}
    \text{Precision}_M = \dfrac{\sum \limits_{c \in \mathbb{C}}
        \frac{TP_c}
             {TP_c + FP_c}}
    {|\mathbb{C}|}
\end{equation}

\begin{equation}
    \text{Recall}_M = \dfrac{\sum \limits_{c \in \mathbb{C}}
        \frac{TP_c}
             {TP_c + FN_c}}
    {|\mathbb{C}|}
\end{equation}



\noindent
Using above metrics for a multiclass classification, we are able to tune the parameter $n$ which controls the length of character-level $n$-grams. Tuning is done by first dividing both OHPE and OHJA into seven splits which corresponds each week, then taking the set of authors who submitted and using their previous work as a training data, finally predicting the authors of the last exercises and collecting calculating the average performance. For example, when we evaluate our authorship identification on the first week of OHPE, we take the subset of authors $A' \subseteq A$ who submitted to the last exercise of the first week. Then the last exercise is left out as the test data, and for each author $a \in A'$, we collect their submissions to form the training data. 


Our final result will be a set of detected documents for both OHPE's and OHJA's exam tasks, and we will use a human expert who manually goes through the retrieved documents and classifies which ones she considers as real plagiarism. By using a human judgement, we get as unbiased and realistic evaluation as possible, but also information about the decision process. When the human expert has gone through all documents and evaluated them, we calculate following four metrics to score our final result: number of true positives, number of false positives, detected cluster sizes and Jaccard similarity. 



\newpage