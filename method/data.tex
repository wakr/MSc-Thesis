\subsection{Data set}

Our approach is aimed to the traditional programming MOOC setting which is for example used by undergraduate-level programming courses \emph{Introduction to Programming} (OHPE) and \emph{Advanced course in Programming} (OHJA) in University of Helsinki. We use three authentic data sets in Java language; students submissions done to both of latter courses during the implementation in fall 2016 and data from SOCO task from 2014. Both OHPE and OHJA includes proven cases of plagiarism, however to avoid any bias more specific information about them is kept hidden as a golden standard until the final evaluation. SOCO's training dataset on the other hand, contains prelabeled document pairs that have conducted plagiarism which all have been discovered by human experts, and test dataset pairs labeled by majority of the submitted competitors \cite{saez2014pan}.

To implement our models, we first use SOCO to train and evaluate our similarity detection model, then train and test authorship identification with OHPE and OHJA. Our proposed model is built based on these results and plagiarism is detected individually for both courses. The reason to use SOCO for similarity detection is simply that it's the only data set that contains fully labeled cases of plagiarism. OHPE and OHJA contains multiple files per author making author identification possible, but they hide the plagiarism cases. Therefore we make use of both sets and consider our model to be successful if it has a high precision, minimizing the amount of false-positives \ie false plagiarism accusations, and maximizing the amount of true-positives \ie true plagiarists. As this setting means that we need a high precision and a high recall, we resort to balancing between these metrics using the $F_1$-score. 

\paragraph{Course overview}\mbox{}\\
OHPE and OHJA shares the same structure; students first register to automatic scoring system called \emph{Test My Code} (TMC) \cite{Vihavainen:2013:SSL:2462476.2462501} which also distributes the exercises as an plugin to \emph{NetBeans IDE}, then independently work during seven weeks by completing programming exercises within deadlines \cite{Vihavainen:2012:MSM:2380552.2380603}. Both of these courses follow \emph{Extreme Apprenticeship method} \cite{Vihavainen:2011:EAM:1953163.1953196}; theoretical material is available online for students, students learn by doing \ie there exists mandatory programming exercises, weekly exercise sessions are available for those who require assistance, instructors can give feedback and students are able to track their process. In addition, there are exercises in which students are required to have a pair to program with. This is referred as \emph{pair programming}.

Students earn points from exercises depending if all tests were successfully passed via TMC, and complete an exam at end of the course which is a programming exam that ultimately decides if a student has learned the minimum level of knowledge required to pass the course. The exam in fall 2016 was a home exam, meaning that students were able to do it individually wherever they wanted to. As there were no mandatory lectures, students were able to pass the whole course working individually without any physical attendance. Because there was a freedom to do the course wherever the students want, students were told at the beginning of the course that plagiarism is prohibited.

\paragraph{SOCO overview}\mbox{}\\
Source code reuse (SOCO) data is from a 2014 competition \emph{PAN@FIRE}, where two sets were given to detect monolingual source code re-use \cite{saez2014pan}. SOCO2014 offered a train and a test set for competitors, which contained files written in \cpp\, and Java by various authors. The train set contains source code files and annotations which are made by three experts flagging pairs that are considered as plagiarized. Test set on the other hand, contains six individual scenarios labeled by majority voting from multiple submissions. The competitors were asked to retrieve which pairs are plagiarized, but the direction was completely ignored, meaning that they didn't have to show who was the plagiarist and who was the sharer.  

SOCO contains mainly submissions to a single exercise and documents, that are transformed from C to Java \cite{saez2014pan}. As only the plagiarized file pairs are annotated and SOCO has been used successfully used in other studies \cite{AIR2015, RCISCP2017, OTIOLSS2015, USCR2014}, we
will use SOCO to train and evaluate our similarity detection model. The number of authors is not explicitly reported in SOCO, so we make a simplifying assumption that there exist one file per one author and also that all submissions are for the same task. 



\paragraph{Corpus statistics}\mbox{}\\
As we are going to focus strictly to Java language, we only use the Java-specific part of SOCO, whereas OHPE and OHJA are fully utilized because they only contain Java files. Some non-transformative steps has been made beforehand to both OHPE and OHJA; exams are added to data set and submission containing multiple files are concatenated into one file. This allows us to assume also in OHPE and OHJA that there exist one file per submission, and we also get the benefit of having exam submissions where plagiarism is absolutely not allowed. Statistics for all these three corpora are reported in Table \ref{tbl-corporastats}, only applying modifications given above.




\begin{table}[ht]
\centering
\caption{Descriptive statistics for the unprocessed corpora. SOCO has been divided into three related corpora: train (T), test scenario without plagiarism (C1) and test scenario with plagiarism (C2). Bold values represents maximum value per metric.}
\label{tbl-corporastats}
\scalebox{0.8}{
    \begin{tabular}{|l||c|c|c|c|c|} \hline
    \backslashbox{\bf Metric}{\bf Corpus}  & SOCO-T & SOCO-C1 & SOCO-C2 & OHPE & OHJA\\  \hhline{|=|=|=|=|=|=|}
    \textbf{Authors}         & 259 & 124 & 88 & \textbf{316} & 270   \\  \hline
    \textbf{Exercises}       & 1 & 1 & 1 & \textbf{151} & 92     \\  \hline
    \textbf{Documents}       & 259 & 124 & 88 & \textbf{33\,363} & 15\,196    \\  \hline
    \textbf{Documents per author AVG.} & 1 & 1 & 1 & \textbf{106} & 56\\ \hline
    \textbf{Synthetic}       & Partly & Partly & Partly & No & No \\  \hline
    \textbf{LOC $\min$}         & \textbf{12} & 7 & 7 & 1 & 1      \\  \hline
    \textbf{LOC AVG.}        & 149 & \textbf{155} & 144 & 44 & 109     \\  \hline
    \textbf{LOC $\max$}         & \textbf{1696} & 1398 & 661 & 679 & 637   \\  \hline
    \textbf{Expression AVG.}       & 63 & \textbf{76} & 67 & 17 & 38 \\ \hline
    \textbf{Character AVG.} & \textbf{3898} & 3848 & 3751 & 1139 & 2794   \\  \hline
    \end{tabular}
}
\end{table}

\noindent
 Table \ref{tbl-corporastats} reports ten different metrics: number of total authors, exercises and documents; does the corpus contains synthetic data; averages for documents per author, character count, lines of code (LOC) and expressions ending to semicolon; and lastly minimum and maximum line counts. We can see from the Table \ref{tbl-corporastats}, that SOCO has the smallest amount of authors but the tasks are more complex indicated by the largest LOC, amount of expressions and the average character count. When comparing OHPE to OHJA, OHPE has relatively smaller submissions than OHJA, which is mostly due to OHPE having easier tasks due to being the introductory course where students are not expected to know anything about programming beforehand. OHPE also has the most largest average document-to-author ratio (106) compared to SOCO (1) and OHJA (56), making it the most richest data set when it comes to having a large amount of submissions per author.  Comparing to other corpora presented in chapter \ref{subsec-liter-data}, our OHPE corpus is one of the largest with OHJA. They both have over four times as many authors than any of the corpora used in other studies.

A problem however arises when average line count with respect to the exercises is visualized for both OHPE and OHJA. Figure \ref{fig-hists} visualizes this by histograms, where bin sizes are set to 50. 


\begin{figure}[ht]
\centering
\captionsetup[subfigure]{justification=centering}

\begin{subfigure}{\textwidth}
    \setlength\figureheight{4cm}
    \setlength\figurewidth{\textwidth}
    \input{plots/ohpe_avgloc.tikz}
    \label{fig-ohpeavgloc}
\end{subfigure}

\begin{subfigure}{\textwidth}
  \setlength\figureheight{4cm}
    \setlength\figurewidth{\textwidth}
    \input{plots/ohja_avgloc.tikz}
    \label{fig-ohjaavgloc}
\end{subfigure}

\caption[Two histograms for corpora]{Histograms showing average line of count per exercise for OHPE (top) and OHJA (below). OHJA has more evenly distributed length of submissions, where as OHPE's submissions are mostly under 100 lines in length.}
\label{fig-hists}
\end{figure}

\noindent
From Figure \ref{fig-hists} we see that majority of the submissions for OHPE has under 100 lines of code. This can create an issue, as there exists tasks where the submission can only contain a few dozen lines meaning that the similarities between solutions will be naturally high as the solution spaces of these tasks are very limited. 


% --------------- Unnecessary part, as detection only occurs for exam
\iffalse
The data supports this claim, as the mean length of every weeks last exercise submissions in OHPE is seen in Table \ref{tbl-OHPE-last-week}.

\begin{table}[ht]
\centering
\caption{Average line count for submission of the final exercise of each week for OHPE. The only outlier is the last weeks exercise.}
\label{tbl-OHPE-last-week}
\begin{tabular}{l|c|c|c|c|c|c|c}
\bf Week        & 1.  & 2.  & 3.   & 4.  & 5.   & 6.   & 7.   \\ \hline
\bf Average LOC & 71 & 66 & 149 & 95 & 146 & 206 & 123 \\ \hline
\bf No. longest & 2nd  & 1st & 1st  & 1st & 1st   & 1st   & 4th  
\end{tabular}
\end{table}

% TODO: Add OHJA ^

\noindent
In Table \ref{tbl-OHPE-last-week} we don't take in account pair exercises which are meant to be done with another student, and we leave them completely out from the corpus as they violate our single author assumption. Pair programming tasks are quite sparse as OHPE contains 12 pair programming tasks out of 151 and OHJA 10 out of 92.
\fi